\documentclass[10pt]{hedrabook}
\usepackage[brazilian,polutonikogreek]{babel}
\usepackage{ucs}
\usepackage[utf8x]{inputenc}
\usepackage[svn,center,cam,a5]{hedracrop}
\usepackage{hedrabolsolayout,hedraextra}
\usepackage[chapterdot]{hedratoc}
\usepackage[protrusion=true,expansion]{microtype}
\usepackage{comment,lipsum,footmisc}
\usepackage{gfsporson,relsize}
\usepackage[minionint,mathlf]{MinionPro}
%\usepackage{gmverse}
%\usepackage{hedraebook}		   % hyperlinks: todos os \part têm que estar em CAIXA BAIXA
%\usepackage{makeidx,hedraindex}  % cria índice
%\makeindex			   % ...  índice

\begin{document}
\SVN $Id: MILL.tex 13087 2014-06-18 18:56:04Z jorge $


\selectlanguage{brazilian}

\newenvironment{grk}[1]{\selectlanguage{polutonikogreek}\relsize{-1}\textporson{#1}\selectlanguage{brazilian}\relsize{1}}{\selectlanguage{brazilian}\relsize{1}}

\title{Sobre a liberdade} %O príncipe
\author{Mill} %Maquiavel
\begingroup\pagestyle{empty}
%\begin{blackpages}
	\maketitle
	\begin{techpage}{5cm}
		\vspace{-1.5cm}
		\putline{Copyright}{Hedra \the\year}
		\putline{Tradução$^\copyright$}{Ari R.~Tank Brito} 
%		\putline{Título original}{}
%		\putline{Edição consultada}{\emph{ }, }
%		\putline{Primeira edição}
%		\putline{Indicação}{}
%		\putline{Agradecimento}
		\putline{Corpo editorial}{
			Adriano Scatolin,\\
			Alexandre B.~de Souza,\\
			Bruno Costa,
			Caio Gagliardi,\\
			Fábio Mantegari,
			Iuri Pereira,\\
			Jorge Sallum,
			Oliver Tolle,\\
 		    Ricardo Musse,
			Ricardo Valle
		}
		\ \\

		\putline{Dados}{\fichacatalografica{M580}% Código de autor
				% Autor (data nascimento--morte)
				{Mill, Stuart (1806–1873)}% 
				% Dados bibliográficos
				{Sobre a liberdade. / Stuart Mill. 
				Tradução e organização de Ari R.~Tank. 
				– São Paulo: Hedra, 2010.}%
				% Isbn
				{978-85-7715-200-1}%
				% Catalogação. Ex. 1.~Blabla.(Não esquecer ~)
				{1.~Filosofia. 
				2.~Filosofia Inglesa. 3.~Mill, John Stuart (1806--1873). 4.~Pensamento Liberal. 5.~Liberdade. 
				\textsc{i}.~Título. 
				\textsc{ii}.~Brito, Ari R.~Tank, Tradutor. \textsc{iii}.~Brito, Ari R.~Tank, Organizador.}%
				%CDU
				{123}
				%CDD 
				{123} %
			       }
		\direitos
		\dadoseditora
		\depositolegal
	\end{techpage}
	\begin{frontispiciopage}{4cm}{}	
%	\putline{\hspace{12ex}Tradução}{\textsc{   }} %{Introdução|Organização|...}
%	\putline{\hspace{12ex}Introdução}{\textsc{   }} %{Introdução|Organização|...}
   \putline{\hspace{12ex}Organização e tradução}{\textsc{Ari R.~Tank Brito}}   %{Introdução|Organização|...}
	\end{frontispiciopage}
%Jorge: Teste
\SVN $Id: PRETAS.tex 7796 2010-11-24 19:29:08Z oliveira $
\begin{resumopage}

\item[John Stuart Mill] (Londres, 1806---Avignon, 1873), filósofo, 
economista e parlamentar inglês, é um dos mais importantes pensadores 
liberais do século \textsc{xix}. Na infância foi submetido a uma 
educação rigorosa por seu pai, James Mill, e posteriormente por 
Jeremy Bentham, fundador do utilitarismo. Rapidamente Mill 
reconhecerá no utilitarismo clássico uma ameaça à liberdade 
individual e empreenderá uma profunda reformulação em seus 
princípios, sem todavia jamais deixar de ser utilitarista. 
Conhece a obra de Auguste Comte em 1828, com o qual manterá 
uma intensa correspondência entre os anos de 1841 e 1846. 
Em 1851 desposa Harriet Taylor, famosa defensora dos direitos 
da mulher, e com a qual tinha mantido até então mais de vinte 
anos de amizade. Dentre as inúmeras obras que publicou vale 
destacar: \textit{Sistema de lógica dedutiva} (1843), 
\textit{Princípios de economia política} (1848) e \textit{Utilitarismo} (1861). 
Fortemente engajado em movimentos de emancipação da mulher, 
em 1865 é eleito para a Câmara dos Comuns, a qual será dissolvida 
em 1868, obrigando Mill a mudar para o sul da França, em Avignon. 

\item[Sobre a liberdade] (On Liberty, 1859) é uma defesa da individualidade 
e de sua autonomia diante da sociedade e do Estado. Para Mill, a “tirania 
da maioria” impõe uma homogeneidade que não é saudável para o desenvolvimento 
da sociedade. Os indivíduos são livres para fazerem o que quiserem, desde 
que não causem prejuízo aos seus semelhantes.

\item[Ari R. Tank Brito] é mestre em Filosofia pela Universidade 
de Varsóvia, Polônia (1990) e doutor pela Universidade de São Paulo (2007). 
Atualmente é professor da Universidade Federal do Mato Grosso. Organizou 
e traduziu, para a Coleção de Bolso Hedra, a \textit{Carta sobre a tolerância}, de John Locke. 
\end{resumopage}



% Coloca página preta vazia à direita se texto preencher apenas uma página
\ifodd\thepage\paginabranca\fi

%\end{blackpages}
\endgroup
\setcounter{tocdepth}{0}     % amplitude da presença das partes no índice
\setcounter{secnumdepth}{-2} % amplitude da numeração das partes

\hedratoc

% Favor não alterar o segundo parâmetro (\baselineskip). 
% Para acertar entrelinhas, usar comando \linespread
\fontsize{10.5pt}{\baselineskip}\selectfont
\baselineskip=12.6pt   % Parâmetro válido apenas para corpo 10.5, para outros, acrescer 20% do valor do corpo%%



\chapter[Introdução, por Ari R. Tank Brito]{introdução}
\hedramarkboth{Introdução}{Ari R. Tank Brito}

\noindent\textsc{O filósofo} britânico John Stuart Mill escreveu
extensamente sobre todos os assuntos tidos como importantes e ganhou em
vida a reputação de sábio. Também foi portador de uma
personalidade excepcional, austera e justa, sendo chamado de ``Santo do
Utilitarismo'' por um importante e sagaz político britânico da época. O
lugar que Mill obteve no panteão dos sábios e dos santos laicos, que
isso fique claro desde o início, é absolutamente merecido. Ele foi um
dos principais filósofos de sua época e um dos maiores pensadores
liberais de todos os tempos. E muito embora a sua vida pessoal esteja
repleta de peculiaridades, hoje em dia dificilmente compreensíveis,
pode"-se perceber nela um comprometimento com a causa da liberdade
humana que poucos jamais tiveram, seja antes ou depois dele. 

Talvez a chave desse comprometimento de John Mill resida na sua
formação, uma vez que foi educado para ser um gênio desde a mais tenra
infância, no que teve sucesso, embora à custa de muito sofrimento
pessoal. Todo o processo teve lugar em seu próprio lar: seu pai, o
também filósofo James Mill (1773--1836), assumiu a sua educação
pessoalmente, seguindo um rígido processo de aprendizado. Aos dez anos,
Mill já lia e escrevia em grego antigo e em latim, estudando obras que
mesmo adultos achariam difíceis e complicadas, como, por exemplo, os diálogos de
Platão. Ao estudo de autores antigos acrescente"-se o estudo de
matemática, direito, história, lógica, economia, ciências, além de uma
bela dose de literatura. Não se pode afirmar que John Mill teve uma
infância feliz e alegre, e que simplesmente não teve infância nenhuma
talvez não seja uma afirmação exagerada. Não bastasse ter de seguir o
estafante esquema de seu pai, com aulas de manhã até a noite, tão logo
John Stuart Mill chegou à juventude, um outro preceptor lhe foi dado, o
fundador do Utilitarismo, Jeremy Bentham (1748--1832), amigo de seu
pai. Esses dois, James Mill e Bentham, fizeram o melhor que puderam
para transformar o menino Mill em perfeito utilitarista. Durante um bom
tempo pareceu que tinham conseguido uma proeza: a criação, através de um
rígido esquema pedagógico, de uma mente inteligente, fria e racional.
Nada foi esquecido, no que dizia respeito a inculcar na cabeça de John
Mill não apenas os conteúdos programáticos da doutrina utilitarista,
mas também um imenso cabedal de conhecimentos que facilitasse a defesa
dessa doutrina. Aos treze anos, Mill estava de posse de uma instrução
universitária completa de sua época, com a diferença de que não tinha
sido imposta nenhuma fé religiosa, pois foi educado para ser um
ateu, assim como eram seus preceptores. 

A criatura não decepcionou por completo os criadores: durante toda a sua
vida John Mill continuou sendo um utilitarista. O que aconteceu é que a
versão do utilitarismo que lhe foi inculcada acabou por
decepcioná"-lo. Isso resultou numa revolta sentimental e intelectual,
seguida de uma profunda busca por outras bases filosóficas para o
utilitarismo, diferentes daquelas propostas por seu pai James Mill e
por seu preceptor Jeremy Bentham. 

O princípio básico do utilitarismo é que ``a maior felicidade para
o maior número (de pessoas) é o fundamento da moral e da legislação'',
princípio que Bentham generosa e honestamente reputou a escritores
anteriores como o jurista italiano Beccaria e o homem de ciências
britânico Priestley, mas que apenas com ele, Bentham, tornou"-se
absolutamente relevante. Este assim chamado Grande Princípio da
Felicidade, ou o Princípio da Utilidade, formulado em sua obra
\textit{Princípios da moral e da legislação}, de 1789, constituiu o
ponto de partida sobre o qual ergueu"-se todo um sistema de normas e
diretrizes, tidas como racionais e que, se seguidas, deveriam levar a
um bom sistema político, a uma boa vida social, além de resultados
correlatos, como bons asilos para os pobres e boas prisões para os
malfeitores. Pois, para Bentham, a sua filosofia utilitarista era um
projeto de reforma de caráter universal, sendo o próprio Bentham o
legislador de que o mundo precisava para ser posto em eixos racionais.
E ele bem que tentou passar adiante seus extensos planos de reformas,
se esforçando para que os governantes do seu e de outros países
pusessem em prática as ideias utilitaristas. Bentham sequer chegou
perto de ter o sucesso prático que almejava, mas no campo do Direito
exerceu uma influência até hoje sentida. Não apenas na forma de se
entender o que vem a ser o Direito, a Lei, mas também na prática, em
forma de legislação, particularmente a penal, o que não deixa de ser um
sucesso para alguém que tinha como profissão a advocacia. Mas é
irônico que o defensor de um princípio à primeira vista tão
liberal como o do utilitarismo se tornasse conhecido principalmente por
propor um tipo de edifício em que tudo o que viesse a acontecer
estivesse sob a vigilância de guardas. O \textit{Panopticon} (``onde
tudo é visto'') proposto por Bentham foi primeiro percebido como um
prédio especialmente construído para vigiar incessantemente seus
ocupantes, isto é, os prisioneiros, sem que estes pudessem ter um
instante de privacidade. A arquitetura peculiar desse edifício, e
deve"-se anotar que penitenciárias foram de fato construídas segundo
as prescrições de Bentham, com sua torre central e suas celas
direcionadas para esta torre, não é tão interessante como a sua ideia
central: a de que os seres humanos devem ser vigiados o tempo todo, sob
pena de se rebelarem ou não trabalharem a contento. A intenção de
Bentham era, pelas suas luzes, humanitária. Vigiados, os prisioneiros
não se rebelariam, o que sempre acarretava perdas de vidas e
propriedades, e trabalhariam com afinco, escapando assim das garras da
preguiça, a principal causa de seus erros e de sua miserável situação.
Vigiados para não cometer tolices e obrigados a trabalhar, os marginais
(bandidos e pobres, pois o Panopticon deveria servir tanto para cadeias
como para os lugares onde os moradores de rua de então eram deslocados)
deixariam de ser perniciosos para a sociedade. Sendo assim, a
felicidade geral aumentaria consideravelmente. Pois, numa outra
formulação do Princípio da Utilidade, encontrada em uma obra de
Bentham publicada em 1776, ``a felicidade do maior número 
é a medida do certo e do errado''. 

Infelizmente para Bentham e os defensores do Panopticon, a realidade
revelou"-se bem outra: o controle total prometido, ao ser aplicado,
tornava as coisas ainda piores, já que os prisioneiros, sem privacidade
nenhuma em suas vidas, apelavam para ações de resistência passiva para
as quais seus vigias não estavam preparados, isto é, deixavam de
cumprir o que lhes era exigido. Também as rebeliões não deixaram de
acontecer, não obstante toda a vigilância. O Panopticon, tal como
proposto por Bentham, não é útil. Mesmo que prisioneiras, as pessoas
devem ter certo grau de privacidade, de liberdade mesmo. Mas o que
levou Bentham a propor um esquema assim? 

O utilitarismo que surge nos finais do século \textsc{xviii}, o Século das Luzes,
mantém e intensifica características das ideias de \textit{philosophes}
como Voltaire, Diderot e D'Alembert. Ele é racional, antirreligioso,
cientificista e contrário aos poderes constituídos. Duas
especificidades utilitaristas são fundamentais para uma boa compreensão
da obra \textit{Sobre a liberdade} de Mill: a posição tomada contra os
direitos naturais, que separa essa obra dos movimentos que levaram à
Independência dos Estados Unidos e à Revolução Francesa, e o método que
levaria a maior felicidade ao maior número de pessoas, o ``Cálculo
Felicífico''. Ser contra os direitos naturais significava ser contra a
suposição de que os direitos humanos existissem anteriormente a
qualquer Lei que os tivesse criado. Não havendo Deus, como propunham os
utilitaristas, e não dando a Natureza nenhum direito aos homens, os
direitos humanos só poderiam existir se uma lei positiva, feita pelos
homens dentro de um contexto legal, assim o proclamasse. A Declaração
de Independência dos Estados Unidos, que contém a célebre frase
``Mantemos ser autoevidente que todos os homens são criados
iguais, e que são agraciados pelo seu Criador com certos direitos
inalienáveis, entre os quais estão o direito à vida, à liberdade e à
busca da felicidade'', e a Declaração dos Direitos do Homem proclamada
pelos revolucionários franceses, não passavam, para Bentham, de exemplos
do pomposo absurdo que era a doutrina dos direitos humanos naturais. O
que significa, entre outras coisas, que Mill não poderia, para defender
a causa da liberdade, utilizar nenhum argumento que se
relacionasse com a suposição de que a liberdade é um direito natural do
ser humano. Isso não poderia ser feito sem abandonar o utilitarismo
como um todo. Os argumentos em \textit{Sobre a liberdade} têm e só
podem ter um caráter utilitarista. O que foi conseguido com a
introdução de mudanças de monta no segundo aspecto acima mencionado, o
do Cálculo Felicífico. Como medir a felicidade? Para Bentham, a questão
se limitava a fugir da dor e se aproximar do prazer, de forma
quantitativa. Mais dor, mais infelicidade. John Mill, ao contrário,
praticamente liquida com a possibilidade de se calcular matematicamente
o índice de felicidade de uma pessoa, com a introdução do aspecto
qualitativo dessa felicidade. Não é só importante o quanto uma pessoa é
feliz, o quanto ela está afastada das dores e próxima dos prazeres, mas,
e principalmente, como esta felicidade está construída, isto é, se ela é
de qualidade ou não. Afinal, como colocou John Mill, ``é melhor
ser um Sócrates insatisfeito do que um homem satisfeito''. Não se trata
de poder calcular o quanto de felicidade um homem satisfeito teria a
mais que um filósofo insatisfeito, mas sim de declarar que a posição do
segundo é melhor, isto é, qualitativamente superior à do primeiro. O
que vem a significar que muitas vezes é melhor ser infeliz do que
feliz, uma proposição que não teria sentido para um utilitarista ortodoxo. 

A introdução de critérios qualitativos para aferir não se alguém é mais
ou menos feliz, mas sim se vive uma vida melhor ou pior, poderia levar
a crer que Mill tivesse tornado o utilitarismo inconsistente, em
desacordo consigo próprio, e não faltaram ou faltam críticos para
apontar as possíveis inconsistências que apareceram no tipo de
utilitarismo proposto por Mill. Se esses críticos têm ou não razão é
uma questão espinhosa, e talvez insolúvel. Cabe lembrar apenas que
para Mill, autor de uma \textit{Lógica} que fez muito sucesso, a sua
posição não era nem ilógica nem incompatível com os princípios
utilitaristas. Pelo contrário, considerava as alterações necessárias
para que o utilitarismo pudesse dar conta dos problemas que tratava. 

Como se viu anteriormente, ao defender amplas reformas na legislação, o
utilitarismo envolveu"-se na política. Passada a época em que Jeremy
Bentham ainda acreditava que as classes governantes aceitariam ser os
responsáveis por essas reformas, os utilitaristas se tornaram um grupo
político específico, independente tanto dos Tories (conservadores)
quanto dos Whigs (liberais). Eles denominaram a si mesmos de Radicais
Filosóficos; e era como um Radical que John Mill via a si mesmo, e como
Radical ele foi durante algum tempo membro do Parlamento britânico. As
palavras mudam de sentido, o que não deixou de acontecer com a palavra
``radical'': longe de querer implicar mudanças de caráter socialista, o
que os utilitaristas propunham com radical significou na prática
reformas no sistema capitalista de então para torná"-lo mais racional
e eficiente. Embora essas propostas tivessem um viés democrático, como, 
por exemplo, propor que a cada homem correspondesse um voto, isso
quando na Grã"-Bretanha de então a população simplesmente não tinha
direito de voto, elas carregavam consigo um pendor antipopular. Os
utilitaristas, os Radicais, se propunham ser a elite intelectual de que,
segundo eles, a nação precisava, sem que de fato atentassem ao país
como esse realmente era. Tudo o que não era racional e utilitário era
tido como antiquado e obviamente destinado a ser destruído e
substituído por algo mais científico. Pode"-se dizer que os Radicais
queriam fazer a parte que achavam ter sido a dos Girondinos na
Revolução Francesa, sem promover a efusão de sangue trazida pelo Terror
jacobino. E isso sem se ligar a ``falácias anárquicas'', como
consideravam, por exemplo, os direitos humanos. O que também
significava não se importar com os costumes, os modos de vida e as
crenças religiosas da população. A reforma viria do alto, à massa
cabendo apenas seguir os líderes e aprender a se comportar. 

No entanto, nem a população em geral nem a maior parte das classes
dominantes se dispuseram a seguir completamente o receituário
utilitarista. Os que mandavam, os Lordes e os grandes capitalistas, não
trocaram os seus interesses de classe pelo seu ``interesse bem
compreendido'', preferindo o mais das vezes ou ser contra qualquer
reforma ou a favor de reformas paulatinas. Não se sentiam inclinados a
apoiar algo tão abstrato e incomum como as propostas radicais. Já o
povo, de modo geral, passou a ter medo dos utilitaristas, medo que se
percebe claramente expresso no romance de Charles Dickens
\textit{Tempos difíceis} (\textit{Hard Times}, 1854). Neste romance, a figura tétrica de um
utilitarista, o senhor Thomas Gradgrind --- ``Um homem de
realidades. Um homem de fatos e cálculos. Um homem que age sob o
princípio de que dois e dois são quatro, e nada mais [\ldots]'' ---, é utilizada por
Dickens para mostrar como o utilitarismo chegava à vida das pessoas:
sem coração e sem alma, já que o utilitarista só conseguia calcular, e
neste cálculo punha por terra o que a vida tinha de bom. Amor,
sentimentos, poesia, eram deixados de lado, substituídos por um infame
calcular de lucros e perdas, que se aplicava tanto à vida familiar como
às fábricas, onde os trabalhadores eram submetidos a rígidos e
extenuantes horários de serviço e a cotas de produção desumanas. Neste
mesmo romance, aliás, a educação dos filhos do senhor Gradgrind é uma
paródia da educação a que John Mill foi submetido, com ênfase na
memorização mecânica de dados exatos sobre as coisas. Talvez os
utilitaristas tivessem boas intenções, mas o julgamento que se fazia a
seu respeito era que queriam pavimentar o caminho para o inferno. 

Para John Mill, certa inexperiência nas coisas da vida, tanto de 
Bentham quanto de seu pai, James Mill, foi responsável diretamente 
por este mal"-estar e pelas propostas inexequíveis (mesmo em termos 
utilitaristas, já que tendiam a aumentar a infelicidade das pessoas)
que o geraram. Bentham, principalmente, era para John Stuart Mill um homem
de ideias claras, mas pouco amplas, já que para ter ideias claras sobre
assuntos complexos é necessário, como ele escreveu em um ensaio crítico
a respeito de seu ex"-professor (``Essay on Bentham'', 1838),
negar ``a completamente não analisável experiência da raça humana''.
Bentham, em suma, era um doutrinário, talvez mesmo um dogmático, e como
tal, via as coisas de modo parcial e incompleto.

No entanto, se Mill escreveu e publicou quando tinha 22 anos
uma crítica ao seu mestre Bentham, isso não quer dizer que 
ele tenha  deixado de ser um adepto do utilitarismo. Isso nunca ocorreu,
John Mill continuou a ser um utilitarista até o fim da sua vida. Ele
não se tornou um apóstata, mas sim um herético. Isto é, ele introduziu
mudanças significativas no que se conhecia como utilitarismo, mudanças essas
que, como se verá adiante, acabaram por tornar sua filosofia bem mais
complexa e bem menos \mbox{dogmática}. Essas mudanças não ocorreram apenas por
razões teóricas, mas por razões pessoais e profundas.

O que veio primeiro foi uma desilusão sentimental sobre o utilitarismo.
Quando muito jovem ainda, mas já seriamente empenhado em discutir e 
espalhar os princípios utilitaristas, Mill, conforme nos conta em sua
\textit{Autobiografia} (publicada postumamente em 1874), fez a si mesmo 
a seguinte pergunta: ``Suponha que todos os seus objetivos na vida foram realizados,
que todas as mudanças nas instituições e opiniões pelas quais você vem
se esforçando pudessem acontecer neste exato momento: Isso seria uma
grande alegria e felicidade para você?''. A resposta foi um triste
\textit{Não!}, e essa resposta, vinda de seu íntimo, mudou a sua vida.
Se a concretização dos ideais pelos quais lutava não o deixaria feliz,
nada mais poderia fazê"-lo. Com essa ``crise mental'', John Stuart Mill
se tornou então a pessoa mais triste do mundo, carregando o peso de ter
de se esforçar por coisas que, ele sabia, não iriam jamais
satisfazê"-lo. Não contou a seu pai ou seus amigos nada a respeito
dessa revelação ou constatação, que manteve por anos em segredo.

Um filósofo utilitarista que descobre que o utilitarismo, mesmo
funcionando, não o deixaria feliz, certamente tem um grande problema
nas mãos. A questão, para Mill, foi solucionada com um grande custo
pessoal. Exteriormente, ele continuou como antes, o jovem radical,
sempre disposto a estudar, a escrever e a polemizar, defendendo as
ideias básicas do utilitarismo. Interiormente, passou a viver sob uma
sombra, com os seus sofrimentos e dilemas dilacerando"-o aos poucos.
Já que não podia abandonar as ideias aprendidas, ele tomou o único
caminho possível, o de remodelá"-las. E o fez introduzindo suas
\mbox{angústias} e expectativas no inadequado programa dentro do qual e pelo
qual vinha lutando.

Foram vários os passos que John Mill teve que dar para sair de sua crise
mental, todos dolorosos. Intelectualmente, o mais difícil certamente
foi o de ter de admitir que alguns inimigos do utilitarismo não estavam
tão errados assim. A sua leitura do poeta e crítico romântico Coleridge
certamente é a mais sintomática --- Coleridge atacou as ideias de James
Mill por escrito, o que levou John Stuart Mill a defender seu pai
também por escrito. Contudo, as críticas de Coleridge várias vezes
acertaram o alvo, já que de fato o utilitarismo pecava por ser ingênuo,
inclusive simplório, quando tratava da vida social, da história, da
arte. John Mill estava preparado para reconhecer isso, já que ele
próprio era um apreciador da literatura romântica, a qual proporcionava
alívio para os seus sofrimentos. Como tudo aquilo que consola pode ter
alguma utilidade, já que nos afasta da dor, a poesia romântica (Jeremy
Bentham não apreciava nenhum tipo de poesia e James Mill apenas a poesia
greco"-latina) não podia ser descartada sem mais. 

Mas, como obrigatoriamente tinha de ser, quando se tratava de John
Stuart Mill, não era apenas o caso da poesia sentimental de cunho
romântico ter o dom de aliviar suas dores. O que os românticos pensavam
sobre a sociedade o fazia duvidar, refletir e tentar expor melhor suas
próprias ideias. Os românticos, de modo geral, eram o contrário dos
utilitaristas: sentimentais, tradicionalistas, intuicionistas e, muitas
vezes, com pendores revolucionários. O que John Mill aprendeu com eles
foi a expressar ideias que já tinha consigo, mas que não conseguiria
formular claramente se usasse apenas as categorias utilitaristas. 

Tendo sofrido uma crise pessoal e intelectual, a saída possível de Mill
teria que tentar contemplar ambos os aspectos. No campo pessoal, o
encontro com uma mulher, Elizabeth Harriet Taylor, significou o ponto
de mudança tão esperado. Ela sendo casada, John Mill teve de esperar
anos até que ela enviuvasse e eles pudessem se casar, o que parece
uma história banal, até que se observe que Mill e Harriet jamais
foram amantes e que a importância dela na sua vida não foi
estritamente sentimental, mas também teve um caráter filosófico e
político. Harriet era uma radical também, mas de cunho religioso,
enquanto Mill era um ateu convicto, mas essa diferença não impediu que
Mill se sentisse impelido a acompanhar Harriet em empreendimentos
reformistas. Mill teria sido menos ousado em suas ideias sem o
incentivo da amiga e depois esposa. E nunca deixou de reconhecer a
importância que ela teve na sua vida intelectual, como demonstra a
dedicatória de \textit{Sobre a liberdade}, a última obra de Mill, que
contou com a colaboração de Harriet. Há aqueles que alegam que a
influência de Harriet foi deletéria para o pensamento de Mill, que as
ideias utilitaristas dele não se harmonizaram bem com as ideias religiosas e
democráticas dela, mas esse julgamento é mais que falho, é tolo. Mesmo
que se admita que as ideias de Mill se tornaram inconsistentes porque
ele teve de lidar com aspectos que o utilitarismo clássico não
comportava, além de incorporá"-los em sua obra, a abertura
proporcionada ao pensamento de Mill, a amplitude que ele pôde
desenvolver nas suas ideias, seria mais que suficiente para compensar
alguma inconsistência baseada em parâmetros pequenos e demasiado
rígidos. É o que se percebe lendo \textit{Sobre a liberdade}, a sua
obra mais influente. Se a ortodoxia do utilitarismo de Mill pode ser
posta em dúvida, o resultado dessa possível heterodoxia é nitidamente
superior a qualquer coisa que outros utilitaristas escreveram
anteriormente sobre a política e a sociedade, como se pode perceber à
medida em que a obra é lida. 

Logo no primeiro capítulo, Mill
apresenta o seu famoso princípio da liberdade: 

\begin{hedraquote}
que o único fim
pelo qual se permite que a humanidade, coletiva ou individualmente,
interfira com a liberdade de ação de qualquer um dos seus números é a
autoproteção. Que o único propósito pelo qual o poder pode ser exercido
de forma justa sobre qualquer membro de uma comunidade civilizada,
contra a vontade dele, é o de prevenir danos aos outros.
\end{hedraquote}

Este princípio, longe de ser simples e facilmente inteligível, foi motivo de
discussões acirradas, grande parte das quais girando sobre a sua
coerência. Mas nem sempre se dá atenção ao detalhe de que este
princípio, tal como formulado, é um princípio de restrição da
liberdade, já que Mill trata da liberdade civil ou social e ``a
natureza e limites do poder que pode ser legitimamente exercido pela
sociedade sobre o indivíduo''. O princípio mesmo se divide em dois
aspectos, um em relação à humanidade como um todo, e o segundo em
relação a uma sociedade já organizada. Tanto no primeiro como no
segundo caso, vale a mesma restrição, mas no segundo trata"-se
da aplicação de um poder legal, enquanto no primeiro as condições de
aplicação não estão bem definidas, valendo para todas as sociedades
humanas. Vale também para ambos os casos a mesma pergunta, sempre
repetida quando se trata de discutir as bases e a amplitude deste
princípio: o que significa ``autoproteção'' e ``prevenir danos às outras
pessoas''? Num primeiro momento, trata"-se do direito à vida, o direito
à autopreservação e à preservação da vida de outrem, o exemplo
clássico. Mas é claro que é preciso ir muito mais além: Mill continua o
seu texto delineando os casos em que o seu princípio se aplicaria ou
não. Por exemplo, até quanto pode um pai de família gastar a sua renda
em bebidas? Não pode gastar muito, se esse gasto prejudicar o
bem"-estar dos seus. Um homem solteiro e sem ligações, ele já pode
gastar o que quiser, já que o único prejudicado é ele mesmo. A única
pena que Mill aceita neste caso é a reprovação moral da sociedade. De
exemplos assim saíram muitas discussões inflamadas, girando em torno da
candente questão sobre se esses e outros exemplos ampliariam ou não o grau de
liberdade de uma sociedade. Como se Mill estivesse se referindo a uma
futura sociedade permissiva, quando homens e mulheres viveriam ``cada um
na sua'', e não à moralista, religiosa e intransigente sociedade
vitoriana. E como se a reprovação moral fosse algo que se pudesse
ignorar com um dar de ombros, debaixo da proteção de leis que garantam
as diferenças de ``experiências de viver'', justamente as garantias que
não existiam e que Mill queria implementar. A questão da liberdade
seria a questão vital do futuro, não por ser uma novidade, mas pelas
condições já alcançadas pelas ``mais civilizadas partes'' do globo. 

Essa liberdade que não podia, exceto em circunstâncias bem definidas,
ser afetada, era uma constante histórica oriunda da Grécia e Roma
antigas e que chegava até a Inglaterra. A menção conjunta à Inglaterra,
Roma e Grécia não é ocasional, pois indica o caminho da liberdade para
Mill. No mundo antigo, liberdade era entendida como ``proteção
contra a tirania dos governantes''. Não se trata de uma diferença entre
a liberdade dos antigos e dos modernos, mas da mesma liberdade, a
liberdade de não ser oprimido. Mill mostra o caminho que foi percorrido
para controlar o poder dos governantes de oprimir os governados,
primeiro pelo reconhecimento de certas imunidades, as liberdades ou
direitos políticos, e pelo posterior estabelecimento de controles
constitucionais. O primeiro passo acabou sendo aceito pela maioria dos
países europeus, mas não o segundo, que se tornou em toda parte o
``principal interesse dos amantes da liberdade''. Como
essa segunda etapa já começa a ser implantada e fortalecida pelo menos
nas nações mais avançadas, o problema de não ser oprimido pelos
governantes deixa de ser tão importante quanto o era anteriormente.
Cresce, porém, outro tipo de controle, o controle das sociedades sobre
os seus membros, que numa sociedade composta por muitos membros
caracteriza"-se por ser uma tirania da maioria, ``quando a
sociedade é ela mesma a sociedade"-tirana, coletivamente, sobre os
indivíduos que a compõem''. Se antes essa tirania era exercida por meio
dos atos das autoridades públicas, com o controle dessas pelas saídas
constitucionais, que fazem os governantes responsáveis diante da
comunidade (isto é, podendo ser punidos se exercerem mal o poder a eles
confiado), agora ela se exerce diretamente pela própria sociedade,
mesmo que ao arrepio da lei. Apenas a proteção diante do magistrado não
basta: 

\begin{hedraquote}
Precisa"-se também de proteção contra a tirania da
opinião e sentimento prevalecentes, contra a tendência da sociedade em
impor, por outros meios que as penas civis, as suas ideias e práticas
próprias como regras de conduta sobre aqueles que divirjam delas.
\end{hedraquote}

A proteção à liberdade é primordial, seja no campo social ou no campo do
conhecimento: só a manutenção da liberdade garante o aumento do
conhecimento. Este, em si, nunca está completamente garantido, mas
sempre há possibilidades de melhorá"-lo, tornando"-o mais acurado,
desde que haja liberdade de discussão e pesquisa. Na medida em que o
grau de conhecimento científico aumenta, mais e mais conclusões são
tidas como corretas e tiradas do terreno da discussão. Mas esse aumento
de certezas depende inicialmente da liberdade de defender pontos de
vista divergentes. Mill vai bem longe na defesa da liberdade de
opinião, pois se uma pessoa tem uma opinião, seja ela falsa ou
verdadeira, silenciá"-la é ruim, muito mais para aqueles contrários a
tal opinião do que para aqueles que a defendem.

O que já foi exposto nos permite compreender por que o já falecido
filósofo liberal britânico Isaiah Berlin (1909--1997), 
em seus \textit{Quatros ensaios sobre a liberdade}, considerou John
Stuart Mill como o grande filósofo da liberdade e, em termos mais
práticos, ``o mais apaixonado e o mais famoso defensor na
Inglaterra dos humilhados e oprimidos'', o defensor, pela vida toda,
``dos heréticos, dos apóstatas, dos blasfemos, da liberdade e do
perdão''. No que talvez seja uma colocação melhor, pode"-se afirmar
que, não importando o assunto em questão, Mill sempre entendeu o tema
da liberdade como extremamente relevante a esse assunto. Que dessa
importância dada à liberdade se tire, como Berlin, a conclusão de que
nas propostas e nos atos de John Mill o que de fato importava era única
e somente a liberdade e a justiça (a qualquer preço), e não a
capacidade de ser útil, não é, porém, nenhum exagero. Mas mesmo Berlin
tem de admitir que, para Mill, a liberdade não era um fim, mas sim um
meio. Como está colocado em \textit{Sobre a liberdade}, o objetivo da
\mbox{liberdade}, a sua razão de ser, a sua utilidade, é o aumento do
bem"-estar da humanidade. A liberdade constitui"-se na única e
infalível fonte do aprimoramento da humanidade. O avanço desta ocorre
quando o ``espírito da liberdade'' ou ``do progresso ou melhoria'', como é
chamado dependendo das circunstâncias, prevalece sobre o ``despotismo do
costume''. Nem sempre o espírito da melhoria é igual ao espírito da
liberdade, como explica Mill, já que o progresso pode ser imposto a um
povo que não o deseja. Neste caso, o espírito da liberdade pode se
aliar provisoriamente aos opositores do progresso. Esta concessão de
Mill em relação a um espírito de liberdade retrógrado não parece
significar muito, se é que de fato não passa de uma incoerência. Para
Mill, o progresso só ocorre pela obra de uma elite intelectual, que
carrega consigo o espírito da liberdade, em confronto com o conformismo
(ou revolta anárquica) das massas. A proposição de que uma elite
realize o desenvolvimento – o qual exige a livre discussão de ideias –
através de métodos despóticos pode ser creditada não a algum pessimismo
escondido dentro do utilitarismo milliano, mas sim a seu otimismo, que
o levava a considerar que o pior já tinha passado, pelo menos nos
países mais civilizados. O problema estaria em que, como o progresso
não se interrompe, ao contrário do que temia ou desejava Mill, uma
argumentação semelhante sobre o papel progressista do despotismo
poderia ser e realmente foi utilizada para justificar regimes
ditatoriais que surgiram depois da época vitoriana, muitos deles na
Europa já ``civilizada''. A tal tirania da maioria, tão temida, nem
sempre foi a responsável pelos horrores que se seguiram à
democratização (que Mill apontava como a mais forte tendência do mundo
moderno, mesmo que não fosse acompanhada pela implementação de
instituições políticas populares). 

Em \textit{Sobre a liberdade}, a liberdade que vale é a
liberdade individual, de pessoas conscientes, adultas e bem educadas.
São elas, seus gostos, seus modos de vida e suas ideias que devem ser
protegidos, tanto das ameaças que vêm de cima como de baixo. Que
Mill esteja também falando em causa própria é uma conclusão óbvia, mas
ela não esgota a questão. A proteção à liberdade, como já foi mostrado,
implica a continuidade do desenvolvimento e, portanto, a possibilidade
de ampliar não só as liberdades, mas o usufruto das benesses trazidas
pelo progresso à maioria da população. E é por este caminho que o
civismo ou republicanismo de Mill faz a sua entrada. A defesa da
liberdade individual que Mill verte página após página, frase após
frase, está ligada a uma concepção de dever séria, dura, exigente, que
oferece pouca margem a liberalidades no viver. Este é o grande golpe de
mágica de Mill: ao defender justamente o que poderia parecer
indefensável, a liberdade de qualquer um fazer o que quiser, desde que
não prejudique os outros, Mill está de fato defendendo de serem
perseguidos não só os melhores, como abrindo a possibilidade de que as
outras pessoas possam também se autoaperfeiçoar. Esta defesa tem um
custo, já que muitos farão, com toda a certeza, mal uso desta
liberdade. Este é um preço que se tem de pagar. Mas há também lucros,
já que a humanidade é a maior ganhadora quando suporta que cada qual
viva como lhe apetece. 

Novamente, a importância da manutenção da liberdade torna"-se geral, de
interesse público. Liberdade de pensamento e de gosto, não de ação:
``Ninguém defende que as ações possam ser tão livres quanto as
opiniões''. Ações têm consequências que estão sujeitas ao controle e
repressão governamentais. A liberdade de expressão de opiniões também
não é irrestrita: uma coisa é afirmar, numa roda de amigos, que toda a
propriedade é um roubo. Bem outra é expressar a mesma opinião aos
brados, acompanhado de uma multidão enfurecida, diante da mansão de uma
pessoa rica. Falar em público é, de certa forma, agir ou levar à ação.
E as más consequências da ação ou fala de uma determinada pessoa não
são protegidas pelo princípio da liberdade, encontrando"-se além de
seus limites, já que as consequências dos atos devem ser imputadas
àquele que pratica a ação, e a sociedade pode e deve punir severamente
aquele que por seus atos prejudicar as outras pessoas. 

Mill não é nem um pouco dócil em relação à aplicação de penas merecidas,
mas esta severidade não deve ser entendida como mal colocada dentro de
sua filosofia política. Embora Mill seja duro com os que quebram as
leis, essas devem ter os seus limites e aplicações muito bem pensadas
(na prática, restringidas) de modo a impedir injustiças. Como a questão
das provas, para os utilitaristas, pende a favor dos acusados, o rigor
de Mill é um tanto, se não muito, matizado. A linguagem de Mill está
sempre carregada de um rigorismo moral que se encontra espalhado por
toda a sua obra. Este, no entanto, desigualmente distribuído, as
classes populares sendo beneficiárias de uma maior amplitude de
compreensão que as elites. As primeiras, fracas, desunidas e
subjugadas, são mentirosas (embora envergonhadas dessa característica)
e propensas a cometerem ações que acabam por infligir mais mal ainda a
elas próprias. Para as elites bem pensantes, restam os deveres e os
prazeres intelectualmente mais refinados. Apesar de Mill afirmar
que não se trata de restringir a individualidade ou de impedir
a experiência com novas e originais maneiras de se viver, constitui
sempre um problema até onde essas novas experiências de viver podem ser
estendidas. Na vida particular de cada pessoa praticamente não há
limites \textit{a priori}, mas como essa vida particular ocorre
conjuntamente com a social, os problemas levantados por Mill sobre os
limites dessa liberdade tornam"-se extremamente relevantes. Os ``vícios
morais'' de uma pessoa podem afetar apenas a ela e àqueles que
compartilharem dos mesmos gostos, mas os excessos de novas e originais
experiências podem se tornar perigosos, cair na boca do povo, por assim
dizer. Daí Mill propor a contenção, o autocontrole, aliados ao
trabalho extenuante e a diversões de alto nível (Mill foi um botânico
amador respeitado, que descobriu inclusive espécies não"-catalogadas) 
como adequadas à elite, para a qual ``o mais apaixonado
amor à virtude e ao autocontrole'' são, ou deveriam ser, qualidades
intrínsecas. Como para as qualidades ou virtudes pessoais vale o mesmo
que para os defeitos pessoais, isto é, elas só valem para cada pessoa,
então o amor à virtude e o autocontrole devem ter um valor social que
os ligue à sociedade. Virtudes são, quase por definição, totalmente
públicas, e cabe à elite tê"-las e mostrá"-las como exemplo e como
amostra da capacidade de liderança e, por consequência, de suas
obrigações para com a população e o país. Mill reconhece que esta sua
visão tem um caráter passadista, já que o ``pouco de
reconhecimento que a ideia da obrigação para com o público consegue na
moderna moralidade é derivado de fontes gregas e romanas''. Mill não
propõe um retorno completo ao ambiente moral da Antiguidade, mas
sim a preservação de aspectos dessa moralidade dentro de um novo mundo.
O que faltaria aos tempos atuais é o espírito público, sem o qual uma
sociedade não poderia nem se desenvolver nem perseverar. Na verdade, é
necessária a ampliação desse espírito. O último baluarte contra o
domínio das massas num mundo democratizado, reservado até então às
elites, deve passar em grande parte para a responsabilidade do próprio
povo. Sem esta passagem, tudo está perdido, pois nem a liberdade nem o
desenvolvimento sobreviverão. Que Mill não tenha uma opinião muito
favorável sobre a grande massa do povo em nada impede que ele veja na
educação desta o supremo dever dos que possuem a riqueza intelectual e
também – em menor escala – dos que têm sob o seu controle as riquezas
materiais. O caminho pode ser longo e árduo, mas é o único. E, dentre
as fraquezas morais da massa, algumas podem ser utilizadas para ajudar
o desenvolvimento moral e político delas próprias, pois, ``de
modo geral, a humanidade não somente é moderada no seu intelecto, mas
também nas suas inclinações''. As massas podem ser
modeladas através dos bons exemplos, que ajudam a criar uma
consciência. O povo, para Mill, não é mau, não procura ativamente fazer
o mal, apenas não tem uma consciência moral evoluída, pois ``não
é porque os desejos dos homens são fortes que eles agem mal, mas sim
porque as suas consciências são fracas''. Ensinado, treinado, o povo, as
pessoas em geral aprenderão a se comportar bem e a assumir
responsabilidades que lhes permitirão participar dos negócios públicos,
cada qual segundo a sua capacidade, de forma racional e que atente aos
seus verdadeiros interesses. Para isto é necessário educação e também
participação na direção dos negócios públicos de caráter mais local e
próximo ao cotidiano das pessoas.

\textit{Sobre a liberdade} fez sucesso a partir do momento em que foi
publicado pela primeira vez, e durante anos foi considerado, em
universidades inglesas e norte"-americanas, como o texto básico sobre
o liberalismo. As ideias de Mill, no entanto, não se deram bem durante
a maior parte do século \textsc{xx}, quando movimentos autoritários e
antiliberais, como o comunismo, o fascismo e o nazismo, atraíram a
atenção e levaram ao engajamento de milhões e milhões de pessoas numa
busca desenfreada por uma nova época, uma nova ordem, um novo ser
humano, busca que acabou resultando na morte de dezenas de milhões de
pessoas. De fato, num mundo em que o que se queria era simplesmente
extirpar os inimigos, fossem eles de raças ou de classes diferentes, 
defender que é  melhor e muito mais útil que todas as opiniões sejam
ouvidas e debatidas, num clima de liberdade de opinião e de ação, era
pedir para ser tido como velho e antiquado. John Stuart Mill foi então
rotulado como um economista burguês, como um vitoriano ingênuo que
acreditava no progresso, e como um mero liberal que defendia sua
posição de classe. Hoje em dia, quando as ameaças contra a liberdade
avultam de todos os lados, como aliás também nos dias de Mill, a
leitura e a consideração de \textit{Sobre a liberdade} pode ser um
meio de se entender e enfrentar os problemas atuais. A liberdade é
sempre uma questão espinhosa, e quase sempre os espinhos dela
constituem a liberdade dos outros, daqueles que não pensam como nós.
Mill entendeu isso como ninguém e se empenhou em passar essa mensagem
para as gerações futuras. Se durante um bom tempo parecia pouco
provável que essa mensagem ainda encontrasse leitores, talvez seja hoje
o momento de reavaliar essa impressão. 



\part{sobre a liberdade} 

\hedramarkboth{Sobre a liberdade}{Stuart Mill}


\vspace*{.9\textwidth}
\epigraph{O grande e predominante princípio, para o qual todo argumento
desenvolvido nessas páginas converge diretamente, é a absoluta e
essencial importância do desenvolvimento humano em sua mais rica diversidade.}
{\textit{Sphere and Duties of Government}\break [Esfera e deveres do
governo]\break Wilhelm von Humboldt} %

\chapter{Dedicatória}


\textsc{Dedico} este volume à amada e pranteada memória daquela que foi a inspiradora e, em
parte, a autora de tudo que há de melhor em meus escritos --- à amiga e
esposa cujo exaltado sentido do verdadeiro e do correto foi meu mais
forte incentivo, e cuja aprovação foi a minha principal recompensa. Como tudo o mais que tenho escrito por tantos
anos, ele pertence tanto a ela quanto a mim; mas a obra, tal como ela
está, teve apenas em grau muito deficiente a inestimável vantagem de
ter sido revisada por ela, pois algumas passagens ficaram à espera de
cuidadoso e novo exame, que estão agora destinadas a
nunca receber. Se fosse eu capaz de apresentar ao mundo metade dos 
grandes pensamentos e dos nobres sentimentos que agora estão
enterrados no seu túmulo, seria então o intermediário de um
benefício para este mundo maior do que jamais poderia surgir de algo que eu
possa escrever sem o incentivo e ajuda de sua sabedoria incomparável. 



\chapter{Capítulo introdutório}

\textsc{O assunto} deste ensaio não é a assim chamada Liberdade da Vontade,
contraposta de modo tão infeliz à incorretamente denominada doutrina
da Necessidade Filosófica, mas sim a Liberdade Civil ou Social: a
natureza e os limites do poder que pode ser exercido legitimamente pela sociedade
sobre o indivíduo. Uma questão dificilmente posta às claras e quase
nunca discutida em termos gerais, mas que pela sua presença latente influencia profundamente as
controvertidas práticas de nossa época, e que
possivelmente logo se fará reconhecer como sendo a questão vital do
futuro. Longe de ser uma novidade, em certo sentido ela tem dividido
a humanidade quase desde os tempos mais remotos, mas, no patamar de
progresso em que entraram agora as porções mais civilizadas da espécie humana,
ela se apresenta sob novas condições e requer um tratamento diferente e
mais fundamental. 

O conflito entre Liberdade e Autoridade é a mais evidente faceta dos eventos históricos com que estamos familiarizados, particularmente os da Grécia,
Roma e Inglaterra. Mas nos tempos antigos essa competição permanecia
entre os súditos ou entre algumas classes dos súditos e o governo. Por
liberdade se queria significar proteção contra a tirania dos dirigentes
políticos. Considerava"-se que os dirigentes (exceto alguns governos
populares da Grécia) estavam necessariamente numa posição
de antagonismo em relação ao povo que governavam. Eles consistiam ou
de um governante apenas ou de uma tribo ou casta governante, que obtinha sua
autoridade por herança ou conquista; alguém que, de qualquer modo, não a
possuía devido à concordância dos governados, e cuja supremacia os homens
não se aventuravam a contestar ---  e talvez nem sequer o desejassem --- 
quaisquer que fossem as precauções que tomassem contra o seu exercício
opressivo. O poder dos governantes era visto como necessário, mas também como
altamente perigoso; como uma arma que eles poderiam vir a usar tanto
contra seus súditos quanto contra os inimigos externos. Para impedir
que os membros mais fracos da comunidade fossem atacados por
inumeráveis abutres, era necessário que houvesse um animal de rapina
mais forte que o resto, que tivesse por função mantê"-los sob
controle. Mas como o rei dos abutres não estaria menos propenso a
atacar o rebanho do que qualquer uma das harpias menores, era indispensável
que se mantivesse uma atitude de defesa contra o seu bico e as suas garras. O
objetivo, portanto, dos patriotas era o de antepor limites ao poder de que
o governante dispunha sobre a comunidade, e era essa
limitação que eles reconheciam como liberdade. Tentaram estabelecê"-la de
duas maneiras. A primeira, através do reconhecimento de certas
imunidades, chamadas de liberdades políticas ou direitos, pelas quais sua 
infração por parte do governante era vista como uma violação do dever; se
de fato houvesse a infração, então uma resistência específica, ou uma
rebelião geral, era tida como justificada. A segunda maneira geralmente
consistia em uma medida posterior, através do estabelecimento de constrangimentos
constitucionais, do consentimento da comunidade ou de algum
tipo de corpo, que se supunha representar os interesses desta,
a condição necessária para alguns dos atos mais importantes do poder
dominante. Muitos países europeus foram compelidos, em maior 
ou menor grau, a se submeter ao primeiro desses modos de 
limitação dos poderes dominantes. O mesmo não se passou com o segundo; e obtê"-lo, quando não
se o tem, ou aumentá"-lo, quando já se o possui parcialmente, tornou"-se
em toda a parte o objetivo dos amantes da liberdade. Enquanto
a humanidade esteve satisfeita em combater um inimigo depois do outro, e em ser
governada por um senhor, com a condição de estar
protegida de forma mais ou menos eficaz contra a tirania dele, os amantes da liberdade não levaram suas
aspirações além deste ponto.

Entretanto, no progresso dos assuntos humanos chegou um momento em que
os homens deixaram de pensar que é uma necessidade da natureza que seus
governantes tenham de ter um poder independente, oposto aos seus
interesses. Pareceu"-lhes muito melhor que os vários magistrados do
Estado pudessem ser seus representantes ou delegados e que pudessem ser
revogados quando lhes aprouvesse. Somente dessa maneira, assim pareceu,
poderiam eles ter total segurança de que os poderes do governo nunca
seriam utilizados para a sua desvantagem. Gradativamente, essa nova
demanda por magistrados eletivos e temporários tornou"-se o principal objetivo dos esforços do partido
popular, onde quer que um partido com esse pendor existisse, e superou, em
larga escala, os esforços anteriores para limitar o poder dos
governantes. Enquanto se dava o conflito para fazer com que o
poder dominante emanasse de uma escolha periódica por parte dos
governados, algumas pessoas começaram a pensar que  tinha sido dada importância
demasiada à limitação dos poderes.\textit{ Esse} (assim
parecia) era um recurso contra os governantes cujos interesses eram
habitualmente opostos aos do povo. O que se precisava agora era que os
governantes se identificassem com o povo, que o seu interesse e vontade
fossem o interesse e a vontade da nação. A nação não precisaria ser
protegida de sua própria vontade. Não havia medo de que ela viesse a
abusar de si própria. Que fossem então os governantes de fato
responsáveis por ela, prontamente revogáveis por ela, e a eles poderia ser
confiado um poder do qual ela mesma poderia impor o modo com que seria usado. O poder deles não
era senão o poder da nação, concentrado e numa forma conveniente para
o seu exercício. Essa forma de pensar, ou melhor, de perceber, era comum
entre a última geração do liberalismo europeu 
e ainda parece ser predominante na parte continental da Europa. 
Aqueles que admitem qualquer limite para o que o governo pode fazer, 
exceto no caso de governos que, segundo eles, não deveriam existir, se destacam
como brilhantes exceções entre os pensadores políticos do continente.
Um sentimento semelhante poderia nesse momento ter prevalecido em 
nosso próprio país, se as circunstâncias que
por um período o encorajaram tivessem permanecido inalteradas.

Para teorias políticas e filosóficas, tanto quanto para pessoas, o
sucesso faz todavia aparecer falhas e fraquezas que o fracasso poderia ter
ocultado. A noção de que o povo não precisa limitar o
poder que tem sobre si mesmo poderia parecer axiomática se o
governo popular fosse apenas algo com que se sonhava a respeito, ou
se fosse interpretado como algo que tivesse ocorrido num distante período do
passado. Essa noção não foi necessariamente distorcida por aberrações
temporárias como a Revolução Francesa, o pior dela tendo sido obra
de uns poucos \mbox{usurpadores} e que, em qualquer caso, ocorreu não devido
ao trabalho permanente das instituições populares, mas sim devido ao
súbito e convulsivo levante contra o despotismo monárquico e
aristocrático. Com o tempo, no entanto, a república
democrática veio a ocupar uma larga porção da superfície terrestre e
fez de si mesma um dos mais poderosos membros da comunidade das nações; 
o governo eletivo e responsável tornou"-se alvo das observações e
críticas que tendem a recair sobre acontecimentos de grande
valor.\footnote{ Mill se refere aos Estados Unidos da América, 
independentes da Coroa Britânica a partir de 1776. [\versal{N.T.}]}
 Agora se percebe que frases tais como ``governo"-próprio'' e ``o poder do
povo sobre si mesmo'' não expressam o verdadeiro estado das coisas. O
``povo'' que exerce o poder não é sempre o mesmo que aquele sobre o qual o
poder é exercido, e o ``governo de si mesmo'' de que se fala 
não é o governo de cada um por si mesmo, mas sim o governo de cada um
por todo o resto. Além disso, a vontade do povo significa
praticamente a vontade da parte mais numerosa ou da mais ativa do povo;
a maioria, ou aqueles que conseguem fazer se passar por ela; o povo,
por conseguinte,\textit{ pode} desejar oprimir uma parte de sua 
totalidade, e são necessárias precauções contra isso tanto quanto 
contra qualquer outro tipo de abuso de poder. Portanto, a limitação do
poder do governo sobre os indivíduos não perde nada de sua importância
quando os detentores do poder são, de forma regular, responsáveis
diante da comunidade, isto é, em relação ao partido mais poderoso desta.
Esse ponto de vista, que se recomenda por si mesmo igualmente para a
inteligência dos pensadores e para os pendores daquelas importantes
classes da sociedade \mbox{europeia} a cujos \mbox{interesses}, reais ou supostos, a
democracia é contrária, não teve nenhuma dificuldade em se estabelecer,
e nas especulações políticas ``a tirania da
maioria''\footnote{ Ver Tocqueville, \textit{De La Démocratie en Amérique}, 
v.~\textsc{ii}, p.~142. [\versal{N.T.}]}  é agora geralmente incluída entre os males
contra os quais a sociedade precisa estar de sobreaviso. 

Como outras tiranias, a da maioria se fez primeiro temida, e ainda
geralmente o é, principalmente através dos atos das autoridades
públicas. Mas pessoas ponderadas perceberam que, quando a sociedade
mesma é o tirano --- a sociedade tomada coletivamente, acima dos
interesses dos indivíduos separados que a compõem ---, seus meios de
tiranizar não ficam restritos a atos que podem ser realizados pelas
mãos dos funcionários políticos. A sociedade pode executar e efetivamente 
executa as suas próprias ordens, e se ela emite algumas ordens erradas ao invés
de corretas, ou qualquer ordem que seja em assuntos em que ela não
deveria se imiscuir, ela pratica uma tirania social muito mais
terrível do que outros tipos de opressão política, já que, apesar de
não ser seguida de penalidades extremas, ela deixa menos vias de
escape, penetrando profundamente nos detalhes da vida e escravizando a 
alma ela mesma. Portanto, a proteção contra a tirania do magistrado não é
suficiente; há necessidade de proteção também contra a tirania das opiniões,
contra a tendência da sociedade em impor, por meios diversos que as penas
civis, suas próprias ideias e práticas como regras de conduta para
aqueles que discordam delas; há necessidade de impedir o desenvolvimento e, se
possível, a formação de qualquer individualidade que não esteja em
harmonia com os modos da sociedade, e compelir todos a se amoldar no 
modelo que ela quiser. Há um limite para a interferência
legítima da opinião coletiva na independência individual, e descobrir
esse limite e protegê"-lo contra o seu cerceamento é tão
indispensável para a boa condução dos negócios humanos quanto a
proteção contra o despotismo político. 

 Apesar dessa proposição não parecer contestável em termos gerais, a
questão prática, onde colocar os limites --- como produzir o ajuste
adequado entre a independência individual e o controle social ---, é um
assunto em que tudo permanece por se fazer. Tudo o que faz a
existência ter valor para alguém depende da imposição de restrições
às ações das outras pessoas. Algumas regras de conduta, portanto,
devem ser impostas pela lei em primeiro lugar e depois pela opinião
pública nas muitas coisas que não estão sujeitas ao controle legal.
Quais devem ser essas regras é a principal questão nos negócios
humanos, mas, se excetuarmos um ou outro caso mais óbvio, essa é uma
daquelas questões em que se obteve muito pouco progresso no
sentido de resolvê"-la. Duas épocas, e dificilmente dois países, não
a resolveram de forma igual; a decisão de uma época ou país é motivo
de assombro para outras épocas e países. No entanto, a população de
qualquer era ou país não suspeitou jamais que a sua solução
apresentasse qualquer dificuldade, como se esse fosse um assunto sobre
o qual a humanidade tivesse sempre permanecido em concordância. As regras
que as pessoas produzem entre elas parecem sempre autoevidentes e
autojustificadas. Essa ilusão universal é um dos exemplos da
influência mágica dos costumes, que não apenas são, como reza o ditado,
uma segunda natureza, mas são continuadamente tidos como a
primeira. O efeito do costume, ao  prevenir qualquer \mbox{ressalva} em relação
às regras de conduta que a humanidade impõe sobre as pessoas, é mais
completo ainda porque esse é um assunto em que não se espera que as
razões sejam dadas, quer de uma pessoa para outra, quer de cada um para
si mesmo. As pessoas estão acostumadas a acreditar, e têm sido
encorajadas nessa crença por alguns que almejam a posição de filósofos,
que em assuntos dessa natureza seus sentimentos são melhores que
as razões, tornando assim as razões desnecessárias. O princípio prático que as
guia em suas opiniões sobre a regulação da conduta humana é o
sentimento, na mente de cada pessoa, de que todo mundo deve ser instado
a agir como ela e aquelas pessoas com quem simpatiza gostariam
que agisse. De fato, ninguém reconhece para si mesmo que o seu padrão
de julgamento é o seu gosto; mas uma opinião sobre um ponto da conduta,
não sustentada por razões, pode ser apenas considerada como a
preferência de uma pessoa, e mesmo se as razões, quando apresentadas,
tiverem um apelo para uma preferência semelhante sentida por outras
pessoas, mesmo assim seria apenas a preferência de muitos, ao invés da
de um só. Para o homem comum, no entanto, a sua preferência
não apenas é uma razão perfeitamente satisfatória, mas é a única que
ele possui, geralmente, para quaisquer noções que venha a ter de
moralidade, gosto ou decoro, e que não sejam explicitamente expressas no
seu credo religioso; e que é também o principal guia mesmo para esse
último. A opinião dos homens, correspondentemente, sobre o que é
louvável ou criticável é afetada por todas as múltiplas causas que
influenciam os seus desejos em relação às condutas dos outros, e que são
tão numerosas quanto aquelas que determinam os seus desejos em qualquer
outro assunto. Às vezes, suas razões, outras vezes seus
preconceitos ou superstições, frequentemente os seus afetos sociais, e
muitas vezes os antissociais, as suas invejas e os seus ciúmes, a sua
arrogância ou menosprezo, mas, mais comumente, os seus desejos e medos, os
seus próprios interesses, legítimos ou ilegítimos. Quando há uma classe
ascendente, uma larga porção da moralidade de um país emana dos
interesses dessa classe e de seus sentimentos de superioridade. A
moralidade entre espartanos e hilotas,\footnote{ Servos do estado espartano 
que não gozavam de direitos políticos. [\versal{N.T.}]} 
entre fazendeiros e negros, entre príncipes e súditos, 
entre nobres e lacaios, entre homens e mulheres, tem sido, na sua maior
parte, a criação desses interesses e sentimentos de classe, e os
sentimentos assim gerados reagem, por sua vez, sobre os sentimentos
morais dos membros das classes ascendentes, em suas relações externas.
Onde, por outro lado, uma classe anteriormente ascendente perdeu a sua
ascendência, ou onde essa ascendência é impopular, o sentimento moral
prevalente mostra com frequência uma forte faceta de superioridade.
Outro grande e determinante princípio das regras de conduta, tanto nos
atos quanto nas proibições dos atos, e que foi imposto por lei ou
opinião, tem sido o servilismo da humanidade para com as supostas
preferências e a aversão por seus mestres temporais ou por seus deuses.
Esse servilismo, apesar de ser essencialmente egoísta, não consiste em
hipocrisia, ocasionando sentimentos genuínos de desprezo e fazendo os
homens queimarem bruxos e heréticos. Entre tantas influências, o óbvio
e geral interesse da humanidade tem tido uma participação sem dúvida grande na direção dos sentimentos morais, entretanto, menos por uma questão de razão ou motivos próprios do que em
consequência de simpatias e antipatias que nascem daqueles interesses;
e simpatias e antipatias que nada ou pouco têm em comum com os
interesses da sociedade têm se mostrado com grande força no
estabelecimento das moralidades. 

 Aquilo de que gosta ou não gosta uma sociedade, ou uma porção poderosa dela, é,
portanto, o que, em termos práticos, determina as
regras declaradas para o cumprimento geral, sob as penalidades da lei e
da opinião. E, em geral, aqueles que têm sido mais avançados que a
sociedade em pensamentos e sentimentos, têm em princípio mantido esse estado de
coisas imutável, por mais que entrassem em conflito
com ela em questões de detalhe. Essas pessoas se ocuparam mais em
inquirir se o que agrada e o que desagrada uma sociedade deve ser uma lei para
os indivíduos. Elas preferiram tentar alterar os sentimentos da
sociedade em questões particulares em que elas próprias 
são heréticas, ao invés de fazer uma causa comum com os outros 
heréticos em defesa da liberdade. O único caso em que uma
posição superior foi tomada por princípio e mantida com consistência
por quase todos, exceto uma ou outra pessoa aqui e ali, é o das crenças
religiosas, um caso de muitas maneiras instrutivo, não menos por
representar um exemplo da falibilidade do que é chamado de senso moral,
pois é o \textit{odium theologicum}, num fanático sincero, uma das
causas mais inequívocas do senso moral. Aqueles que primeiro lançaram
fora o jugo daquela que chamava a si mesma de Igreja Universal estavam,
em geral, tão pouco dispostos a permitir diferenças de opiniões
religiosas quanto aquela igreja. Mas quando o pior do conflito já tinha
passado, sem que nenhuma das partes obtivesse uma vitória
completa, e com cada igreja ou seita tendo, portanto, que limitar suas
esperanças e apenas conservar o terreno já conquistado, as minorias,
percebendo que não tinham chance de se tornar maiorias, se viram na
necessidade de pedir para aqueles a quem não podiam converter que
lhes fosse dada uma permissão para serem diferentes. É apropriado que 
os direitos do indivíduo contra a sociedade tenham sido definidos em 
amplas bases de princípios tão"-somente neste 
campo de batalha, e que a pretensão da sociedade em exercer
autoridade sobre os dissidentes fora abertamente contestada. Os
grandes escritores, aos quais o mundo deve a liberdade religiosa que ora
possui, declararam a liberdade de consciência como um direito
inalienável, e negaram, de forma absoluta, que um ser humano deva dar
conta de suas crenças religiosas para os outros. No entanto, tão natural é
a intolerância da humanidade naquilo que realmente importa a ela que a
liberdade religiosa só foi de fato implementada na maioria dos lugares
em que a indiferença religiosa, que não quer ver a sua paz de espírito
estremecida por controvérsias religiosas, pôs o seu peso na balança. Na
mente da maioria das pessoas religiosas, mesmo nos países mais
tolerantes, o dever da tolerância é admitido com ressalvas tácitas. Uma
pessoa pode tolerar a dissidência dentro do governo da igreja, mas não
em relação aos dogmas, outra pode tolerar a todos, exceto os papistas
ou os unitaristas, já outra ainda pode tolerar qualquer um que acredite em 
uma religião revelada, e umas poucas pessoas estendem sua caridade um
pouco mais, mas param na crença em um Deus e na vida futura. Onde quer
que o sentimento da maioria seja ainda intenso e genuíno, descobre"-se
que em pouco se mitigou a sua ânsia em ser obedecida. 

Na Inglaterra, devido às circunstâncias peculiares de nossa história,
apesar do peso da opinião ser talvez ainda maior, o da lei é mais leve que
na maioria dos países da Europa, e há considerável repúdio à
interferência direta do poder legislativo ou executivo na conduta
privada, não tanto por alguma justa preocupação com a independência do
indivíduo, mas sim devido ao ainda existente hábito de ver o governo
como representante de um interesse oposto ao da população. A maioria ainda
não aprendeu a sentir o governo como o seu poder, e a opinião
deste como a sua. Há ainda uma grande quantidade de sentimentos
prontos para serem postos em ação contra qualquer tentativa da
lei de controlar os indivíduos em assuntos nos quais estes não estão
acostumados a ser controlados, e há pouca discriminação a respeito de
que assunto pode estar ou não dentro da esfera legítima do controle
legal, o que quer dizer que esse sentimento, por mais salutar que seja
no geral, talvez seja mais mal colocado do que bem fundado nas instâncias
particulares de sua aplicação. De fato, não há nenhum princípio
reconhecido pelo qual a impropriedade ou não de uma interferência
comumente seja testada. As pessoas decidem de acordo com suas
preferências pessoais. Algumas, vendo um bem que pode ser realizado ou
um mal a ser remediado, poderiam ser instadas favoravelmente à ação
governamental, enquanto outras preferem suportar qualquer
quantidade de mal social ao invés de adicionar um departamento de
interesses humanos a mais ao poder governamental. E diante de um caso
particular, os homens se colocam de um lado ou do outro, de acordo com
a direção geral dos sentimentos ou ainda de acordo com o grau de
interesse que eles sintam sobre a coisa particular que é proposta para
a ação governamental, ou ainda de acordo com a crença que tenham sobre
se o governo fará ou não algo da maneira que eles preferem, mas muito
raramente em relação a alguma opinião à qual eles aderem com
consistência, sobre as coisas adequadas que devem ser praticadas por um
governo. Para mim, parece que devido a essa ausência de uma regra ou
princípio, na atualidade um lado está tão frequentemente em erro quanto
o outro; a interferência do governo é com igual frequência
impropriamente invocada e impropriamente condenada.

O objetivo deste ensaio é afirmar um princípio básico muito simples, o
modo correto para ordenar de forma absoluta as relações da sociedade para
com o indivíduo, seja por meio da compulsão e controle, seja por meios
de força física na forma de sanções penais, seja ainda pela coerção moral
da opinião pública. Esse princípio diz que o único objetivo  pelo
qual a humanidade pode, de forma individual ou coletiva, interferir
com a liberdade de ação de qualquer de seus membros, é a proteção dela
própria. E que o único propósito pelo qual o poder pode ser
constantemente exercido sobre qualquer membro de uma comunidade, contra a
vontade deste, é o de prevenir danos para os outros membros. O próprio
bem dele, seja físico ou moral, não é causa suficiente. Ele não pode
ser compelido a fazer ou a deixar de fazer algo porque isso seria
melhor para ele, ou porque iria fazê"-lo mais feliz ou porque, na
opinião dos outros, isso seria o melhor ou mesmo o correto. Pode haver
boas razões para criticá"-lo, para conversar com ele, para tentar
persuadi"-lo ou para discutir com ele, mas não para obrigá"-lo 
ou causar"-lhe algum mal se ele fizer diferente. Para justificar uma
intervenção, a conduta que se deseja impedir da parte dele deve 
ameaçar outra pessoa. A única parte da conduta de
qualquer pessoa,  pela qual ela é responsável perante a sociedade, é
aquela que diz respeito às outras pessoas. Naquela parte que só diz
respeito a si mesma, a independência de cada pessoa é, por direito,
absoluta. Sobre si mesmo, sobre seus próprios corpo e mente, o
indivíduo é soberano. 

Talvez seja necessário afirmar que esta doutrina deve ser
aplicada somente em seres humanos que estejam na maturidade de suas
faculdades. Não nos referimos às crianças ou aos jovens abaixo da idade
que a lei pode fixar como a maioridade. Aqueles que ainda estão
em uma situação na qual necessitam que outras pessoas tomem conta deles
devem ser protegidos, tanto de suas próprias ações como de danos
externos. Pela mesma razão, podemos deixar fora de consideração aqueles
estágios atrasados da sociedade nos quais a própria raça pode ser
considerada como menor de idade. As dificuldades iniciais no
caminho do progresso espontâneo são tão enormes que dificilmente há
alguma possibilidade de escolher os meios para superá"-las; a um
líder cheio de determinação e desejoso de melhorar as coisas é
permissível o uso de qualquer meio que leve a esse fim, que de outra
forma talvez seja inalcançável. O despotismo é um modo legítimo de
governo quando se lida com bárbaros, desde que o objetivo seja a
melhoria destes, e os meios justificados para a obtenção de fato daquele
objetivo. A liberdade, como um princípio, não tem nenhuma aplicação em
um estado de coisas anterior ao tempo em que a humanidade se tornou
capaz de se aperfeiçoar através de uma discussão livre e igualitária.
Até que esse momento chegue, nada resta para as pessoas exceto a
obediência a um Akbar ou a um Carlos Magno, se elas tiverem sorte
bastante para encontrar um desses. Mas tão logo a humanidade tenha
alcançado a capacidade de ser guiada, no seu caminho para o progresso,
pela convicção ou pela persuasão (um período há muito alcançado pelas
nações que aqui nos interessam), a coação, seja na forma direta ou na
de sofrimentos e punições pela não obediência, deixa de ser admissível
como um meio para o próprio bem da humanidade, sendo apenas
justificável para a segurança das pessoas. 

A propósito, rejeito qualquer vantagem que poderia ser
derivada para o meu argumento da ideia de direitos abstratos como algo
independente da utilidade. Vejo a utilidade como o tribunal final em
todas as questões éticas, mas ela deve ser utilidade em seu sentido
mais amplo, firmada nos interesses do homem enquanto um ser que
progride. Esses interesses, afirmo, autorizam a sujeição da
espontaneidade individual ao controle externo somente em relação às
ações de cada pessoa que concernem aos interesses dos outros. Se alguém
comete um ato danoso a outra pessoa, há um caso \textit{prima facie}
para puni"-lo, pela lei ou, quando as penalidades legais não possam ser
seguramente aplicadas, pela desaprovação geral. Também há muitos atos
positivos que beneficiam os outros e que podem ser impostos a alguém,
tais como ser testemunha numa corte de justiça, assumir a sua 
parte na defesa comum, ou em qualquer trabalho em comum que seja
necessário para o interesse da sociedade da qual ele aproveite a
proteção; também para realizar certos atos de
benemerência individual, tais como salvar a vida de um semelhante ou
interferir para proteger os indefesos de serem maltratados, coisas nas
quais é óbvio o dever de um homem praticá"-las, e pelas quais ele
pode ser corretamente responsabilizado pela sociedade por não tê"-las
cumprido. Uma pessoa pode causar danos aos outros não só pelas suas
ações, mas também pela sua inação, e em ambos os casos ela
é responsável pelos danos ocorridos. O segundo caso, é bem verdade,
exige um tipo mais cauteloso de exercício de correção que o primeiro.
Tornar qualquer um responsável pelos danos que possa vir a causar aos
outros é a regra, torná"-lo responsável por não prevenir o mal é,
comparativamente falando, a exceção. No entanto, há vários casos claros
e sérios o suficiente para justificar essa exceção. Em todos os casos
nos quais há uma relação com os contatos externos do indivíduo, ele é
\textit{de jure} responsável diante daquelas pessoas cujos interesses
estão relacionados com os dele e, se preciso, diante da sociedade,
como protetora que é dos interesses das outras pessoas. Quase
sempre há muitas boas razões para não lhe entregar essa
responsabilidade, mas essas razões devem surgir das circunstâncias
específicas de um caso, quer por que a pessoa seja, de modo geral,
inclinada a agir melhor quando deixada por sua própria conta do que
quando controlada por quaisquer meios que a sociedade disponha
para isso, seja por que a tentativa de exercer esse
controle faça surgir outros males, piores do que aqueles que se quer
prevenir. Quando tais razões aconselham a não imposição da
responsabilidade, a própria consciência do agente deve ocupar o lugar
na cadeira vazia do juiz, julgando a si mesma ainda mais duramente,
porque o caso em questão não admite que ela seja 
responsabilizada pelo julgamento de seus pares.

Mas há uma esfera de ação na qual a sociedade, distinta do indivíduo,
tem apenas um interesse indireto, se de fato tiver algum, e que
compreende toda a porção da vida de uma pessoa e que afeta a ela mesma
somente, e que se afeta a outras pessoas, o faz somente através da
livre, voluntária e consciente participação delas. Quando digo a
ela somente, quero dizer diretamente e em primeira instância: o
que quer que o afete, pode afetar aos outros através dele próprio; a
objeção que pode ser estabelecida nessa contingência receberá atenção
mais adiante. E este é portanto o lugar apropriado da liberdade humana.
Primeiro, ela compreende o domínio inteiro da consciência demandando
liberdade de consciência, no sentido mais amplo, liberdade de
pensamento e de sentimento, liberdade absoluta de opinião em todos os
assuntos, práticos ou especulativos, científicos, morais ou teológicos.
A liberdade de expressar e publicar opiniões públicas parece fundar"-se
em um princípio diferente, já que pertence àquela parte da conduta
do indivíduo que concerne a outras pessoas, mas, sendo quase tão
importante quanto a liberdade de pensamento propriamente dita e se
baseando em grande parte sobre as mesmas razões, é praticamente
inseparável desta. Em segundo lugar, o princípio requer liberdade de gosto e
de inclinações, em podermos montar o nosso plano de vida de acordo com
nossos próprios caracteres, em fazer como
quisermos, sujeitos a consequências que poderão se seguir, sem
impedimentos de nossos pares, enquanto não lhes causarmos danos, mesmo
que eles achem nossa conduta imbecil, pervertida ou errônea. Terceiro,
desta liberdade de cada indivíduo advém a liberdade, dentro dos mesmos
limites, da combinação entre indivíduos; a liberdade da união, para
qualquer propósito que não envolva danos aos outros; as pessoas
envolvidas sendo supostamente maiores de idade e não
forçadas ou enganadas. 

Nenhuma sociedade na qual essas liberdades não sejam, no seu todo,
respeitadas, é livre, qualquer que seja a sua forma de governo, e
nenhuma na qual essas liberdades não existam de
forma absoluta e sem qualificações é completamente livre. A única liberdade que merece esse
nome é a de perseguir o nosso próprio bem de nossa própria maneira,
isso enquanto não tentarmos privar os outros da sua liberdade, ou
obstruirmos seus esforços para obtê"-la. Cada um é o guardião de
sua própria saúde, seja ela física, mental ou espiritual. A humanidade
é a grande vencedora ao permitir que cada um viva como lhe pareça
melhor, mais do que o seria se coagisse cada pessoa a viver de acordo
com o que parecesse melhor para o resto das pessoas. 

Embora essa doutrina não seja de forma nenhuma uma novidade e, para
certas pessoas, constitua um truísmo, não há doutrina 
mais oposta à tendência das práticas e opiniões atuais. A sociedade tem
despendido muitos esforços na tentativa de, de acordo com as suas luzes,
fazer com que as pessoas conformem as suas noções de excelência pessoal e
social. As comunidades antigas pensavam a si mesmas como
capazes de regular toda e qualquer parte da conduta privada a partir da
autoridade pública, com o que os filósofos antigos concordavam, sob a
alegação de que o Estado tem um profundo interesse na disciplina mental
e física de cada um de seus cidadãos, um modo de pensar que pode ter
sido admissível em pequenas repúblicas cercadas por inimigos poderosos,
e que viviam sob a constante ameaça de serem subvertidas por um ataque
estrangeiro ou por problemas internos e para as quais um pequeno
período de descanso, em termos de energia e comando, poder"-se"-ia
mostrar facilmente fatal, de tal forma que elas não podiam se dar ao
luxo de esperar pelos saudáveis e permanentes efeitos da liberdade. No
mundo moderno, o tamanho maior das comunidades políticas e, acima de
tudo, a separação entre o poder espiritual e o temporal (que colocou
a direção da consciência dos homens em mãos diferentes daquelas que
controlavam os seus negócios mundanos), preveniu uma maior interferência
da lei nos detalhes da vida privada; mas os mecanismos da repressão
moral têm atuado com mais rigor contra a divergência em relação à opinião reinante
do que mesmo aos assuntos sociais; a religião, o mais poderoso dos
elementos que entraram na formas do sentimento moral, tendo sido
quase sempre governada ou pela ambição de uma hierarquia que procurava
controlar qualquer âmbito da conduta humana ou pelo
espírito do puritanismo. E alguns dos reformadores modernos que se
colocaram em franca oposição às religiões do passado têm estado
atualmente ao lado de igrejas ou seitas na sua asserção do direito à 
dominação espiritual. O senhor Comte,\footnote{ Isidore Auguste Marie
François Xavier Comte, 1798---1857, filósofo francês, fundador e
principal representante do Positivismo. O \textit{Sistema de política
positiva} foi publicado em quatro volumes de 1851 a 1854. [\versal{N.E.}]} em
particular, cujo sistema social, tal como delineado em seu
\textit{Système de Politique Positive}, almeja estabelecer (através da
moral mais do que por dispositivos legais) um despotismo da sociedade
sobre o indivíduo que ultrapassa qualquer coisa contemplada no ideal
político dos mais rígidos disciplinadores dentre os filósofos antigos.

Além das inclinações particulares de pensadores individuais, também há em geral
no mundo uma inclinação crescente para ampliar indevidamente
os poderes da sociedade sobre os indivíduos, tanto pela força da
opinião quanto pela da legislação. E como a tendência de todas as
mudanças que estão acontecendo no mundo é a de fortalecer a sociedade e		\EP[-1]
diminuir o poder dos indivíduos, esse cerceamento não é um daqueles
males que tendem a desaparecer espontaneamente, mas, ao contrário, a
crescer mais e mais. A disposição da humanidade, seja em relação aos
governantes ou em relação aos cidadãos, em supor sua própria opinião
e inclinação como sendo regras de conduta para os outros, é aprovada de
forma enérgica pelos melhores --- e piores --- sentimentos aos quais está
sujeita a espécie humana, que dificilmente é posta sob restrições por
outra coisa que não seja a falta de poder; e como o poder não está
declinando, mas sim aumentando, a menos que uma forte barreira de
convicção moral possa ser erguida contra esse mal, podemos esperar, nas
presentes circunstâncias do mundo, que ele cresça ainda mais. 

Seria conveniente para este argumento se, ao invés de entrar de uma vez
por todas na sua tese geral, nos limitássemos num primeiro momento a um
único ramo dela, no qual o princípio aqui colocado é reconhecido pela opinião
corrente, se não completamente, pelo menos até certo ponto. 
Este ramo é o da Liberdade de Pensamento, que é impossível 
de separar da liberdade cognata de falar e de escrever. Apesar dessas
liberdades, na sua maioria, fazerem parte da moralidade política de	\EP[-1]
todos os países que professam a tolerância religiosa e instituições
livres, os fundamentos tanto filosóficos quanto práticos nos quais
elas se assentam talvez não sejam tão familiares para a mente comum e
não sejam completamente entendidos por muitos dos líderes da opinião,
como se poderia esperar. Esses fundamentos, quando entendidos
corretamente, possuem uma aplicação muito maior do que a de uma parcela
do assunto, e uma cuidadosa consideração do assunto será aceita como
a melhor introdução para o resto. Aqueles para os quais não 
é novo nada do que estou prestes a dizer, irão, assim espero, me desculpar, se em um
assunto que por mais de três séculos tem sido amiúde discutido ouso
inaugurar mais uma discussão. 

\oneside
\chapter[Sobre a liberdade]{Sobre a liberdade de pensamento e discussão}

\textsc{Já passou} o tempo, assim se espera, em que seria 
necessária uma defesa da liberdade de imprensa como uma das
salvaguardas contra um governo corrupto ou tirânico. Nenhum argumento,
podemos assim supor, seria preciso agora contra a permissão de que um
poder legislativo ou executivo, não identificado em seus interesses com
os do povo, possa prescrever opiniões e determinar quais
doutrinas ou quais argumentos ele pode ouvir. Além disso, esse aspecto
da questão tem sido tão frequentemente defendido, e de forma tão triunfal 
por escritores anteriores, que não há necessidade de se
insistir em especial neste ponto. Apesar da lei da Inglaterra, 
em relação à imprensa, ser hoje em dia tão servil quanto era no
tempo dos Tudors, não há muito perigo de ser posta em prática contra
as discussões políticas, exceto durante algum pânico temporário, quando
o medo da insurreição tira dos ministros e juízes a
compostura\footnote{ Mal essas palavras foram escritas quando, 
como que para contradizê"-las frontalmente, aconteceu a Perseguição 
Governamental da Imprensa de 1858. Essa interferência 
fora de propósito na liberdade de discussão
pública não me fez, no entanto, alterar uma única palavra do texto, e
não enfraqueceu as minhas convicções de que, exceto em momentos de pânico,
a época dos sofrimentos e punições relacionadas às discussões políticas
já passou neste país. Pois, em primeiro lugar, os processos não foram
continuados e, em segundo, nunca ocorreram propriamente como
perseguições políticas. A acusação não foi a de criticar instituições
ou os atos e as pessoas dos governantes, mas sim a de fazer circular o
que foi tido como uma doutrina imoral, a da legalidade do
tiranicídio.\\ Se os argumentos do presente capítulo possuem qualquer
valor, deve existir a mais total liberdade de professar e discutir,
como tema de convicção ética, qualquer doutrina, não importando o
quanto ela seja considerada imoral. Portanto, seria irrelevante e
deslocado examinar se a doutrina do tiranicídio merece o título de
imoral. Eu me contento em dizer que o assunto tem sido, em todas as
épocas, uma questão moral aberta; que o ato de um cidadão privado ao
derrubar um criminoso que, ao se elevar acima da lei, colocou"-se além
do alcance de punições ou controle legais, foi considerado por muitas
nações, e por alguns dos melhores e mais sábios dos homens, não como
um crime, mas sim um ato da mais exaltada virtude e que, certo ou
errado, não está na natureza do assassinato, mas sim da guerra civil.
Como tal, eu sustento que a instigação para esse ato, num caso
específico, pode ser objeto de uma justa punição, mas somente se um ato
aberto se segue, e que ao menos se tenha estabelecido uma conexão
provável entre o ato e a instigação. E mesmo então, não um governo
estrangeiro, mas apenas o governo que foi atacado pode, num gesto de
autodefesa, legitimamente punir ataques diretos contra a sua própria
existência. \mbox{[\versal{N.A.}]}} --- falando de forma geral, 
não se deve temer, em países constitucionais, que o governo, 
seja ele inteiramente de confiança para o povo ou não, tente com frequência
controlar a expressão da opinião, já que fazendo isso ele se tornaria o
alvo da intolerância geral do público. Suponhamos, por exemplo, que o
governo esteja com o povo, e que jamais pense em exercer qualquer
poder coercitivo exceto em concordância com o que pensa ser a voz do
povo. Mas nego o direito do povo de exercer tal coerção, seja por ele
mesmo, seja através do governo. O poder em si mesmo é ilegítimo. O
melhor governo tem tanto direito a ele quanto o pior. Esse poder é
ruim, e ainda pior quando exercido de acordo com a opinião pública do
que contrariamente a esta. Se toda a humanidade, exceto uma pessoa, tivesse uma
opinião, e essa pessoa tivesse uma opinião contrária, a humanidade não
teria mais justificativas para silenciá"-la do que ela para silenciar a
humanidade. Fosse uma opinião apenas um objeto pessoal, sem nenhum
valor exceto para o seu proprietário, e se o impedimento do usufruto dela
fosse apenas um dano privado, então poderia fazer alguma diferença se esse
dano atingisse apenas algumas pessoas ou muitas. Mas o
prejuízo característico de silenciar a expressão de uma opinião reside no fato de que isto é
roubar a raça humana, tanto a posteridade quanto a geração atual, tanto
aqueles que discordam da opinião quanto aqueles que a sustentam, e esses
ainda mais que os primeiros. Pois, se a opinião está certa, eles são
privados da oportunidade de trocar o erro pela verdade e, se
ela está errada, eles perdem a
percepção mais clara e vívida da verdade, produzida pela colisão
desta com o erro, um benefício tão grande quanto o primeiro. 

É necessário considerar separadamente essas duas hipóteses, 
pois o argumento correspondente a cada uma delas segue um caminho diferente. Nunca podemos
saber se a opinião que queremos silenciar é falsa, e se ela for falsa,
ainda assim silenciá"-la seria um mal.

Primeiro: a opinião que se tenta suprimir pela autoridade pode
possivelmente ser verdadeira. Aqueles que desejam suprimi"-la negam
obviamente a sua validade, mas não são infalíveis. Eles não possuem
autoridade para decidir a questão pela humanidade inteira e para excluir
todas as pessoas da possibilidade de julgá"-la. Recusar"-se a ouvir
uma opinião, por se estar certo de sua falsidade, é assumir que a sua
certeza é o mesmo que uma certeza absoluta. Todo silenciar da discussão
é uma presunção de infalibilidade. A sua condenação, portanto, pode se
fazer por este argumento comum, que não é mais fraco por ser comum. 

Infelizmente para o bom senso da humanidade, a constatação de sua falibilidade
está longe de ter peso para o seu julgamento prático, que na
teoria sempre atenta para ela, pois enquanto cada um sabe bem que é
falível, poucos pensam que seja necessário tomar alguma preocupação contra
sua própria falibilidade, ou admitir que uma opinião, que sentem
estar certa, pode ser um dos exemplos de erro ao qual eles
reconhecem estar propensos. Príncipes absolutistas, ou outros que
estejam acostumados a uma deferência sem limites, geralmente sentem uma
completa confiança em suas opiniões sobre quase todos os assuntos. Pessoas
que têm uma situação mais feliz, que de vez em quando têm suas opiniões
contestadas, e que não são completamente desacostumadas a serem
corrigidas quando em erro, colocam a mesma confiança sem limites apenas
naquelas opiniões compartilhadas por todos que as rodeiam ou para
aquelas com as quais habitualmente concordam: pois, em proporção à
falta de confiança que um homem tem no seu julgamento solitário, ele se
apoia, com confiança implícita, na infalibilidade do ``mundo'' em geral.
E o mundo, para cada indivíduo, significa a parte do mundo com que ele entra
em contato, o seu partido, a sua seita, a sua igreja, a sua classe na
sociedade: um homem pode ser chamado comparativamente de quase liberal e
de mente aberta, se o mundo para ele significar algo tão amplo como o seu
país ou a sua época. Nem sequer a sua fé nessa autoridade coletiva é abalada
pela consciência de que outras épocas, outros países, seitas, igrejas,
classes e partidos pensaram e mesmo continuam a pensar exatamente o
oposto que ele pensa. Ele devolve ao seu mundo a responsabilidade de \mbox{estar}
certo contra os mundos dissidentes das outras pessoas, e nunca lhe
causa problemas que apenas um mero acidente decidiu qual dentre esses
numerosos mundos é o objeto de sua confiança. E que as mesmas causas
que o fazem ser um adepto da Igreja Anglicana em Londres poderiam
tê"-lo tornado um budista ou um confucionista em Pequim. No entanto, é
evidente em si mesmo, e qualquer argumentação pode esclarecer isso, que
as épocas não são mais infalíveis que os indivíduos, e que cada época
sustentou opiniões que outras épocas tomaram não só como sendo falsas,
mas absurdas, e que é certo que muitas opiniões, agora comuns, serão
rejeitadas em épocas futuras, assim como outras, uma vez comuns, foram
rejeitadas pela época presente. 

A objeção mais esperada a esse argumento provavelmente tomaria a
seguinte forma: Não há uma presunção maior de infalibilidade ao se
proibir a propagação do erro do que em todas as outras coisas que são
feitas pela autoridade pública seguindo seus próprios juízos e
responsabilidades. O juízo é dado aos homens para que estes possam
utilizá"-lo. Porque é possível que seja utilizado erradamente, deve"-se dizer
que não deve ser utilizado de nenhuma maneira? Proibir o que se pensa
ser pernicioso não é afirmar que se está livre de erros, mas sim
cumprir um dever que as pessoas têm, mesmos sendo falíveis, isto é, o de agir
de acordo com a sua convicção consciente. Se nunca fossemos agir
baseados em nossas opiniões, porque essas poderiam estar erradas,
deixaríamos de tomar conta do que nos interessa e nenhum de nossos
deveres seria cumprido. Uma objeção que se aplica a todas as condutas
não pode ser uma objeção válida a qualquer conduta em particular. É
dever do governo e dos indivíduos formar, de modo cuidadoso, a opinião
mais verdadeira que possam e nunca impô"-las aos outros, a menos
que tenham certeza de estarem com a razão. Mas, se estiverem certos,
não é consciência mas covardia se abster de agir de acordo com as suas 
opiniões e permitir que doutrinas que eles honestamente pensam
ser perigosas para o bem da humanidade, quer nesta vida quer na outra,
sejam divulgadas sem nenhuma restrição, tudo porque outras pessoas, em
tempos menos iluminados, perseguiram opiniões que agora se acredita
serem verdadeiras. Vamos tomar cuidado para não cometer os mesmos
erros, pode"-se dizer assim, mas governos e nações têm cometidos erros
em coisas de que não se pode negar que sejam assuntos próprios ao
exercício da autoridade: foram lançados impostos ruins, provocadas
guerras injustas. Será que devemos então não lançar imposto nenhum e,
não importando a provocação, nunca travar uma guerra? Não há nada
parecido como a certeza absoluta, mas há segurança suficiente para os
propósitos da vida humana. Podemos e devemos assumir que a nossa
opinião seja verdadeira, com o fito de guiar a nossa conduta; não se
assume mais que isso quando se proíbe homens ímprobos de perverter a
sociedade com a propagação de opiniões que percebemos como falsas e perniciosas. 

Respondo que se está assumindo aqui muito mais. Há uma grande
diferença entre presumir que uma opinião é verdadeira por quê, tendo sido
dadas todas as oportunidades de se demonstrar que ela é falsa, tal não
ocorreu, e assumir que ela é verdadeira com o propósito de não permitir
que seja refutada. A completa liberdade de contradizer e refutar nossa
opinião é a condição que nos justifica a assumir que nossa opinião seja
verdadeira para finalidades de ação; em nenhum outro termo um ser
com faculdades humanas pode ter outra garantia de estar certo.

Quando consideramos ou a história da opinião ou a conduta ordinária da
vida humana, a que se deve que nem uma nem outra sejam piores do que
são? Certamente não à força inerente do entendimento humano, já que,
em um assunto que não for autoevidente, haverá sempre 99
pessoas que serão incapazes de julgá"-lo para cada uma que será
capaz; a capacidade da centésima pessoa é apenas comparativa, pois a
maioria dos homens eminentes de cada geração já sustentou
muitas opiniões que hoje sabemos errôneas, e fizeram, ou
concordaram com que se fizesse, muitas coisas que ninguém hoje
aprovaria. Como é que há então uma predominância de opiniões e
condutas racionais na humanidade? Se realmente existir essa
predominância --- e ela deve existir a menos que os assuntos humanos
estejam, e sempre tenham estado, num estado quase desesperador ---, ela se
deve a uma qualidade da mente humana, a fonte de tudo que é respeitável
no homem como um ser intelectual e moral, a saber, que os erros são
corrigíveis. O homem é capaz de retificar seus enganos através da
discussão e da experiência. Não apenas pela experiência. Devem
acontecer discussões para que se mostre como a experiência deve ser
interpretada. Opiniões e práticas errôneas cedem gradualmente diante do
fato e do argumento: mas, para que produzam qualquer
efeito na mente, os fatos e os argumentos devem ser 
trazidos e postos diante dela. Poucos fatos
são capazes de contar a sua própria história, sem comentários que façam
aparecer o seu significado. A força e o valor, portanto, do julgamento
humano depende desta única propriedade, a de que possa ser 
corrigido quando estiver errado; a \mbox{confiança} pode ser posta nela quando
os meios de correção são deixados constantemente à mão. No caso de uma
pessoa cujo julgamento é confiável, como se chegou a isso? Porque ela
manteve sua mente aberta para críticas às suas opiniões e conduta.
Porque tem sido seu costume o de ouvir tudo o que pode ser dito contra
ela, o de lucrar com o que nessas críticas foi justo, e o de expor
para si mesma e, dependendo do momento, para outras pessoas, a falácia
do que era falacioso. Porque ela sentiu que o único caminho pelo qual
um ser humano pode chegar próximo de saber tudo sobre um assunto é
ouvir o que pessoas de cada variedade de opinião podem falar sobre ele
e estudar todos os modos pelos quais um assunto pode ser visto por
todos os tipos de mente. Nenhum homem sábio jamais adquiriu sua
sabedoria de outro modo senão deste, nem está na natureza do intelecto
humano se tornar sábio por alguma outra maneira. O hábito frequente de
corrigir e completar a sua própria opinião comparando"-a com as de
outras pessoas, longe de causar dúvidas e hesitações quando se trata de
pô"-la em prática, é o único fundamento estável para se ter uma justa
confiança nessa opinião porque, tendo conhecimento de tudo o que pode
ser dito, ao menos de uma forma óbvia contra ela, e tendo tomado
posição contra todos os opositores --- sabendo que procurou as
objeções e dificuldades, ao invés de evitá"-las, e que não deixou de
tentar fazer brilhar sobre o assunto qualquer luz que pudesse, seja de
onde ela viesse ---, o homem tem o direito de pensar que o seu julgamento 
é melhor que o de qualquer outra pessoa que não passou por um processo semelhante. 

Seria demais esperar que aquilo que os sábios da humanidade, aqueles que
são os mais capazes de confiar no seu próprio julgamento, acham
necessário para \mbox{fundamentar} a sua confiança nele, fosse submetido à
coleção compósita de alguns poucos sábios e muitos tolos, chamada de
público? A mais intolerante das igrejas, a Igreja Católica, mesmo para
canonizar um santo, admite e ouve pacientemente ao ``advogado do
diabo''. Ao mais santo dos homens, assim parece, não podem ser admitidas
honras póstumas, até que aquilo que o diabo tenha a dizer sobre ele
seja ouvido e considerado. Se a filosofia newtoniana não pudesse ser
questionada, a humanidade não poderia sentir a completa segurança que
sente agora sobre a sua veracidade. As crenças que hoje pensamos serem as
mais sólidas não possuem outra salvaguarda na qual se apoiar, exceto o
permanente convite para que o mundo todo venha e as provem infundadas.
Se o desafio não é aceito ou, se aceito, as tentativas fracassaram, então
estaremos ainda muito longe da certeza absoluta, mas teremos feito o
melhor que o atual estado da razão humana permite, isto é, não teremos
negligenciado nada que pudesse dar à verdade uma possibilidade de nos
alcançar: se as listas forem mantidas abertas, podemos ter esperança de
que uma verdade mais completa exista, que será encontrada quando a
mente humana for capaz de recebê"-la. Enquanto isso podemos estar
cônscios de que chegamos tão perto da verdade quanto é possível nos
nossos dias. Essa é a quantidade de certeza alcançável por um ser
falível, e o único meio de consegui"-la. 

É de fato estranho que os homens possam admitir a validade do argumento
para a discussão livre mas objetem que esta seja levada ao limite,
não percebendo que a menos que as razões sejam boas para um caso
extremo, elas não serão boas para caso nenhum. Estranho que eles possam
imaginar que não estejam assumindo uma posição de infalibilidade
quando reconhecem que deve haver discussão livre em todos os assuntos
duvidosos, mas pensem que algum princípio particular ou doutrina tenha
a sua discussão proibida porque sabem com certeza que ela é
correta. Chamar qualquer proposição de correta, enquanto há alguém que
a negaria se pudesse, mas de fato não pode, é assumir que nós e aqueles que
concordam conosco somos os juízes da certeza, sem mesmo ouvir o outro lado. 

Na época presente --- que foi descrita como ``destituída de fé, mas temerosa
do ceticismo'' --- e em que as pessoas sentem não que suas opiniões são
verdadeiras, e sim que não saberiam o que fazer sem elas ---, os
protestos de uma opinião que queira ser protegida de ataques públicos
não se baseiam tanto na sua veracidade, porém na sua importância para
a sociedade. Alega"-se que certas crenças são tão úteis, para não
dizer indispensáveis ao bem"-estar, que não apenas é dever dos governos
sustentar essas crenças como também proteger quaisquer outros
interesses da sociedade. Em caso de tal necessidade e estando 
isso entre as suas obrigações, algo menor que a infalibilidade pode, assim se
afirma, permitir e mesmo obrigar os governos a agir segundo as suas
próprias opiniões, confirmadas pela opinião geral da humanidade. Também
é arguido com frequência, e mais frequentemente ainda pensado, que
ninguém além de homens maus poderia desejar enfraquecer essas
saudáveis crenças; que não pode haver nada de errado em se reprimir
homens maus e proibir o que apenas eles desejariam praticar. Esse modo
de pensar transforma a justificação da restrição a  discussões não em uma
questão sobre a  verdade das doutrinas, mas sobre a sua utilidade; e exalta a
si mesma por esse meio de escapar da responsabilidade de
afirmar ser um juiz infalível de opiniões. Mas aqueles que se
satisfazem desta maneira não percebem que a presunção de infalibilidade
foi apenas mudada de um ponto para outro. A utilidade de uma opinião é
em si mesma uma questão de opinião: tão contestável, tão aberta à
discussão e requerendo a mesma discussão que a própria opinião. Existe
a mesma necessidade de um infalível juiz de opiniões para decidir se
uma opinião é prejudicial como para decidir se ela é falsa, a menos que
a opinião condenada tenha tido todas as oportunidades de se defender. E
não adianta dizer que aos heréticos pode ser permitido manter a
utilidade ou a inofensividade de suas opiniões, apesar de proibidos de
manter que elas são verdadeiras. A verdade de uma opinião é parte de
sua utilidade. Se não pudermos saber se é desejável ou não que uma
proposição deva ser acreditada, é possível excluir a consideração sobre
se ela é verdadeira ou não? Na opinião não dos piores homens, mas dos
melhores, nenhuma crença que seja contrária à verdade pode ser
realmente útil: e pode você impedir esses homens de usar este argumento
quando acusados de renegar alguma doutrina que lhes
afirmam ser útil, mas que eles acreditam ser falsa? Aqueles que ficam ao
lado das opiniões recebidas nunca deixam de se valer de toda a vantagem
possível dessa argumentação, não se os encontra tratando da questão da
utilidade como se ela pudesse ser abstraída completamente da questão da
verdade: pelo contrário, justamente pelo fato de a sua doutrina ser ``a verdade''
é que o conhecimento ou a crença sobre ela é considerada indispensável.
Não pode ocorrer uma discussão justa sobre a questão da utilidade,
quando um argumento tão vital pode ser empregado por uns dos lados mas
não pelo outro. E, na verdade, quando a lei ou os sentimentos públicos
não permitem que a verdade de uma opinião seja disputada, eles são
também pouco tolerantes a respeito de sua utilidade. O máximo que
permitem é uma diminuição da sua necessidade absoluta, ou da positiva
responsabilidade da sua rejeição. 

Para ilustrar de forma mais ampla o dano que causa a negação de se ouvir
opiniões por que nós, pelas nossas próprias luzes, as condenamos, será
aconselhável que restrinjamos a discussão a um caso concreto, e eu
escolhi, por preferência, os casos que me são menos favoráveis ---
aqueles nos quais o argumento contra a liberdade de opinião, tanto no
que diz respeito à verdade quanto à utilidade, é
considerado o mais forte. Que as opiniões a serem impugnadas sejam
as crenças em um Deus e num estado futuro ou qualquer das doutrinas
herdadas de moralidade. Travar uma batalha neste terreno dará uma
grande vantagem para um adversário desonesto; uma vez que ele certamente
dirá (e muitos que não têm o menor desejo de serem incorretos irão
dizê"-lo para si mesmos): são essas as doutrinas que você afirma não
serem suficientemente certas para serem tomadas sob a proteção da lei?
A crença em um Deus, sentir"-se seguro dela, é uma das opiniões que você
sustenta ser uma amostra de infalibilidade? Mas me deve ser permitido
observar que não é o sentir"-se seguro de uma doutrina (seja esta qual
for) que eu chamo de presunção de infalibilidade. É esforçar"-se para
resolver essa questão pelos \textit{outros}, sem deixá"-los ouvir o
que pode ser dito pelo lado contrário. Eu denunciaria e reprovaria essa
pretensão da mesma forma, se ela fosse posicionada ao lado de minhas	\EP[-1]
mais solenes convicções. Não importando o quão positiva a persuasão de
alguém possa ser, não somente em falsidade mas em consequências
perniciosas --- não somente em consequências perniciosas, mas (para adotar
expressões que eu condeno fortemente) também em imoralidade e impiedade
de uma opinião, ainda assim se, seguindo o seu julgamento privado e,
apesar de apoiado pela opinião pública de seu país ou de seus
contemporâneos, esse alguém impede que uma opinião seja defendida, então ele
assume a própria infalibilidade. E longe de tal presunção ser menos
contestável ou menos perigosa por ser a opinião considerada imoral ou
ímpia; é nesses casos que essa presunção é a mais fatal. São exatamente
essas as ocasiões em que os homens de uma determinada geração
cometem aqueles enganos horrorosos que causam o espanto e o
horror da posteridade. Desses horrores encontramos memoráveis exemplos
na história, quando o braço da lei foi empregado para extirpar os
melhores homens e as mais nobres doutrinas, com deplorável sucesso no
que se refere aos homens, apesar de algumas das doutrinas terem
sobrevivido para serem (como se por ironia) invocadas na defesa de
conduta semelhante para com aqueles que discordavam \textit{delas}, ou
de duas interpretações ortodoxas. 

Nunca é demais para a humanidade ser relembrada de que houve uma vez um
homem chamado Sócrates, e que entre ele e as autoridades legais e a
opinião pública de sua época ocorreu um choque memorável. Nascido numa
época e num lugar plenos de grandezas individuais, este homem foi
reconhecido, por aqueles que melhor o conheciam bem como a seu tempo,
como o mais virtuoso dos homens de então, enquanto nós o conhecemos
como o líder e protótipo dos professores de virtude posteriores, a
fonte tanto da etérea inspiração de Platão quanto do judicioso
utilitarismo de Aristóteles, ``\textit{i maestri di color che sanno}'',\footnote{ Dante, \textit{Inferno}, 
Canto \textsc{iv}, verso 1131. [\versal{N.T.}]} as duas fontes
principais da ética e de toda a filosofia. Esse mestre reconhecido de
todos os pensadores eminentes que viveram desde então --- cuja fama, que
cresce ainda depois de mais de dois mil anos, excedeu a de todos os que
fazem o nome de sua cidade natal tão ilustre ---, foi executado pelos
seus concidadãos depois de ser judicialmente condenado por impiedade e
imoralidade. Impiedade, por negar os deuses reconhecidos pelo Estado;
de fato, seu acusador alegava (ver a \textit{Apologia})\footnote{ \textit{Apologia 
de Sócrates}, de Platão. [\versal{N.T.}]}  que Sócrates
não acreditava em nenhum deus. Imoralidade, por suas doutrinas e
instruções, por ser um ``corruptor da juventude''. Destas acusações, 
pelo que se pode saber, o tribunal honestamente o declarou culpado, e
condenou a ser executado como um criminoso o homem que, dentre todos que
então viviam, merecia o melhor da humanidade. 

Citemos o único exemplo de iniquidade judicial cuja menção, depois
da morte de Sócrates, não constituiria um anticlímax: o acontecimento que teve
lugar no Calvário mais de 1800 anos atrás. O homem que
deixou na memória daqueles que testemunharam a sua vida e as suas palavras
tal impressão de grandeza moral, o homem a  que os dezoito séculos 
seguintes prestaram homenagens como se fosse o Todo Poderoso em
pessoa, foi da forma mais reles executado; e como o quê? Como um
blasfemador. Os homens não somente entenderam mal o seu benfeitor, mas
o tomaram como sendo justamente o contrário do que ele era, e o
trataram como aquele prodígio de impiedade que, pelo tratamento que
lhe deram, agora se pensa que eles são. Os sentimentos com os quais a
humanidade atualmente percebe esses lamentáveis acontecimentos,
especialmente o segundo deles, a torna extremamente injusta no
julgamento dos infelizes atores que neles tomaram parte. Esses não
foram, pelo que parece, homens ruins --- não piores do que os homens
comumente são, mas até o contrário: homens que tinham a plena medida, e
às vezes mais do que a plena medida, dos sentimentos religiosos, morais
e patrióticos de sua época e de seu povo, o tipo de homens que em todas
as épocas, incluindo a nossa, têm a chance de passar pela vida
como respeitáveis e impolutos. O sumo sacerdote que rasgou seus votos quando
palavras que, de acordo com as ideias de seu país constituíam a culpa
mais negra, foram proferidas, estava muito provavelmente sendo tão
sincero em seu horror e indignação quanto a generalidade dos homens
pios e respeitáveis o são agora nos sentimentos morais e religiosos que
professam; muitos dos que agora tremem diante da conduta do sumo
sacerdote, se tivessem vivido naqueles tempos e nascido judeus, teriam
agido precisamente como ele. Cristãos sinceros que estejam tentados a
pensar que aqueles que apedrejaram até a morte os primeiros mártires
devem ter sido homens piores do que eles são, deveriam se lembrar que
um dos perseguidores foi São Paulo. 

Acrescentando mais um exemplo, o mais espantoso de todos, se o valor de um
erro puder ser medido pela sabedoria e virtude daquele que incorre nele. Se
jamais algum homem que deteve algum poder teve motivos para pensar
sobre si mesmo que era o melhor e o mais sábio dentre seus
contemporâneos, esse foi o imperador romano Marco Aurélio. Monarca
absoluto de todo o mundo civilizado, ele conservou durante toda a sua
vida não apenas a mais impecável justiça como também, o que não seria
tanto de se esperar devido à sua educação estoica, o mais tenro dos
corações. Os poucos sentimentos que lhe foram atribuídos estavam todos
do lado da indulgência, enquanto os seus escritos, a mais elevada
produção ética da mente antiga, dificilmente podem ser vistos como
diferentes, se é que possuem mesmo alguma diferença, dos mais
característicos ensinamentos de Cristo. Esse homem, um cristão melhor, 
exceto no sentido dogmático da palavra, que quase todos os
ostensivos soberanos cristãos que reinaram desde então, perseguiu o
cristianismo. Colocado no ápice de todas as conquistas prévias da
humanidade, possuidor de um intelecto aberto e sem grilhões, com um
caráter que o levou por si mesmo a encarnar em seus escritos morais o
ideal cristão, ainda assim ele falhou em ver que o cristianismo era
para ser um bem e não um mal para o mundo, com os deveres de que estava profundamente 
imbuído. Ele sabia que a sociedade de então se encontrava num estado deplorável. 
Mas sendo ela como era, ele viu, ou achou que viu, que ela estava sendo 
sustentada e impedida de se tornar ainda pior pela crença e reverência às divindades
estabelecidas. Como governante da humanidade, ele considerava ser seu
dever não permitir que a sociedade se despedaçasse, e não percebeu que se
os laços existentes fossem desfeitos, outros poderiam ser formados 
para ligar a sociedade novamente. A nova religião abertamente visava
dissolver esses laços: portanto, a menos que fosse seu dever adotar
essa religião, era seu dever destruí"-la. Uma vez que a
teologia do cristianismo não lhe parecia ser verdadeira ou de origem
divina, uma vez que essa estranha história de um Deus crucificado não lhe
era crível, e um sistema que afirmava se sustentar inteiramente sobre
uma fundação que para ele era totalmente inacreditável, ele não pôde
prever que o cristianismo era o agente renovador que, feitos todos
os descontos, acabou provando ser; o mais gentil e amigável dos
filósofos e governantes, sob um solene senso de dever, autorizou a
perseguição do cristianismo. Para mim, esse é um dos mais lamentáveis
fatos de toda a história. É amargo pensar quão diferente teria sido o
cristianismo, se a fé cristã tivesse sido adotada como a
religião do império sob os auspícios de Marco Aurélio e não sob os de
Constantino. Mas seria igualmente injusto para com ele, e uma
falsidade, negar que toda razão que pudesse ser apresentada
para a punição de ensinamentos anti"-cristãos foi levada
em conta por Marco Aurélio, tal como ele fazia com a propagação da
cristandade. Nenhum cristão acredita mais firmemente que o ateísmo é
falso e leva à dissolução da sociedade do que Marco Aurélio;
acreditava nas mesmas coisas a respeito do cristianismo, ele que, de
todos os homens que viviam então, poderia ser pensado como o mais
capaz de entender isso. A menos que alguém que aprove a punição pela
propagação de opiniões se lisonjeie a si mesmo imaginando ser mais
sábio e melhor do que Marco Aurélio --- mais conhecedor da
sabedoria de seu tempo e elevado em seu intelecto para além dela,
mais consciencioso em sua busca pela verdade, ou mais fiel em sua
devoção a ela quando encontrada ---, que essa pessoa se abstenha da dupla
presunção de infalibilidade, a sua e a da multidão, como o grande
Antoninus fez, com resultados tão infortunados. 

Ciente da impossibilidade de defender o uso de punições para a repressão
de opiniões irreligiosas por qualquer argumento que não acabasse também
por justificar Marco Aurélio, os inimigos da liberdade religiosa,
quando pressionados duramente, de vez em quando aceitam esta
consequência, e dizem, juntamente com o Dr. Johnson, que os
perseguidores da cristandade estavam certos; que a perseguição é uma
prova pela qual a verdade deve passar, e na qual ela sempre é bem
sucedida, as punições legais sendo, no fim, impotentes contra
a verdade, embora às vezes beneficamente efetivas contra erros
danosos. Este é um argumento a favor da intolerância religiosa
suficientemente extraordinário e espantoso para que possa sem mais ser deixado de lado.

Uma teoria que afirme que a verdade pode ser justamente perseguida
por que a perseguição não pode lhe trazer nenhum mal, não pode ser
atacada como intencionalmente hostil à recepção de novas
verdades; mas não podemos elogiar a generosidade de seu tratamento para
com as pessoas com as quais a humanidade está em dívida por essas
verdades. Apresentar ao mundo algo que o interessa profundamente e do
qual ele era antes ignorante; provar a ele que andou em erro em algum
ponto vital de interesse temporal ou espiritual é um
importante serviço que um ser humano pode render aos outros
e, em certos casos, como nos dos primeiros cristãos e dos iniciadores da
Reforma, que pessoas que pensam como o Dr. Johnson acreditam ser o bem
mais precioso que poderia ser doado à humanidade. Que os autores de tão
esplendidos benefícios tivessem sido levados ao martírio, que as suas
recompensas tivessem sido a de serem tratados como os mais vis dos
criminosos, não é, segundo essa teoria, um erro deplorável e um infausto
acontecimento, pelo qual a humanidade deveria ficar de luto, mas sim o
estado de coisas normal e justificável. Aquele que propõe uma nova
verdade, de acordo com essa doutrina, deve ficar, como ficaram aqueles
que, entre os locrianos\footnote{ Antiga população grega. [\versal{N.T.}]}, 
propunham novas leis, com um laço em volta do   
pescoço, a ser instantaneamente apertado se a assembleia de cidadãos,
depois de ouvir suas razões, não adotasse, naquele mesmo momento e
local, as suas máximas. Pessoas que defendem essa maneira de se tratar
os que fazem o bem não podem ser consideradas como pessoas que dão muito valor aos
benefícios; penso que esta visão deste assunto se restringe àquele
tipo de pessoa que julga que novas verdades um dia podem ter sido
desejáveis, mas que atualmente já as temos em quantidade mais que suficiente. 

Mas, de fato, dizer que a verdade sempre triunfa sobre as perseguições
é uma daquelas mentiras agradáveis que os homens repetem uns para os
outros até que elas se transformam em lugares comuns, mas que toda a
experiência refuta. A história está cheia de exemplos de verdades
derrubadas por perseguições. Se não suprimida para sempre, pelo menos
recusada por séculos. Para mencionar apenas opiniões religiosas: a
Reforma apareceu pelo menos umas vinte vezes antes de Lutero, e foi
reprimida. Arnoldo de Brescia foi destruído. Frei Dolcino foi		%notas para Lutero, Arnoldo de Brescia, Frei Dolcino, Savanarola etc.
destruído. Savanarola foi destruído. Os albigenses foram destruídos.
Os valdenses foram destruídos. Os lollardos foram destruídos. Os
hussitas foram destruídos. Mesmo depois da época de Lutero, onde quer
que as perseguições tenham continuado elas foram bem sucedidas. Na
Espanha, Itália, em Flandres, no Império Austríaco, o protestantismo
foi extirpado, e muito provavelmente assim teria sido na Inglaterra, se
a rainha Mary tivesse vivido ou se a rainha Elizabeth tivesse morrido.
A perseguição sempre foi bem"-sucedida, exceto onde os heréticos
formavam um agrupamento forte demais para ser perseguido com sucesso.
Nenhuma pessoa razoável negaria que o cristianismo poderia ter
sido \mbox{extirpado} no Império Romano. O cristianismo se espalhou e se
tornou predominante por que as perseguições foram apenas ocasionais,
durando um curto período, e separadas por longos intervalos de quase
imperturbada propaganda. Não passa de fútil
sentimentalismo achar que a verdade, meramente como verdade, tem algum
poder inerente, que é negado ao erro, o de prevalecer diante das
masmorras e da estaca. Os homens não são mais zelosos com a verdade do que
o são com o erro, e uma aplicação suficiente de penas legais ou mesmo
sociais será geralmente bem"-sucedida para interromper a
propagação de ambos. A vantagem real que a verdade possui consiste em
que, quando uma opinião é verdadeira, ela pode ser extinta uma, duas,
ou muitas vezes, mas no decorrer dos tempos haverá pessoas que virão a
redescobri"-la, até que em uma dessas ocasiões a sua reaparição se
dará em um momento em que, por circunstâncias favoráveis, ela escapará das
perseguições até ter estabelecido uma posição que lhe permitirá resistir
a todas as tentativas subsequentes de suprimi"-la. 

Será afirmado que nós não matamos aqueles que introduzem novas opiniões:
ao contrário de nossos pais, não matamos os profetas, na verdade
construímos até sepulcros para eles. Verdade que não mais executamos
heréticos, e que a quantidade de aplicação de penas que o sentimento
moderno provavelmente toleraria, mesmo contra as mais asquerosas
opiniões, não seria suficiente para extirpá"-las. Mas não fiquemos
a nos lisonjear pensando que estamos mesmo livres da mancha da
perseguição legal. Crimes de opinião, ou pelo menos pela expressão de
opiniões, ainda existem na lei, e a aplicação de penas contra eles não
deixou de acontecer, mesmo em nossos dias, o que não torna
inacreditável que um dia elas não possam ser revividas com plena força.
No ano de 1857, num tribunal na região da Cornualha, um homem sem 
sorte,\footnote{ Thomas Pooley, Bodmin Assizes, 
31 de julho de 1857. Em dezembro ele foi perdoado pela Coroa. [\versal{N.A.}]}
do qual se dizia ser de uma conduta excepcional em todos os aspectos da
vida, foi sentenciado a 21 meses de cadeia por pronunciar e
escrever num portão algumas palavras ofensivas em relação ao
cristianismo. Na mesma época, um mês depois, no tribunal de Old Bailey,
em Londres, duas pessoas, em ocasiões separadas,\footnote{ George Jacob 
Holyoake, 17 de agosto de 1857; Edward Truelove, julho de 1857. [\versal{N.A.}]} foram rejeitadas para
a função de jurados, e uma delas grosseiramente insultada pelo juiz e
por um dos advogados, porque ambas declararam honestamente que não
possuíam nenhuma crença teológica, e, pela mesma razão, a uma terceira pessoa, um
estrangeiro,\footnote{ Baron de Gleicher, Tribunal de polícia Marborough"-street, 
4 de agosto de 1857. [\versal{N.A.}]} foi negada justiça contra um ladrão.
Essa negativa de compensação ocorreu devido à doutrina legal de que
nenhuma pessoa pode fornecer evidências a uma corte de justiça sem
antes professar a crença em Deus (qualquer deus já bastaria) e num
estado futuro; o que equivale a declarar as pessoas que não o fazem
como estando fora da lei, excluídas da proteção das leis, podendo não
só ser roubadas ou assaltadas com impunidade, se ninguém mais estiver
presente além delas mesmas ou de outras pessoas de opinião semelhante,
mas que todo o mundo restante pode ser roubado ou assaltado com impunidade,
se a prova do fato depender da evidência apresentada por essas pessoas.
Isto está baseado na presunção de que um juramento não tem valor se
dado por uma pessoa que não acredita numa vida futura; uma proposição
que demonstra muita ignorância da história da parte daqueles que
concordam com ela (desde que é historicamente verdadeiro que uma grande
proporção dos infiéis em todas as eras foi de grande
integridade e honra); e não seria defendida por ninguém que tivesse a
menor ideia de quantas pessoas tidas pelo mundo como dotadas de grande
reputação, tanto em virtudes quanto em realizações, são bem
conhecidas, pelo menos por seus íntimos, por serem não crentes. Além
disso, a regra é suicida, e joga fora seu próprio fundamento. Sob o
pretexto de que todos os ateístas devem ser mentirosos, ele admite o
testemunho de todos os ateístas que estejam dispostos a mentir, e
rejeita apenas o daqueles que preferem enfrentar a publicidade
confessando uma crença detestada ao invés de afirmar uma mentira. Uma
regra assim, que se condena a si mesma no que diz respeito ao seu
pretendido propósito, só pode ser mantida como um sinal de ódio, uma
relíquia da perseguição; uma relíquia que tem a peculiaridade de que a
qualificação para promovê"-la é a amostra clara de não merecer essa
qualificação. A regra, e a teoria que a implica, é tão insultante para
os fiéis quanto para os que não o são. Pois se aquele que não acredita
numa vida futura mente necessariamente, segue"-se que aqueles que
acreditam são impedidos de mentir, se é que realmente o são,
apenas pelo medo do inferno. Não impingiremos aos autores e defensores
dessa regra a injúria de supor que a concepção que eles formaram da
virtude cristã advém de suas próprias consciências. 

De fato, estes são os trapos e restos da perseguição, e pode"-se
pensar que são não tanto uma indicação do desejo de 
perseguir quanto um exemplo daquela frequente fraqueza da mentalidade inglesa, que
leva os ingleses a ter um estranho prazer na asserção de um
princípio ruim, quando eles já não são maus o bastante para pô"-lo de
fato em prática. Mas, infelizmente, não há segurança no estado mental do
público de que a suspensão das piores formas de perseguição legal, que
tem perdurado pelo período de uma geração, continuará. Nessa nossa época, a
quieta superfície da rotina é tão frequentemente perturbada por
tentativas de ressuscitar os males do passado quanto pela introdução de
novos benefícios. O que é proclamado atualmente como um
reavivamento da religião é sempre, pelo menos para as mentes estreitas e
incultas, no mínimo também um reavivamento do fanatismo, e onde há um
forte e permanente fermento de intolerância nos sentimentos de um povo,
o que sempre permanece nas classes médias deste país, pouco se precisa
para provocá"-lo a uma perseguição ativa contra aqueles que ele nunca
deixou de pensar como objetos apropriados para uma
perseguição.\footnote{ Um aviso claro pode ser retirado do grande fluxo de paixões de um
perseguidor, que se misturou com a ampla amostra dos piores aspectos do
nosso caráter nacional por ocasião da insurreição dos Sepoys. As
loucuras de fanáticos ou charlatães num púlpito podem ser indignas de
atenção; mas os líderes do partido evangélico anunciaram como seu
princípio para o governo de hindus e maometanos que
nenhuma escola na qual a Bíblia não seja ensinada será sustentada por
dinheiro público, e que, por consequência necessária, nenhum emprego
público será dado para cristãos, reais ou fingidos. Um
subsecretário de Estado (William N.~Massey) num discurso feito aos
seus eleitores em 12 de dezembro de 1857, parece ter dito ``A
tolerância para com a sua fé'' (a fé de centenas de milhões de súditos
britânicos), ``com a superstição que eles chamam de religião, pelo
governo britânico, tem tido o efeito de retardar a ascendência do nome
britânico, impedindo o saudável crescimento do cristianismo [\ldots]. A
tolerância foi a grande pedra fundamental das liberdades religiosas
deste país; mas que eles não abusem dessa preciosa palavra, tolerância.
Tal como ele a entende, ela significa a liberdade completa para todos,
liberdade de crença, \textit{entre cristãos, que adoram sob os mesmos
fundamentos}. Ela significa a tolerância para todas as seitas e
denominações de \textit{Cristãos que creem na mesma mediação}.'' Chamo a atenção
para o fato de que um homem que foi considerado apto para um alto cargo
no governo deste país, sob um ministério liberal, mantenha a doutrina
que todos aqueles que não acreditam na divindade do Cristo estão além
da tolerância. E que, depois dessa amostra imbecil, pode ter a
ilusão de que as perseguições religiosas passaram, para nunca mais
retornar? [\versal{N.A.}]}  Pois é isso --- esta é a opinião que os homens mantêm e os sentimentos
que acalentam, em relação aos que minimizam os sentimentos que lhes
parecem importantes ---, que faz com que este país não seja um lugar de liberdade
mental. Já há um bom tempo, a principal calamidade causada pelas penas
legais é a de que elas fortalecem o estigma social. É esse estigma que
é de fato efetivo, e tão efetivo que a confissão de opiniões que são
proibidas pela sociedade é muito menos comum na Inglaterra do que em
muitos outros países em que a declaração dessas crenças incorrem no risco
de punições judiciárias. Em relação a todas as pessoas exceto
aquelas cujas condições financeiras as tornam independentes da boa
vontade dos outros, a opinião, neste assunto, é tão eficaz quanto a
lei; os homens podem ser tanto aprisionados quanto excluídos dos meios
de ganhar a vida. Aqueles que têm o seu pão assegurado, e que não
desejam favores dos homens que estão no poder, nem de grupos de pessoas
ou do público, nada têm a temer da expressão aberta de suas opiniões,
exceto serem malvistos e malfalados, mas isso não exige
uma compleição heroica para ser suportado. Não há lugar para um apelo
\textit{ad misericordiam} por essas pessoas. Mas apesar de não infligirmos
tanto mal quanto dantes àqueles que pensam diferente, pode ser que
façamos tanto mal a nós mesmos como sempre, pelo tratamento que damos a
eles. Sócrates foi executado, mas a filosofia socrática se elevou como
o sol nos céus e espalhou sua luz por todo o firmamento intelectual.
Cristãos foram jogados aos leões, mas a igreja cristã cresceu como uma
alta e frondosa árvore,  sobressaindo"-se entre as mais antigas e menos
vigorosas e acabando por asfixiá"-las na sua sombra. Nossa intolerância
meramente social não mata ninguém, não extirpa nenhuma opinião, mas
induz os homens a escondê"-las ou se abster de qualquer esforço
ativo para difundi"-las. Entre nós, as opiniões heréticas não ganham
nem perdem terreno de forma perceptível com o correr das gerações;
elas nunca brilham para todos os lados, mas ficam fumegando nos
pequenos círculos de pessoas estudiosas e pensantes em que se
originaram, sem jamais iluminar os assuntos comuns da humanidade com
uma luz verdadeira ou enganadora. E assim se mantém um estado de
coisas muito satisfatório para certas mentes, já que, sem o
desagradável processo de multar ou prender alguém, nele todas as
opiniões ficam externamente imperturbadas, enquanto nada impede o
exercício da razão pelos dissidentes afligidos pela doença do pensar.
Um plano conveniente para manter a paz no mundo intelectual e para
manter todas as coisas andando como sempre. Mas o preço pago por esse
tipo de pacificação intelectual é o sacrifício de toda a coragem moral
da mente humana. Um estado de coisas em que os intelectos mais ativos
e inquisitivos acham melhor guardar os princípios genuínos e
fundamentos de suas convicções para si mesmos, tentando, quando se
dirigem ao público, adequar o quanto podem as suas conclusões às
premissas que intimamente já foram postas de lado, não pode apresentar
os caracteres indômitos e os intelectos lógicos e consistentes que uma
vez adornaram o mundo pensante. O tipo de homem que se pode encontrar
aí são ou apenas os conformados com os lugares"-comuns ou servidores
temporários da verdade, cujos argumentos sobre todos os grandes assuntos
levam em conta somente os ouvintes, e não aqueles argumentos de que se
convenceram de serem os verdadeiros. Aqueles que evitam essa
alternativa, o fazem estreitando os seus pensamentos e interesses em
coisas sobre as quais se pode falar, sem se aventurar na região dos
princípios, isto é, se limitam a pequenos assuntos práticos, o que
seria bom para eles, fossem as mentes humanas alargadas e
fortalecidas, o que nunca será de fato correto até que isso se dê;
ao posso que é abandonado aquilo que poderia fortalecer e alargar a mente dos
homens, especulações livres e audaciosas nos assuntos mais elevados. 

Aqueles para cuja visão essa reticência por parte dos heréticos não é um
mal devem considerar em primeiro lugar que a consequência disso é que
nunca ocorre uma discussão justa e completa das opiniões heréticas, e
que aquelas que não poderiam se sair bem em tal discussão, apesar de
impedidas de se espalhar, não desaparecem. Mas não é a mente dos
heréticos as que mais se deterioram com a supressão de toda a
investigação que não chega às conclusões ortodoxas. O maior mal
se faz com aqueles que não são heréticos, cujo desenvolvimento mental é
atrasado, e cuja razão se torna acovardada pelo medo da heresia. Quem
pode contar o que o mundo perde entre a multidão de intelectos
promissores combinados com caracteres tímidos, que não ousam seguir
qualquer trilha de pensamento audaciosa, vigorosa e independente, pois
isso poderia vir a ser considerado irreligioso
e imoral? Entre esses, \mbox{podemos} ocasionalmente ver algum homem,
profundamente consciencioso e com um entendimento sutil e refinado, que
gasta a sua vida fazendo sofismas com um intelecto que ele não pode
calar, e exaure os recursos da engenhosidade numa tentativa de
conciliar anseios de sua razão e consciência com a ortodoxia, o que
talvez ele não consiga nunca realizar. Ninguém pode ser um grande
pensador se não souber que, como um pensador, o seu dever primário é
seguir o seu intelecto até as conclusões, quaisquer que elas sejam.
A verdade ganha mais pelos erros daqueles que, com o
devido estudo e preparação, pensam por si mesmos, que pelas opiniões
verdadeiras daqueles que somente as sustentam porque não se dão ao
incômodo de pensar. Não é apenas ou principalmente para se formar
grandes pensadores que a liberdade de pensamento é necessária. Pelo
contrário, ela é tão ou ainda mais indispensável para capacitar os
seres humanos medianos a alcançarem a estatura mental da qual são
capazes. Houve já, e pode voltar a haver, grandes pensadores
individuais numa atmosfera de escravidão mental. Mas nunca houve, ou
poderá haver, numa tal atmosfera, um povo intelectualmente ativo.
Quando um povo qualquer conseguiu se aproximar provisoriamente dessa
situação, isso se deveu ao fato de que o temor da especulação
heterodoxa foi suspenso por algum tempo. Onde há uma tácita convenção
de que os princípios não devem ser discutidos, onde a discussão das
grandes questões com os quais a humanidade pode se ocupar é
considerada encerrada, não podemos esperar achar aquela alta
escala de atividade mental que tornou alguns períodos da história tão
marcantes. Quando o debate evitou os assuntos grandes e
importantes o suficiente para acender o entusiasmo a
mentalidade de um povo jamais elevou"-se de suas fundações, e tal impulso,
que então seria dado, elevaria mesmo as pessoas com os intelectos mais
comuns a algo com a dignidade de seres pensantes. Disso temos um bom
exemplo nas condições da Europa, nos tempos que se seguiram
imediatamente à Reforma; outro, mais limitado ao continente europeu e
para uma classe mais limitada, o movimento especulativo da segunda
metade do século \textsc{xviii}; e um terceiro, e mais breve ainda, a
fermentação intelectual na Alemanha durante a época de Goethe e Fichte.
Esses períodos diferem enormemente em relação às opiniões particulares
que desenvolveram, mas são semelhantes porque enquanto duraram
o jugo da autoridade foi rompido. Em cada um deles, um despotismo
mental envelhecido foi descartado, e nenhum novo havia ainda tomado o
seu lugar. O impulso dado por esses três períodos fez da Europa o que
ela é agora. Cada simples melhoria que ocorreu, seja na mente humana,
seja nas instituições, pode ser ligada distintivamente a um ou outro
desses períodos. As aparências já por algum tempo têm indicado que
todos esses três impulsos se exauriram, e não podemos esperar nenhum novo
começo até que reafirmemos nossa liberdade mental.

Passemos agora à segunda parte do argumento e, deixando de lado a
suposição de que alguma das opiniões recebidas possa ser falsa,
assumamos que elas sejam verdadeiras, e examinemos qual o valor do modo
com que elas provavelmente serão sustentadas, quando a sua verdade não
puder ser aberta e livremente propagandeada. Não importando com quanto
desgosto uma pessoa que tenha uma opinião formada deva admitir a
possibilidade de que sua opinião possa ser falsa, ela deve se deixar
levar pela consideração que, a despeito de quão verdadeira possa
ser, se não for discutida plenamente, frequentemente e sem receio,
será sustentada como um dogma morto, não como uma verdade viva.

Há uma classe de pessoas (felizmente não tão numerosa como dantes) que
pensa ser suficiente se uma pessoa concorda sem duvidar com aquilo que
elas pensam ser verdade, apesar de essa pessoa não ter o menor
conhecimento dos fundamentos de sua opinião e não poder defendê"-la 
coerentemente contra a mais superficial das objeções. Tais
pessoas, se conseguem fazer com que suas crenças sejam ensinadas pela
autoridade, obviamente pensam que nenhum bem, e algum dano, viria se
fosse permitido que elas fossem questionadas. Onde sua influência
prevalece, eles fazem com que seja quase impossível que as opiniões
recebidas possam ser rejeitadas de forma sábia e considerada, apesar
de elas poderem ser rejeitadas impulsiva e ignaramente, pois impedir
toda e qualquer discussão dificilmente é possível, e quando uma começa,
crenças baseadas em convicções tendem a ceder espaço diante da mais
remota aparência de um argumento. Deixando de lado essa possibilidade ---
assumindo que a verdadeira opinião está alojada na mente, mas em forma
de preconceito, uma crença independente e à prova de argumentos ---;
esse não é o jeito pelo qual a verdade deve ser defendida por um ser
racional. Isso não é conhecer a verdade. Uma verdade assim defendida
não passa de uma superstição a mais, acidentalmente sustentando"-se em
palavras que enunciam uma verdade. 

Se o intelecto e a capacidade de julgar da humanidade devem ser cultivados, algo que
os protestantes pelo menos não negam, no que essas faculdades poderiam
ser mais apropriadamente empregadas por alguém do que naquilo que mais
lhe interessa tanto, que considera necessário que tenha opiniões sobre
elas? Se o cultivo do entendimento consiste em algo, é em conhecer 
os fundamentos da opinião de cada um. O que quer que as
pessoas acreditem, em assuntos nos quais é da máxima importância se ter
a crença correta, elas devem ter a habilidade para defender suas
crenças pelo menos contra as objeções mais comuns. Alguém pode dizer
``Que sejam elas instruídas e \textit{ensinadas} sobre os fundamentos de suas
opiniões. Daí não se segue que opiniões sejam mera papagaíce porque
elas nunca foram contraditadas. Pessoas que aprendem geometria não
colocam simplesmente os teoremas na memória, mas compreendem e aprendem
também as demonstrações, e seria absurdo pretender que eles continuam
ignorantes dos fundamentos das verdades geométricas porque nunca
ouviram ninguém negá"-las e tentar prová"-las falsas.'' Sem dúvida: e
um ensinamento deste tipo é suficiente para um assunto como a
matemática, onde não há nada a ser dito sobre o lado errado da questão.
A peculiaridade da evidência das verdades matemáticas é a de que todo o
argumento se coloca de um lado apenas. Mas em todo assunto onde a
diferença de opinião é possível, a verdade depende de um balanceamento
a ser feito entre dois grupos de opiniões conflitantes. Mesmo na
filosofia natural sempre há alguma outra explicação possível dos mesmos
fatos; uma teoria geocêntrica ao invés de uma heliocêntrica, um
flogístico no lugar do oxigênio, e tem que ser demonstrado por que a
outra teoria não é a verdadeira, e até que isso seja demonstrado, e até
que saibamos como demonstrar, não compreendemos os fundamentos da nossa
opinião. Mas quando nos voltamos para assuntos infinitamente mais
complicados, para a moral, a religião, a política, as relações
sociais e os negócios da vida, três quartos dos argumentos de cada
opinião em disputa consiste em dissipar as aparências que favorecem
alguma outra opinião diferente. O maior dos oradores, exceto um, deixou
registrado que ele sempre estudava o caso de seus adversários com uma
intensidade tão grande quanto o seu próprio, se não maior. O que
Cícero praticava com o intuito de obter sucesso nos tribunais, deve ser
imitado por todos aqueles que estudam qualquer assunto para chegar à
verdade. Aquele que conhece apenas o seu lado do caso conhece pouco
dele. Suas razões podem ser boas, e pode não haver alguém que possa
refutá"-las. Mas se ele também é igualmente incapaz de refutar as
razões do outro lado, se ele não sabe pelo menos quem são os
adversários, ele não tem motivo para preferir uma opinião à outra. Para
ele, a posição racional seria a da suspensão do juízo, e a menos que
ele se contente com essa posição, ele ou será guiado pela autoridade,
ou adotaria, como quase todo mundo, o lado pelo qual sentisse mais
inclinação. Não é suficiente que ele ouça os argumentos de seus
adversários proferidos por mestres que estão do seu lado, apresentados da forma que
eles quiserem, acompanhados do que eles oferecem como refutações. Esse
não é o modo de fazer justiça aos argumentos, ou trazê"-los a um
contato real com sua própria mente. Ele deve ser capaz de ouvi"-los de
pessoas que realmente acreditam neles, que os defendem com sinceridade,
e fazem tudo o que podem em favor deles. Ele deve conhecer esses
argumentos em sua forma mais plausível e persuasiva; ele deve sentir
toda a força das dificuldades que a verdadeira visão do assunto deve
encontrar e derrotar; ou então ele nunca possuirá aquela parte da
verdade que faz frente e remove aquelas dificuldades. Noventa e nove
por cento daqueles que são chamados de homens instruídos estão nessa
posição, mesmo aqueles que podem fluentemente arguir sobre seus pontos
de vista. Suas conclusões podem ser verdadeiras, mas poderiam ser também
falsas, por tudo o que eles podem ser: eles nunca se puseram na posição
mental daqueles que discordam deles, e nunca consideraram o que essas
pessoas teriam para dizer e, por consequência, eles realmente não
conhecem a doutrina que defendem. Não conhecem dela as partes que
explicam e justificam o resto; as considerações que mostram como ou por que um
fato que aparentemente contradiz outro pode ser reconciliado;
ou o porque, de duas razões aparentemente fortes, uma e não a outra
deve ser preferida. Toda aquela parte da verdade que vira o jogo, e
decide o julgamento para uma mente completamente informada, que não 
a conhecem realmente; mas para aqueles que conhecem igual e imparcialmente
ambos os lados e esforçam"-se para ver as razões de ambos mais claramente. 
Tão essencial é essa disciplina para um entendimento real da moral e dos assuntos humanos,
que se os oponentes de todas as verdades importantes não existissem,
seria indispensável imaginá"-los, e lhes conceder os mais poderosos
argumentos que o mais esperto advogado do diabo poderia conjurar. 

Para diminuir a força dessas considerações um inimigo da discussão livre
poderia argumentar que não há nenhuma necessidade para a humanidade em
geral de conhecer e entender tudo o que pode ser dito contra
ou a favor de sua opinião por filósofos e teólogos. Que não seria
preciso, para os homens comuns, ser capaz de expor todas as embromações
e falácias de um oponente sutil. Já seria suficiente que houvesse alguém capaz
de respondê"-las, de forma que nada que desviasse de seu caminho
pessoas de pouca instrução permanece sem ser refutado. Que as mentes
mais simples, depois de aprenderem os fundamentos básicos das opiniões
inculcadas nelas, devem confiar numa autoridade para o resto, ficando
cientes que eles não possuem nem o talento nem o conhecimento para
resolver qualquer dificuldade que viesse a aparecer, podem se
tranquilizar que todas as dificuldades que surgirem serão, ou poderão
ser, respondidas por aqueles treinados para essa tarefa. 

Que se conceda a este ponto de vista sobre o assunto o máximo que pode
ser reclamado por aqueles mais satisfeitos com a quantidade de
conhecimento da verdade que deve acompanhar a crença nesta e, mesmo
assim, o argumento a favor da livre discussão de maneira nenhuma se
enfraquece. Pois este argumento também reconhece que a humanidade deve
possuir uma garantia racional de que todas as objeções foram
satisfatoriamente respondidas; e como as objeções podem ser respondidas
se o que deve ser respondido não pode ser expresso? Ou como pode a
resposta ser reconhecida como satisfatória, se os contraditores não
tiveram oportunidade de mostrar que ela não é satisfatória? Se não o
público, pelo menos os filósofos e teólogos que devem resolver as
dificuldades tem que se familiarizar com essas dificuldades na sua
forma mais complexa, e isso não pode acontecer exceto se elas puderem
ser livremente expressas, e postas sob as luzes mais favoráveis
possíveis. A Igreja Católica tem a sua maneira própria de lidar com esse
embaraçoso problema. Ela faz uma larga distinção entre aqueles que
podem receber as suas doutrinas baseados na convicção, e aqueles que
devem aceitá"-las em confiança. A nenhum, é bem verdade, é dada a
possibilidade de escolher o que vão aceitar; mas os clérigos, pelo
menos o tanto quanto possam ser confiáveis, podem, sob permissão e com
aprovação, conhecer os argumentos dos adversários para poder
respondê"-los, o que os faz ler os livros heréticos; aos leigos isso
é vetado, exceto se obterem uma difícil permissão especial. Essa
disciplina reconhece o conhecimento do caso inimigo como favorável aos
professores, mas encontra meios, consistentes com ela, de negá"-los ao
resto do mundo: concedendo à \textit{élite} mais cultura mental, mas
não mais liberdade mental, do que permite para as massas. Por esse
mecanismo ela consegue obter o tipo de superioridade mental que seu
propósito requer, pois se cultura sem liberdade nunca pode formar uma
mente aberta e liberal, ela pode formar um esperto \textit{nisi prius}
advogado de uma causa. Mas, nos países protestantes, esse recurso não
está disponível, já que os protestantes sustentam que, pelo menos em
teoria, a responsabilidade pela escolha da religião é de cada um, e não
pode ser atribuída aos professores. Além disso, no atual estado de
coisas é praticamente impossível que os escritos prontos para serem
ensinados possam ser postos fora do alcance dos ainda não instruídos.
Se os mestres da humanidade conhecem tudo o que
precisam conhecer, qualquer material deve ser liberado para ser
escrito e publicado sem restrições. 

Se, no entanto, a maléfica ausência de discussão livre,
quando as opiniões recebidas forem verdadeiras, estiver confinada a
deixar os homens ignorantes sobre as bases de suas opiniões, pode"-se
pensar que este é um mal intelectual, e não moral, e que não afeta o
valor das opiniões, vistas do lado de sua influência no caráter. No
entanto, o fato é que não são apenas as bases da opinião que são
esquecidas na ausência da discussão, mas muito frequentemente o próprio
significado da discussão. As palavras que a carregam cessam de sugerir
ideias, ou sugerem apenas uma pequena parte das ideias que
originalmente comunicavam. Ao invés de uma concepção vívida e uma
crença viva, restam apenas umas poucas sentenças decoradas; ou, se
sobra alguma parte, a casca e o casco somente do significado continuam, a
essência mais fina sendo perdida. O grande capítulo da história humana
que esse fato ocupa e preenche não pode ser suficientemente estudado e meditado.

Ele está ilustrado pela experiência de quase todas as doutrinas éticas e
crenças religiosas. Cheias de significado e vitalidade para aqueles que
as iniciam e seus discípulos diretos, o seu significado continua a ser
sentido com a mesma força, e talvez levado a uma consciência mais
completa, enquanto dura o conflito para dar à doutrina ou crença uma
ascendência sobre as outras crenças. Finalmente, ou ela vence, e se
torna a opinião comum, ou o progresso cessa; ela mantém o terreno
conquistado, mas não vai mais além. Quando algum desses resultados
torna"-se evidente, a controvérsia sobre o assunto diminui, e
gradualmente cessa de vez. A doutrina assumiu seu lugar, se não como
uma opinião recebida, pelo menos como uma das seitas ou divisões de
opiniões aceitas: aqueles que a sustentam geralmente a herdaram, não
se converteram a ela; e a conversão de uma dessas doutrinas para outra,
sendo um fato excepcional, ocupa pouco espaço nos pensamentos de seus
professores. Ao invés de estar, como no início, constantemente em
alerta, ou para se defender do mundo ou para tomá"-lo, seus fiéis
caíram na aquiescência, e nem sequer ouvem, se podem evitá"-lo, os
argumentos contra a sua crença, nem perturbam os dissidentes (se houver
algum) com argumentos a seu favor. É deste momento que se pode datar o
declínio no poder vivo da doutrina. Frequentemente ouvimos mestres de
todos os credos lamentando a dificuldade de se manter na mente
dos fiéis uma apreensão vívida da verdade que esses reconhecem
verbalmente, de modo que esta verdade pudesse penetrar os sentimentos e
adquirir um controle real sobre as condutas. Nunca se reclama dessa
dificuldade quando a crença ainda está lutando por sua existência:
mesmo o mais fraco de seus combatentes sabe e sente pelo que está
lutando, e qual a diferença entre essa e as outras doutrinas; e nesse
período da existência de toda crença, não poucas pessoas podem ser
encontradas que, tendo se conscientizado de seus princípios
fundamentais em todas as formas de pensamento, os pesaram e
consideraram em todos os aspectos importantes, experimentaram o pleno
impacto dela sobre o caráter que a fé naquela doutrina deve
ocasionar numa mente completamente imbuída dela. Mas quando se
torna uma crença hereditária, recebida passiva e não
ativamente --- quando a mente não é mais compelida no mesmo grau que
antes a exercitar seus poderes vitais nas questões que a crença
apresenta ---, há uma tendência progressiva a se esquecer toda a crença
exceto suas fórmulas, ou a dar a ela um consentimento morno e
desinteressado, como se sua aceitação por confiança dispensasse a
necessidade de realizá"-la conscientemente, ou de testá"-la pela
experiência pessoal; até ela quase deixar de estar ligada com a vida
interior do ser humano. É por isso que se veem esses casos, tão
frequentes hoje em dia que quase formam a maioria, nos quais o credo
permanece como que fora da mente, incrustando"-a e petrificando"-a
contra todas as influências dirigidas às partes superiores de nossa
natureza; manifestando o seu poder impedindo que alguma convicção
viva e fresca a perturbe, mas em si mesmo nada fazendo por seu
coração e mente, exceto permanecer vigilante para que continuem vazios.


Em qual extensão doutrinas intrinsecamente adequadas para fazer a mais
profunda impressão na mente podem permanecer nesta como uma crença
morta, sem jamais se realizar na imaginação, nos sentimentos, ou no
entendimento, pode ser exemplificado pela maneira pela qual a maioria
dos fiéis mantém a doutrina do cristianismo. Por cristianismo quero
aqui significar o que é tido como tal por todas as igrejas e seitas ---
as máximas e preceitos encontrados no Novo Testamento. Esses são
considerados sagrados, e aceitos como tal por todos os cristãos. No
entanto, não é exagero dizer que nem sequer um cristão em mil guia ou testa
a sua conduta comparando"-as com aquelas leis. O padrão ao qual uma
pessoa se reporta são os costumes de sua nação, sua classe ou sua
religião. Então ela tem, por um lado, uma coleção de máximas éticas,
tidas como tendo sido dadas a ela por uma sabedoria infalível como
regras para o seu governo, e por outro lado um conjunto de juízos e
práticas cotidianos, que se ajustam até certo ponto com algumas dessas
máximas, não tão bem com outras, que ficam em oposição direta com
outras, e que são, no final, um compromisso entre o credo cristão e os
interesses e sugestões da vida laica. Ao primeiro desses padrões ela
rende sua homenagem, ao outro a sua verdadeira aliança. Todos os
cristãos acreditam que abençoados são os pobres e humildes, e aqueles
que são maltratados pelo mundo, que é mais fácil um camelo passar pelo
buraco de uma agulha do que um rico entrar no reino dos céus, que não se 
deve julgar para não ser julgado, que não se deve jurar, que se deve
amar ao próximo como a si mesmos, que se alguém pede um casaco, que se lhe
deve dar também o manto, que não se deve pensar no amanhã, que
para serem perfeitos, eles devem vender tudo o que possuem e dar aos
pobres.\footnote{ Ver Lucas 6, 20---23 (também Mateus 5, 3 e ss.) e Mateus 19, 24; 7, 1;
5, 34 (confronte"-se com Tiago 5, 12); 19, 19; 6, 34 e 19, 21. [\versal{N.T.}]}
 Eles não são insinceros quando dizem acreditar nessas coisas. Eles
acreditam nelas, do modo que as pessoas acreditam no que sempre ouviram
ser elogiado e nunca discutido. Mas no sentido de uma crença viva que
regula a conduta, eles acreditam nessa doutrina até o ponto em que lhes
é usual fazê"-lo. As doutrinas em sua integridade são úteis para se
jogar pedras nos adversários com elas, e entende"-se que elas devem ser
expostas (quando possível) como motivos para qualquer ação que se pense
louvável. Mas qualquer um que os relembre que as máximas exigem uma
infinidade de coisas as quais eles nunca pensam em fazer, não ganharia
nada além de ser classificado entre aquelas pessoas detestáveis que
pretendem ser melhores que as outras. A doutrina não tem muita validade
para os crentes comuns  --- não é poder em suas mentes. Eles têm
um respeito habitual pelos sons delas, mas nenhum sentimento se
espalha das palavras para a coisa significada, e força a mente a
deixá"-las entrar, e fazer com que elas se conformem com a fórmula. Em
qualquer momento em que a conduta importa, eles procuram o Sr.~\textsc{a} \EP{1}
e \textsc{b} para lhes dizer o quão longe ir na obediência a Cristo. 

Podemos estar seguros que o caso não era o mesmo com os cristãos
primitivos. Tivesse sido assim, a cristandade nunca teria se expandido
de uma obscura seita dos desprezados hebreus até chegar a ser a
religião do Império Romano. Quando seus inimigos diziam ``Vede como
esses cristãos amam uns aos outros'' (uma observação que dificilmente
seria feita por alguém agora), os cristãos tinham certamente um
sentimento muito mais vívido do significado de sua crença que jamais
tiveram desde então. E este é provavelmente o motivo principal por que a
cristandade agora consegue tão poucos progressos em aumentar os seus
territórios e que, depois de dezoito séculos, ela esteja quase que
confinada apenas aos europeus e aos descendentes de europeus. Mesmo com
as pessoas mais religiosas, que são muito mais sinceras sobre
suas doutrinas, e dão um maior significado a grande parte delas do que
as pessoas em geral, acontece frequentemente que a parte disso que é
ativa nas suas mentes é aquela que foi obra de Calvino ou Knox, ou
alguma dessas pessoas que estão muito perto deles em termos de
caráter. Os ditos de Cristo coexistem passivamente em suas mentes,
dificilmente produzindo algum outro efeito além do que é causado quando
se ouve palavras amistosas e brandas. Sem dúvida, há inúmeras razões
pelas quais são as doutrinas que identificam uma seita que retém muito
de sua vitalidade, ao contrário daquelas das seitas reconhecidas, e
por que os mestres têm mais trabalho em manter o sentido destas
doutrinas vivo; mas uma razão certamente é a de que as doutrinas
peculiares sofrem mais questionamentos, e tem que ser defendidas mais
frequentemente contra seus opositores. Tanto os professores
como os aprendizes dormem em seus postos, tão logo não haja mais inimigos em campo. 

A mesma coisa é verdadeira, falando de forma geral, para todas as
doutrinas tradicionais --- tanto as de prudência e conhecimento da vida
quanto as de moral e religião. Todas as línguas e literaturas estão
cheias de observações gerais sobre a vida, tanto em relação ao que ela
vem a ser, quanto a como agir nela; observações que todos conhecem e
que todos repetem, ou ouvem com concordância, que são recebidas como
truísmos, mas das quais as pessoas pela primeira vez aprendem o sentido
quando a experiência, normalmente de um tipo doloroso, as torna uma
realidade para elas. Quantas vezes, refletindo diante de um infortúnio
ou desapontamento inesperado, uma pessoa se lembra de algum provérbio
ou dito popular, que lhe foi familiar durante toda a vida, cujo
sentido, se tivesse sido entendido antes como é agora, poderia tê"-la
salvo da calamidade. Há outras razões para isso, além da falta de
discussão, há muitas verdades cujo sentido pleno \textit{não pode} ser
compreendido até que alguma experiência pessoal as tornem familiar. Mas
muito do significado dessas verdades poderia ter sido compreendido, e o
que foi compreendido teria deixado marca mais forte na mente, se o
homem estivesse acostumado a ouvir os prós e contras delas debatidos por
gente que entende do assunto. A tendência fatal da
humanidade de deixar de pensar a respeito de algo quando não há
mais dúvidas sobre isso é responsável por metade dos seus erros. 
Um autor contemporâneo disse bem, referindo"-se ao 
``sono profundo de uma opinião decidida''. 

Mas, então (pode"-se perguntar), é a ausência de unanimidade uma
condição indispensável do conhecimento verdadeiro? Será necessário que
uma parte da humanidade deva persistir no erro, para que alguém seja
capaz de entender a verdade? Será que uma crença deixa de ser
verdadeira tão logo seja aceita por todos --- e uma proposição nunca seja
completamente compreendida e sentida exceto se alguma dúvida sobre ela
permanecer? Tão logo a humanidade tenha unanimemente aceitado uma
verdade, será que a verdade morre com ela? O mais elevado objetivo e o
melhor resultado da inteligência aperfeiçoada, até aqui se pensou
assim, é o de unir a humanidade mais e mais no reconhecimento de todas
as verdades importantes: será que a inteligência só existe enquanto não
tiver alcançado seu objetivo? Será que os frutos da vitória morrem
justamente quando a vitória é completa?

Não afirmo semelhante coisa. Na medida em que a humanidade progride, o
número de doutrinas que não são mais discutidas ou questionadas aumentará,
e o bem"-estar da humanidade quase que pode ser medido pelo número e
pela importância das verdades que alcançaram o ponto de serem
incontestáveis. O esgotamento, numa questão depois da outra, de
controvérsias sérias é um dos incidentes necessários para a consolidação da
opinião; uma consolidação tão salutar no caso das opiniões verdadeiras
quanto perigosa e ruim quando as opiniões são errôneas. Mas, apesar do
estreitamento gradual dos limites da diversidade de opiniões ser
necessário nos dois sentidos do termo, sendo ao mesmo tempo
inevitável e indispensável, não somos obrigados a concluir que todas as
consequências devem ser benéficas. A perda de tão importante ajuda para
uma apreensão vívida e inteligente de uma verdade, tal como é
proporcionada pela necessidade de explicá"-la ou de defendê"-la
contra seus oponentes, apesar de não ser suficiente para anulá"-lo, não
deixa de ser um importante recuo do benefício do seu reconhecimento
universal. Onde essa vantagem não pode mais existir, confesso que
gostaria de ver os professores da humanidade procurarem encontrar um
substituto para ela; algum esquema para tornar as dificuldades da
questão tão presentes para a consciência do aprendiz como se elas
estivessem sendo apresentadas para ele por um dissidente de escol, que
almeje a sua conversão. 

Mas ao invés de procurar esquemas com esse propósito, perderam"-se os
que se tinha antes. As dialéticas socráticas, tão magnificamente
exemplificadas nos diálogos de Platão, foram um esquema desse tipo.
Elas eram essencialmente uma discussão negativa das grandes questões da
filosofia e da vida, dirigida, com habilidade consumada, ao propósito
de convencer qualquer um que tenha adotado os lugares"-comuns das
opiniões, que ele não entende do assunto --- que ele ainda não
conseguiu dar um sentido definido para as doutrinas que professa, para
que, ao se tornar cônscio de sua ignorância, ele possa ser colocado no
caminho para alcançar uma crença estável, que repouse tanto na
apreensão do significado da doutrina quanto de suas evidências.
As disputas escolásticas da Idade Média tinham um objetivo de algum
modo semelhante. Elas intentavam fazer com que o pupilo entendesse a
sua opinião e (numa correlação necessária) a opinião oposta a ela, e
que pudesse reforçar o terreno de uma e confundisse o da outra. Esses
debates tinham um defeito incurável, já que as premissas utilizadas
eram tiradas da autoridade, e não da razão, e como uma disciplina da
mente eram muito inferiores às poderosas dialéticas que formavam os
intelectos dos \textit{Socratici viri},\footnote{ ``Varões socráticos'', expressão de Cícero. [\versal{N.T.}]} 
mas a mente moderna deve a ambas muito mais do que geralmente
quer admitir, e os atuais modos de educação nada apresentam que no mais
ínfimo grau supre o lugar seja de uma seja de outra. Uma pessoa que
obtenha toda sua instrução de professores ou de livros, mesmo que
escape da grande tentação de se contentar com bobagens, não está compelido
a ouvir ambos os lados, e justamente por isso está longe de
ser um atributo frequente, mesmo entre pensadores, conhecer ambos os
lados de uma questão, e a parte mais frágil do que todo mundo diz em
defesa de suas opiniões é o que se pretende ser uma resposta aos 
antagonistas. É a moda dos tempos atuais desprezarem a lógica negativa 
--- aquela que aponta as fraquezas da teoria e os erros da prática, sem
estabelecer verdades positivas. Tal crítica negativa seria de fato por
demais pobre como resultado final, mas como um meio para se conseguir
um conhecimento positivo ou convicção digna deste nome, ela não pode
ser valorizada em excesso; e, até que as pessoas estejam plenamente
treinadas nela, haverá muitos poucos pensadores de monta, e um baixo
grau geral do intelecto, exceto nos departamentos de especulação física
e matemática. Em qualquer outro assunto, a opinião de alguém não merece o
nome de conhecimento, exceto se a essa pessoa tal conhecimento foi imposto por
outros, ou passou ela própria o processo mental que lhe seria exigido
para que pudesse levar adiante uma ativa controvérsia com um oponente.
O que, portanto, quando ausente, é tão indispensável, mas tão difícil,
de criar, é pior do que absurdo deixar de lado quando se apresenta
espontaneamente por si mesmo! Se houver alguém que conteste a opinião
recebida, ou que o faria se a lei ou a opinião permitisse, devemos
agradecer"-lhe por isso, abrir nossas mentes para ouvi"-lo, e nos
alegrar que haja alguém que faça por nós o que nós deveríamos, se
tivesse alguma consideração pela certeza ou pela vitalidade de nossas
convicções, cumprir por nós mesmos com muito mais trabalho.
\linebreak
Falta ainda falar de uma das principais causas que torna a diversidade
de opiniões vantajosa. E que continuará a fazê"-lo até que a
humanidade entre num estágio de avanço intelectual que no presente
parece estar a uma incalculável distância. Temos até aqui considerado
duas possibilidades, a de que a opinião recebida possa ser falsa, e que
haja outra opinião consequentemente verdadeira; ou que a opinião
recebida seja verdadeira, e um conflito com o erro oposto torna"-se então
essencial para uma clara apreensão e um sentimento profundo de sua
verdade. Mas existe um caso mais comum do que esses; o de que as
doutrinas conflitantes, ao invés de ser uma verdadeira e a outra falsa,
compartilham a verdade entre elas, e que a opinião não conformista seja
necessária para suprir o restante da verdade, da qual a opinião
recebida corporifica apenas uma parte. As opiniões populares, em
assuntos não palpáveis aos sentidos, são frequentemente verdadeiras,
mas dificilmente ou nunca apresentam a verdade completa. Elas são parte da
verdade, às vezes uma grande, outras uma pequena parte, mas exagerada,
distorcida e separada das verdades que deveriam acompanhá"-la e
limitá"-la. As opiniões heréticas, por outro lado, geralmente fazem
parte das verdades suprimidas ou negligenciadas, tentando arrebentar as
amarras que as seguram e procurando a reconciliação com a verdade
contida na opinião comum, ou enfrentando esta como inimiga, e se apresentando,
com exclusividade semelhante, como a verdade completa. Esse último
caso é o mais frequente, já que na mente humana o lado único tem sido a
regra, e o multifacetado a exceção. Sendo assim, mesmo nas revoluções
de opinião, uma parte da verdade normalmente esconde"-se, enquanto outra
aparece. Mesmo o progresso, que deveria adicionar, na maior parte das
vezes apenas substitui uma verdade parcial e incompleta por outra, a
melhoria consistindo principalmente nisso, a de que o novo fragmento da
verdade é mais desejado, mais adaptado às necessidades da época, do
que aquele que ele substitui. Sendo o caráter parcial das opiniões
prevalecentes, mesmo que se fundamentando em bases verdadeiras, cada
opinião que encarne alguma parte da verdade que a opinião comum omite
deve ser considerada preciosa, seja qual for o grau de erro e confusão com
que aquela verdade possa estar misturada. Nenhum juiz sóbrio dos
assuntos humanos se sentirá obrigado a indignar"-se porque aqueles
que nos forçam a prestar atenção sobre o que poderíamos não notar também
deixam de notar algumas das verdades que percebemos. Ao contrário, ele
pensará que enquanto a verdade popular tiver apenas um lado, é mais
desejável do que não desejável que a verdade impopular tenha também os
seus defensores unilaterais; esses sendo normalmente os mais
enérgicos, e que provavelmente conseguirão que se preste atenção
relutante ao fragmento da sabedoria que eles anunciam como se fosse toda a sabedoria. 

Assim, no século \textsc{xviii}, quando quase todas as pessoas instruídas, e
todas aquelas que não o eram, mas que as instruídas lideravam, ficaram
perdidas na admiração ao que se chama de civilização, e nas maravilhas
da ciência moderna, da literatura e filosofia e, enquanto exageravam as
diferenças entre os homens dos tempos modernos e aqueles dos tempos
antigos, indulgiam"-se na crença de que, no seu todo, a diferença
estava a seu favor; com que choque salutar os paradoxos de Rousseau
explodiram como petardos no seu meio, deslocando a massa compacta de
opinião rasteira, forçando seus elementos a se recombinar numa forma
melhor e com ingredientes adicionais. Não que as opiniões então
correntes estivessem no seu todo mais longe da verdade do que as de
Rousseau estavam, pelo contrário, elas estavam mais perto dela,
contendo mais da verdade positiva, e muito menos erros. Não obstante,
jaziam na doutrina de Rousseau, e desapareceram na corrente da opinião
popular juntamente com esta, uma considerável quantidade de verdades
que eram exatamente o que a opinião popular necessitava, e esses foram
os depósitos deixados para trás quando a enchente baixou. O valor
superior da vida simples, os efeitos enervantes e desmoralizantes dos obstáculos e
hipocrisias da sociedade artificial, são ideias que nunca estiveram
inteiramente ausentes das mentes cultivadas desde que Rousseau
escreveu, e elas, no seu devido tempo, produzirão o efeito esperado,
apesar de atualmente precisarem ser afirmadas tanto quanto sempre, e
afirmadas por atos, pois as palavras, sobre este assunto, perderam
quase que completamente o seu poder.

Na política, que é quase um lugar"-comum, há um partido da ordem e
estabilidade, e um partido do progresso ou reforma, sendo ambos os
elementos necessários para um estado saudável da vida política, até que
um ou outro tenha alargado de tal modo sua capacidade de compreensão
mental de forma a ser igualmente o partido da ordem e do progresso,
sabendo e distinguindo o que deve ser preservado daquilo que deve ser
jogado fora. Cada um desses modos de pensar deriva sua utilidade das
deficiências do outro, mas é em grande medida a oposição mútua que os
confinam dentro dos limites da razão e da sanidade. A menos que as
opiniões favoráveis à democracia e aristocracia, à propriedade e à
igualdade, à cooperação e à competição, ao luxo e à abstinência, à
sociabilidade e à individualidade, à liberdade e à disciplina, e todos
os outros antagonismos da vida prática, sejam expressas com igual
liberdade, e propostas e defendidas com igual talento e energia, não há
nenhuma chance de ambos os elementos obterem o que lhes é devido, uma
escala certamente \mbox{subirá} enquanto a outra descerá. A verdade, nas
grandes preocupações práticas da vida, é, em grande parte, uma
questão de reconciliar e combinar opostos, e muito poucos têm mentes
suficientemente amplas e imparciais para fazer esses ajustes de forma
correta, e eles então têm de ser feitos pelo duro processo de uma luta
entre combatentes que estão sob bandeiras inimigas. Em qualquer das
questões recém"-enumeradas, se alguma das duas opiniões puder, 
melhor do que a outra, não ser meramente tolerada, mas
sim ser encorajada e sustentada, é aquela que acontece estar num
determinado tempo e lugar na posição de minoria. Esta é a opinião que
representa, por algum tempo, os interesses negligenciados, o lado do
bem"-estar humano que corre o perigo de obter menos do que deveria ser a
sua parte. Estou cônscio de que não há neste país nenhuma intolerância
a diferenças de opinião a respeito desses tópicos. Eles
estão sendo mencionados para mostrar, por exemplos múltiplos e
conhecidos, a universalidade do fato de que somente através da
diversidade de opiniões há, no atual estado do intelecto humano,
uma chance de equanimidade para todas as faces da verdade. Quando se
pode encontrar pessoas que formam uma exceção à aparente unanimidade do
mundo sobre um assunto, mesmo que o mundo esteja certo, será sempre
provável que os dissidentes tenham algo para dizer que valha a pena
ouvir, e que a verdade perdesse algo com o silêncio deles.

Pode ser objetado: ``Mas alguns princípios recebidos, especialmente sobre
os assuntos mais elevados e vitais, são mais do que meia"-verdade. A
moralidade cristã, por exemplo, é a verdade completa nesse campo, e se
alguém ensina uma moralidade que esteja em discordância com a cristã,
essa pessoa está completamente errada.'' Como este é, de todos os
casos, o mais importante, nenhum é mais adequado para testar a máxima
geral. Mas antes de dizer o que a moralidade cristã é ou não é, seria
desejável decidir o que se quer dizer por moralidade cristã. Se quer
significar a moralidade do Novo Testamento, imagino como é que alguém
que retire seu conhecimento do próprio livro pode supor que ele foi
anunciado ou intentado como uma doutrina moral completa. Os
Evangelhos sempre se referem a uma moralidade pré"-existente, e limitam
seus preceitos para casos particulares nos quais aquela moralidade
tinha que ser corrigida, ou superada, por outra moralidade mais ampla e
elevada; além disso, expressando"-se em termos bem gerais, muitas vezes
impossíveis de serem interpretados literalmente, e possuindo mais a
força expressiva da poesia ou eloquência do que a precisão da legislação.
Extrair deles um corpo de doutrinas éticas nunca foi possível sem
retirá"-las do Velho Testamento, isto é, de um sistema de fato
elaborado, mas em vários aspectos bárbaro, e destinado apenas a um povo
bárbaro. São Paulo, um inimigo declarado deste modo judaico de
interpretar a doutrina e preencher com ela o esquema de seu Mestre, da
mesma forma também possuía uma moralidade pré"-existente, a saber, a
dos gregos e romanos; e as suas recomendações aos cristãos, são, em grande
medida, um sistema de acomodação com essa moralidade, chegando mesmo a
aparentemente sancionar a escravidão. O que é chamado de moralidade
cristã, mas que deveria ser denominado moralidade ``teológica'', não foi
obra de Cristo ou dos apóstolos, mas teve uma origem muito posterior,
tendo sido montada pela igreja católica em seus primeiros cinco séculos
e, apesar de não ter sido implicitamente adotada pelos modernos e
protestantes, foi muito menos modificada do que se poderia
esperar. Esses últimos se contentaram, na verdade, em cortar fora as
adições que tinham sido introduzidas na Idade Média, cada seita pondo
no lugar das que foram retiradas outras novas, adaptadas às suas
características e tendências. Que a humanidade deva muito a esta
moralidade, e aos primeiros que a ensinaram, eu seria o último a negar,
mas não tenho escrúpulos em afirmar que ela é, em muitos pontos,
incompleta e unilateral, e que se ideias e sentimentos que ela não
sanciona não tivessem contribuído para a formação da vida europeia, os
assuntos humanos estariam hoje num estado muito pior do que estão. A
assim chamada moralidade cristã tem todas as características de uma
reação, sendo, em grande parte, um protesto contra o paganismo. Seu
ideal é negativo, mais do que positivo, passivo mais do que ativo;
inocência ao invés de nobreza; abstinência do mal ao invés de uma
enérgica busca do bem; em seus preceitos (como foi corretamente afirmado), o ``tu não
o farás'' predomina de forma indevida sobre o ``tu o farás''. No seu
horror da sensualidade, ela fez do ascetismo um ídolo, que foi
gradualmente transformado em um da legalidade. Ela aponta a esperança
pelos céus e as ameaças do inferno como os motivos apropriados
para uma vida virtuosa: isso é estar muito abaixo do que melhor tinha a
Antiguidade, e fazer o que ela indica torna a moralidade humana
egoísta, ao separar os sentimentos humanos sobre o dever dos interesses
das outras pessoas, exceto na medida em que o interesse de uma tenha um
interesse induzido por razões egoístas nas outras pessoas.
Essencialmente, esta é uma doutrina de obediência passiva, que inculca
a submissão a todas as autoridades constituídas, que não devem ser, é
claro, obedecidas quando ordenarem que se faça algo que a religião
proíba, mas contra as quais não se deve oferecer resistência, e muito
menos rebelar"-se, não importando a quantidade de males que cometam
contra nós. E enquanto na moralidade das melhores nações pagãs, o
dever para com o Estado tinha até um lugar desproporcional, que
infringia a liberdade do indivíduo, na ética puramente cristã, aquela
grande divisão do dever mal é notada ou reconhecida. É no
Corão, e não no Novo Testamento, que lemos a
máxima ``Um governante que aponta um homem qualquer para um cargo,
quando em seus domínios há um outro homem mais bem qualificado do que
ele, peca contra Deus e contra o Estado''. O pouco reconhecimento que a
ideia da obrigação para com o bem público consegue na moralidade
moderna é derivado de fontes gregas e romanas, não de fontes cristãs;
assim como o que quer que exista, na moralidade da vida privada, em
termos de magnanimidade, abertura, dignidade pessoal, e mesmo o senso
de honradez, advém da parcela puramente humana da nossa educação, e não
da religiosa, e nunca poderia ter crescido de um padrão ético no qual a
única coisa que vale, tal como é abertamente reconhecido, é a obediência. 

Estou longe de pretender que esses defeitos necessariamente são
inerentes à ética cristã, em todas as maneiras em que esta pode ser
concebida, ou que muitos dos requisitos de uma doutrina moral completa,
que essa ética não contém, são irreconciliáveis com ela. Menos ainda
insinuaria algo semelhante sobre as doutrinas e preceitos do próprio
Cristo. Creio que os ensinamentos de Cristo são, como os entendo,
tudo o que se esperava que fossem; não são inconciliáveis com nada que uma
moralidade mais ampla requeira; que tudo que é excelente na ética pode
ser trazido até esses ensinamentos, sem maior violência à sua linguagem
que a praticada por aqueles que tentaram deduzir desses ensinamentos um
sistema de conduta prática, qualquer que ele fosse. Mas é perfeitamente
consistente com isso acreditar que eles contém, e assim foi intentado
que fosse, apenas uma parte da verdade; que muitos elementos essenciais
de uma moralidade mais elevada estão entre as coisas que os
pronunciamentos registrados do fundador do Cristianismo não apresentam,
e nem se deveria esperar que o fizessem, e que foram postos de lado no
sistema de ética construído com base naqueles pronunciamentos pela
Igreja Cristã. E sendo assim, penso que é um grande erro
persistir nas tentativas de encontrar dentro da doutrina cristã uma
regra completa para nos guiar, algo que seus autores queriam 
sancionar e fazer valer, mas que somente em parte poderia
ser feito. Creio também que essa teoria estreita está se tornando um
grande mal prático, o que faz perder muito o valor do treinamento e
instrução morais que muitas pessoas bem intencionadas agora estão se
esforçando para promover. Temo muito que ao se tentar formar a mente e
os sentimentos de um modo totalmente religioso, descartando os padrões
seculares (como na falta de um nome melhor podem ser chamados) que até
aqui coexistiram com a ética cristã, e a suplementaram, recebendo algo
do espírito dela, e infundindo nesta algo do seu, o resultado será, o
que já está acontecendo agora, a formação de um tipo de caráter baixo,
abjeto, servil, que submetendo"-se o quanto puder ao que parece
ordenar a Vontade Suprema, será incapaz de se elevar à concepção, ou
mesmo de simpatizar com ela, da Suprema Bondade. Creio que outras
éticas, além daquelas que possam ser retiradas exclusivamente das fontes
cristãs, devem existir lado a lado com a ética cristã, para que
aconteça uma regeneração moral da humanidade, e que o sistema cristão
não forma uma exceção à regra de que, num estado imperfeito da mente
humana, os interesses da verdade exigem a diversidade de opiniões. Não
é necessário que, ao se deixar de ignorar as verdades morais que não
estejam incluídas no Cristianismo, se deva ignorar qualquer uma que o
Cristianismo contenha. Um preconceito assim, ou esquecimento, quando
ocorre é um mal, mas é um daqueles de que não se pode esperar que estejamos
sempre isentos, e que deve ser encarado como o preço a ser pago por um
bem inestimável. A pretensão exclusiva de parte da verdade ser o
todo deve ser rejeitada, e se um impulso de reação venha a tornar os
que resistem injustos por sua vez, essa unilateralidade, como a outra,
deve ser lamentada, mas também tolerada. Se os cristãos querem ensinar
os infiéis a serem justos com o cristianismo, eles deveriam ser justos
com os infiéis. Não é um serviço para a verdade tentar ocultar o fato,
conhecido por todos aqueles que tem um conhecimento o mais ordinário da
história literária, que uma enorme porção dos mais valiosos e nobres
ensinamentos morais foram obras de homens que não desconheciam a fé
Cristã, mas que a conheceram e a rejeitaram. 

Não sustento que o uso absolutamente indiscriminado da liberdade de se
enunciar qualquer opinião possível venha a por um fim aos males do
sectarismo religioso e filosófico. Toda verdade da qual os homens de
menor capacidade estejam convencidos certamente será expressa,
inculcada, e de muitos modos levará a ações, como se nenhuma outra
verdade existisse no mundo ou que, pelo menos, se alguma outra
existisse, não poderia de forma nenhuma limitar ou qualificar a
primeira verdade. Reconheço que a tendência de todas as opiniões de se
tornarem sectárias não é curada pelas mais livres das discussões, mas
frequentemente exacerbada por elas; a verdade que deveria ter sido, mas
não o foi, vista, sendo rejeitada de forma ainda mais violenta pois foi
expressa por pessoas tidas como oponentes. Mas não é sobre o apaixonado
partidário, mas sim sobre o mais calmo e mais desinteressado ouvinte que
a colisão de opiniões alcança um efeito benéfico. Não o conflito
violento entre partes da verdade, mas sim a silenciosa supressão de
metade dela, é que é o mal mais formidável; sempre há esperança quando as pessoas são
forçadas a ouvir ambos os lados; somente quando prestam atenção a um
lado apenas é que os erros se enrijecem em preconceitos, e a própria
verdade deixa de ter o efeito da verdade, ao ser inflada por falsidades.
E desde que há poucos atributos mentais mais raros do que a faculdade
judicativa que pode realizar um julgamento inteligente entre os dois
lados de uma questão, das quais apenas uma está representada por uma
advogado diante dela, a verdade não tem possibilidades exceto na
proporção em que cada lado dela, cada opinião que carregue em si uma
fração da verdade, não somente encontre defensores, mas seja defendida de
forma a ser ouvida.
\linebreak
Reconhecemos a necessidade para o bem"-estar mental da
humanidade (do qual todos os outros bens dependem) da liberdade
de opinião e liberdade de expressão das opiniões, por quatro motivos
distintos, que agora recapitularemos. 

Primeiro, que se alguma opinião é induzida ao silêncio, essa opinião
pode, por tudo o que podemos saber, ser verdadeira. Negar isso é
assumir a nossa própria infalibilidade. 

Em segundo lugar, apesar da opinião silenciada ser um erro, ela pode, e
muitas vezes acontece, conter uma parte da verdade, e desde que a opinião
geral ou prevalecente sobre qualquer assunto raramente ou nunca é a
verdade completa, é apenas pela colisão de opiniões adversas que o que
falta da verdade pode aparecer. 

 Em terceiro, mesmo se a opinião recebida não for somente verdadeira,
mas for toda a verdade, a menos que se permita que ela seja, vigorosa e
sinceramente, contestada, ela será, por muitos que a recebem,
sustentada como se fosse um preconceito, com pouca compreensão e
sentimento sobre seus fundamentos racionais. Não somente isso, mas, e
em quarto lugar, o significado mesmo da doutrina correrá o risco de
perder"-se, ou de se enfraquecer, privado de seus efeitos vitais no
caráter e nas ações: o dogma se torna uma profissão de fé meramente
formal, ineficaz para o bem. E atulha ainda mais o solo,
impedindo o crescimento de alguma convicção real e sinceramente
sentida, vinda da razão ou da experiência pessoal. 

Antes de deixarmos de lado o assunto da liberdade de opinião, é adequado
prestar alguma atenção naqueles que dizem que a livre expressão de
todas as opiniões deve ser permitida, sob a condição de que a maneira de
sua expressão seja temperada, não ultrapassando os limites de uma
discussão bem educada. Muito poderia ser dito sobre a impossibilidade
de se saber onde esses limites deveriam ser postos; pois se o teste for
a ofensa para com aqueles cuja opinião estiver sob ataque, penso que a
experiência mostra que a ofensa é feita sempre que o ataque for forte
e persuasivo, e que cada oponente que insiste com energia, e diante do
qual os defensores encontram"-se em dificuldades para responder aos
ataques, aparece a esses últimos, se o oponente mostra algum forte
sentimento sobre o assunto, como alguém destemperado. Mas esta
objeção, apesar de ser de considerável importância de um ponto de vista
prático, se mescla com uma objeção mais fundamental. Sem dúvida, a
maneira de se afirmar uma opinião, mesmo uma verdadeira, pode ser
objetável, e pode incorrer em justas e severas censuras. Mas as ofensas
maiores deste tipo são de tal forma que, exceto por uma autotraição
acidental, acaba sendo impossível conseguir condená"-las. A mais grave
delas é discutir de forma sofística, suprimindo fatos ou argumentos,
apresentar de forma equivocada os elementos do caso, e apresentar a
opinião contrária injustamente. Mas tudo isso, chegando ao mais alto
nível, é feito continuadamente em perfeita boa"-fé, por pessoas que não
são consideradas, e que em muitos outros respeitos podem não merecer
ser assim consideradas, ignorantes ou incompetentes, que dificilmente
será possível, sobre bases adequadas, carimbar a não"-representação como
algo moralmente culpável; e menos ainda poderia a lei querer se
imiscuir nesse tipo de controversa má"-conduta. A respeito do que se
chama comumente de discussão intemperada, a saber, invectivas,
sarcasmos, argumentos de cunho pessoal e assim por diante, a denúncia
dessas armas poderia arranjar mais simpatias se alguma vez fosse
proposto proibi"-las de forma igual para ambos os lados; mas o que se
quer mesmo é restringir o seu uso apenas contra a opinião
prevalecente: contra aquelas que estão por baixo o seu uso não só não
encontra nenhuma objeção generalizada, como ainda por cima aquele que
as utilizar receberá o elogio de possuir um zelo honesto e uma justa
indignação. Quaisquer que sejam os malefícios que surjam de seus usos,
o maior deles ocorre quando são empregados contra aqueles que,
comparativamente, não possuem nenhuma defesa; e qualquer que seja a
vantagem injusta que possa obter uma opinião ao usar esses métodos, o
ganho é quase que exclusivo das opiniões recebidas. A maior ofensa
deste tipo que pode ser cometida por um polemista é estigmatizar
aqueles que defendem a opinião contrária como pessoas más e
imorais. A calúnias deste tipo aqueles que sustentam opiniões
impopulares estão particularmente sujeitos, porque, de modo geral, eles
são poucos e não tem nenhuma influência, e ninguém mais, além deles
mesmos, tem interesse que a justiça lhes seja concedida; mas essa arma
é, pela natureza do caso, negada àqueles que atacam as opiniões
prevalecentes: eles não a podem usar com segurança e, mesmo que
pudessem, ela se voltaria contra eles próprios. De modo geral, as
opiniões contrárias às comumente recebidas somente vem a ser ouvidas se
apresentadas com uma linguagem moderada, e evitando"-se cautelosamente
a utilização de ofensas desnecessárias, e disso não se pode desviar o
menor grau sequer, sob pena de se perder terreno: enquanto que as
vituperações desmedidas empregadas pelo lado da opinião prevalecente de
fato impedem que pessoas professem opiniões contrárias, e até que
ouçam aqueles que o fazem. Portanto, em interesse da verdade e da
justiça, é muito mais importante que se restrinja o uso de uma
linguagem vituperativa desse lado do que do outro e se, por exemplo, se
fosse necessário escolher, seria muito mais necessário se desencorajar
ataques ofensivos contra a infidelidade do que contra a religião. No
entanto, é óbvio que a lei e as autoridades nada tem a fazer nesse
caso, enquanto que a opinião deve, em cada instância, determinar o seu
veredito de acordo com as circunstâncias de cada indivíduo;
condenando quem quer que, não importando o lado do argumento que
esteja, mostre falta de respeito ao advogar sua causa mostre falta de respeito, ou
malignidade, fanatismo ou intolerância; mas não deduzindo que esses
vícios advenham da parte contrária a nós da questão; e concedendo
honras meritosas a qualquer um, não importando as opiniões que tenha,
que tenha a tranquilidade para ver e a honestidade para mostrar como
seus oponentes, e as opiniões destes, realmente são, não exagerando nada
para desacreditá"-los, e nada escondendo que possa ser visto como
sendo favorável a eles. Está é a real moralidade da discussão pública:
e se frequentemente violada, fico feliz em pensar que muitos
debatedores as observam em grande parte, e que um número ainda maior, de
forma consciente, se esforça na sua direção.

\chapter[Da individualidade]{Da individualidade\break como uma das
formas\break de bem"-estar}

\textsc{Sendo essas} as razões que tornam imperativo que os seres humanos devam
ser livres para formar opiniões, e para expressá"-las sem
reservas, e sendo essas as danosas consequências para a natureza moral
do homem, a menos que esta liberdade seja concedida, ou restaurada a
despeito da proibição, vamos agora examinar se as mesmas razões não
requerem que os homens devam ser livres para agir de acordo com suas
opiniões --- para mantê"-las em suas vidas, sem impedimentos físicos ou
morais, causados pelos seus companheiros, desde que o risco seja por
sua própria conta. Essa última cláusula é evidentemente indispensável.
Ninguém acha que as ações devam ser tão livres quanto as opiniões. Ao
contrário, mesmo as opiniões perdem suas imunidades quando as condições
em que são expressas são tais que exprimi"-las leva a uma instigação
de algum ato maléfico. A opinião de que os comerciantes de grãos deixam
o povo passar fome, ou de que a propriedade é um roubo, pode permanecer
sem ser molestada enquanto apenas circular pela imprensa, mas pode
incorrer numa justa punição se pronunciada diante de uma multidão
excitada aglomerada em frente à casa de um comerciante de grãos, ou
quando oferecidas à mesma multidão por meio de um cartaz. Atos que de
uma maneira qualquer e sem causa justificável causam danos a outras
pessoas podem ser --- e nos casos mais importantes é imperativo que o
sejam --- controlados por sentimentos que lhes são desfavoráveis e, quando
tal for necessário, pela interferência ativa da humanidade. A liberdade
do indivíduo deve ser limitada dessa maneira; ele não deve tornar a si mesmo um
problema para as outras pessoas. Mas se ele não molesta os outros nas
coisas que lhes interessam, e meramente age de acordo com a sua
própria inclinação e julgamento nas coisas que lhe dizem respeito, então
as razões dadas anteriormente, que demonstram que a opinião deve ser
livre, mostram também que ao indivíduo deve ser permitido, irrestritamente, 
colocar as suas opiniões em prática, sob sua própria
responsabilidade. Que a humanidade não seja infalível, que as suas
verdades, na maior parte, sejam somente meias"-verdades, que a unidade
de opinião, a menos que o resultado da mais ampla e livre comparação
de opiniões diferentes, não seja desejável, e que a diversidade não seja
um mal, mas um bem, até que a humanidade seja mais capaz de reconhecer
do que agora o faz todos os lados da verdade; esses são princípios
aplicáveis aos modos de ação dos homens, não menos do que às suas
opiniões. Assim como é útil que enquanto a humanidade for imperfeita
haja diferentes opiniões, assim também devem ser diferentes as experiências da vida;
que deva ser dado livre raio de ação às personalidades variadas,
exceto no caso de danos aos outros; e que o valor de diferentes modos
de vida devam ser comprovados na prática, quando alguém pensar que é
capaz de testá"-los. Em suma, é desejável que a individualidade possa 
se afirmar nas coisas que não interessam aos outros de
forma cabal. Ali onde não é o caráter
próprio da pessoa, mas sim as tradições ou os costumes dos outros que
formam as regras de conduta, ali falta um dos principais ingredientes da
felicidade humana e o principal ingrediente do progresso individual e social.

Para manter esse princípio, a maior dificuldade que será encontrada não
estará na apreciação dos meios que levariam a um resultado reconhecido,
mas na indiferença em geral das pessoas com o próprio resultado. Se for
sentido que o livre desenvolvimento da individualidade é um dos
principais componentes do bem"-estar; que ele não é apenas um elemento
coordenado com tudo aquilo que é designado pelos termos civilização,
instrução, educação e cultura, mas que é uma parcela necessária e uma
condição para todas essas coisas, então não deveria haver perigo de que a
liberdade pudesse ser desvalorizada, e o ajuste dos limites entre
ela e o controle social não deveria apresentar nenhuma dificuldade
extraordinária. Mas o ruim é que a espontaneidade individual raramente
é reconhecida pelos modos comuns de pensamento como dotada de valor
intrínseco ou merecedora de alguma atenção por si mesma. Estando a maioria 
satisfeita com os modos da humanidade tais como eles são agora
(já que ela é que os faz como são), ela não pode compreender por que esses
modos não podem ser bons para todos; além disso, a espontaneidade não
faz parte do ideal da maioria dos reformadores sociais e morais, pelo
contrário, ela é vista com inveja, como uma obstrução
problemática e talvez rebelde à aceitação daquilo que esses
reformadores, no seu julgamento, pensam que é o melhor para a
humanidade. Poucas pessoas, mesmo fora da Alemanha, compreenderam o
significado da doutrina que Wilhelm Von Humboldt, tão eminente como
\textit{savant} quanto como  político, expõe no texto de um tratado
 --- que ``o objetivo do homem, o que é prescrito pelos eternos e
imutáveis ditames da razão, e não é \mbox{sugerido} por vagos e passageiros
desejos, é um maior e mais harmonioso desenvolvimento de seus poderes,
até que cheguem a um todo completo e consistente'', e que, portanto, o
objetivo ``para o qual cada ser humano deve dirigir incessantemente os seus
esforços, e para o qual especialmente aqueles que querem influenciar os
outros devem estar alertas, é a individualidade do poder e
desenvolvimento''; e para tal existem dois requisitos, ``a liberdade e
uma variedade de situações''; e da união desses surge ``o vigor
individual e a múltipla diversidade'', que se combinam na 
``originalidade''.\footnote{ \textit{The Spheres and Duties of Government}, 
translated from the german of Baron Wilhelm von Humboldt by Joseph Coulthard. 
Londres: John Chapman, pp. 11-13. [\versal{N.A.}]}

No entanto, por pouco que as pessoas possam estar acostumadas com uma
doutrina como a de Von Humboldt, e por mais surpreendente que possa ser para
elas encontrar um valor tão elevado sendo atribuído à individualidade,
a questão, deve"-se pensar assim, é apenas uma questão de grau. Não há
ninguém que possua uma ideia de excelência na conduta que afirme que as
pessoas não devem fazer nada além de copiar umas às outras. Ninguém
afirmaria que as pessoas não devem pôr, no seu modo de vida e na
condução de seus negócios, marca nenhuma de seus julgamentos próprios
ou de seus caracteres individuais. Por outro lado, seria absurdo
pretender que as pessoas devem viver como se nada tivesse sido
conhecido no mundo antes de aparecerem nele; é como se a experiência
nunca tivesse mostrado que um modo de existência ou de conduta é
preferível a outro. Ninguém nega que as pessoas devem ser ensinadas e
treinadas na juventude, de forma a conhecer e poder se beneficiar dos reconhecidos
resultados da experiência humana. Mas é um privilégio do
ser humano e a sua própria condição, tendo atingido a maturidade
de suas faculdades, usar e interpretar a experiência à sua própria
maneira. Cabe a ele descobrir qual parte da experiência recolhida é
aplicável às suas próprias circunstâncias e caráter. As tradições e os
costumes dos outros são, até certo ponto, evidências do que a
experiência ensinou a \textit{eles}; evidências presuntivas e que, como tais,
possuem certa qualificação para serem ouvidas; contudo, em primeiro lugar,
a experiência deles pode ser muito estreita, ou eles podem não tê"-la
interpretado da forma correta. Em segundo lugar, a interpretação deles
pode ser correta, mas inadequada para outras pessoas. Costumes
adequados foram feitos para circunstâncias costumeiras e caracteres comuns, e as suas
circunstâncias ou o seu caráter podem não ser comuns. Em terceiro, apesar dos
costumes poderem ser bons como costumes e adequados a uma determinada
pessoa, ainda assim se conformando com um costume somente \textit{por ser} um
costume, ele não a educa ou desenvolve nela quaisquer das qualidades que são
as prerrogativas distintas de um ser humano. As faculdades humanas da
percepção, do julgamento, do sentimento discriminativo e mesmo da preferência
moral só podem ser exercidas quando se faz uma escolha. Aquele que faz
algo só porque assim é o costume, não faz uma escolha. Ele não aumenta a
sua prática, seja para discernir seja desejar o que é o melhor. O
mental e o moral, tal como os poderes dos músculos, são aprimorados
apenas quando usados. As faculdades não são usadas quando se faz
algo meramente por que outros o fazem do mesmo jeito, não mais do que
quando se acredita em algo por que outras pessoas também acreditam no
mesmo. Se os fundamentos de uma opinião não são conclusivos para a
razão de uma pessoa, a razão dela não sairá fortalecida, mas muito
provavelmente ficará mais fraca, quando tal motivo é aceito; e se os
incentivos para agir assim não forem concordantes com os sentimentos e o
caráter dessa pessoa (isso quando os afetos ou os direitos dos outros
não estiverem em questão), então tudo isso tornará seus
sentimentos e caráter inertes e túrpidos, ao invés de ativos e energéticos. 

 Aquele que permite ao mundo, ou à parte dele onde lhe coube viver, que
escolha seu plano de vida não necessita de nenhuma outra faculdade que
a da imitação de tipo simiesca. Aquele que escolhe o plano por si mesmo
emprega todas as suas faculdades. Ele deve usar a observação para ver,
o raciocínio e o julgamento para prever, a discriminação para decidir
e, quando tiver decidido, firmeza e autocontrole para manter a decisão
tomada. Essas qualidades são requeridas e utilizadas exatamente em
proporção com as partes de sua conduta determinadas por ele de acordo com o
seu julgamento e sentimento, e que são muitas. É possível que ele seja
guiado por um bom caminho, a salvo de preocupações, sem que nada disso
seja necessário. Mas qual seria o seu valor comparativo como ser humano?
É realmente importante não só o que os homens fazem, mas também que
tipo de homens eles são para fazê"-lo. Dentre as obras humanas em que a
vida é corretamente empregada para o aperfeiçoamento e embelezamento, certamente
a primeira em importância é o próprio homem. Suponha"-se que fosse
possível conseguir que casas fossem construídas, os cereais crescessem,
as batalhas travadas, causas julgadas, e mesmo igrejas erguidas e
orações oferecidas por máquinas --- por autômatos de forma humana ---, seria
uma perda considerável se trocar esses autômatos pelos homens e
mulheres que atualmente habitam as partes mais civilizadas do mundo, e
que seguramente são espécimes mais débeis do que a natureza pode e irá
produzir. A natureza humana não é uma máquina a ser construída segundo
um plano e posta para cumprir uma tarefa específica para ela; ela é
como uma árvore que necessita crescer e se desenvolver por si mesma
para todos os lados, de acordo com a tendência das forças internas que
fazem dela uma coisa viva.

É provável que se admita que é desejável que as pessoas
exercitem seu entendimento, e que uma adesão inteligente
ao costume, ou mesmo, ocasionalmente, um desvio inteligente
do costume, é melhor do que uma adesão cega e simplesmente mecânica. 
Em certa medida, admite"-se que o nosso entendimento
deve ser particular, mas não há a mesma disposição para admitir
que nossos desejos e impulsos devem ser também particulares; ou que
possuir impulsos particulares, seja qual for sua força, é algo que não uma
armadilha e um perigo. Ora, desejos e impulsos são parte de um
ser humano perfeito tanto quanto suas crenças e restrições, e impulsos fortes 
só são perigosos quando não forem devidamente equilibrados; ou quando um conjunto de objetivos
e inclinações é desenvolvido abundantemente, enquanto outros, com que deviam
coexistir, continuam fracos e inativos. Não é porque os desejos dos homens
são fortes que eles agem mal, é porque suas consciências
são fracas. Não há conexão natural entre os impulsos fortes e a
consciência fraca. A conexão natural é o contrário disso. Dizer que
os desejos e sentimentos de uma pessoa são mais fortes e mais variados do que
os de outra, é apenas dizer que ela tem mais matéria"-prima
humana, e portanto é mais capaz, talvez, de mais mal, mas
certamente, de mais bem. Impulsos fortes são apenas outro nome para a energia.
A energia pode ser aplicada de modos ruins; mas sempre pode ser feito mais bem
por uma natureza energética do que por uma indolente e impassível. Aqueles que
têm o sentimento mais natural, são sempre aqueles cujos sentimentos cultivados
podem ser os mais fortes. As mesmas fortes susceptibilidades que fazem
os impulsos pessoais mais vivos e poderosos, são também a fonte geradora 
do mais apaixonado amor pela virtude, e dos mais severos
mecanismos de autocontrole. É através do incentivo a eles que a sociedade
faz o seu dever e protege os seus interesses: sem rejeitar o material de
que os heróis são feitos, porque não sabe como fazê"-los. Uma pessoa
cujos desejos e impulsos são particulares --- são a expressão da sua própria
natureza, desenvolvida e modificada por sua própria cultura --- tem
um caráter. Alguém cujos desejos e impulsos não são
particulares, não tem mais caráter do que uma máquina a vapor.
Se, além de particulares, seus impulsos são fortes, e estão
sob o governo de uma vontade forte, ele tem um caráter enérgico.
Quem pensa que a individualidade dos desejos e impulsos não deve
ser incentivada a se desdobrar por si, deve admitir que a sociedade não precisa
de naturezas fortes --- o que não é o melhor para muitas pessoas que
têm caracteres fortes --- e que uma média geral alta de energia não é desejável.

%It will probably be conceded that it is desirable people 
%should exercise their understandings, and that an intelligent 
%following of custom, or even occasionally an intelligent deviation 
%from custom, is better than a blind and simply mechanical adhesion 
%to it. To a certain extent it is admitted, that our understanding 
%should be our own : but there is not the same willingness to admit 
%that our desires and impulses should be our own likewise ; or that 
%to possess impulses of our own, and of any strength, is anything but 
%a peril and a snare. Yet desires and impulses are as much a part of 
%a perfect human being, as beliefs and restraints : and strong impulses 
%are only perilous when not properly balanced ; when one set of aims 
%and inclinations is developed into strength, while others, which ought 
%to co"-exist with them, remain weak and inactive. It is not because men's 
%desires are strong that they act ill; it is because their consciences 
%are weak. There is no natural connexion between strong impulses and a 
%weak conscience. The natural connexion is the other way. To say that 
%one person's desires and feelings are stronger and more various than 
%those of another, is merely to say that he has more of the raw material 
%of human nature, and is therefore capable, perhaps of more evil, but 
%certainly of more good. Strong impulses are but another name for energy. 
%Energy may be turned to bad uses; but more good may always be made of 
%an energetic nature, than of an indolent and impassive one. Those who 
%have most natural feeling, are always those whose cultivated feelings 
%may be made the strongest. The same strong susceptibilities which make 
%the personal impulses vivid and powerful, are also the source from whence 
%are generated the most passionate love of virtue, and the sternest 
%self"-control. It is through the cultivation of these, that society both 
%does its duty and protects its interests : not by rejecting the stuff of 
%which heroes are made, because it knows not how to make them. A person 
%whose desires and impulses are his own—are the expression of his own 
%nature, as it has been developed and modified by his own culture—is 
%said to have a character. One whose desires and impulses are not his 
%own, has no character, no more than a steam"-engine has a character. 
%If, in addition to being his own, his impulses are strong, and aro 
%under the government of a strong will, he has an energetic character. 
%Whoever thinks that individuality of desires and impulses should not 
%be encouraged to unfold itself, must maintain that society has no need 
%of strong natures—is not the better for containing many persons who 
%have much character—and that a high general average of energy is not desirable.

%In some early states of society, these forces might be, and were, 
%too much ahead of the power which society then possessed of disciplining 
%and controlling them. 
Em alguns estados iniciais da sociedade, estas forças podem ser, e foram,
muito superiores ao poder que a sociedade possuía para discipliná"-las
e controlá"-las. Houve tempos em que o elemento da espontaneidade e
individualidade era excessivo, e o princípio social teve de lutar
ferozmente contra esse elemento. A dificuldade então era a de induzir
homens de corpos ou mentes fortes a obedecer às regras que exigissem
que eles controlassem seus impulsos. Para superar essa dificuldade, a
lei e a disciplina, como na luta dos Papas contra os Imperadores,\footnote{ Isto 
é, do Sacro Império Romano"-Germânico. [\versal{N.T.}]} quando
os primeiros proclamaram ter poder sobre o homem todo, exigindo
controle completo e total sobre sua vida. Foi a forma encontrada de
controlar os caracteres daqueles homens, já que a sociedade não
tinha encontrado outro modo que fosse capaz de controlá"-los. Mas,
agora, a sociedade já levou a melhor sobre a individualidade, e o perigo
que ameaça a espécie humana não é o excesso, mas a falta de
preferências e impulsos pessoais. As coisas mudaram muito desde quando
aqueles que eram poderosos por posição ou por realizações pessoais se
colocavam em conflito com as leis e os regulamentos e precisavam ser
cerceados com o fito de permitir que as pessoas que estivessem ao seu
alcance tivessem o mínimo de segurança. Em nossos tempos, da mais alta classe da sociedade até a
mais baixa, cada um vive sob os olhares de uma censura hostil e temida.
Não somente no que concerne aos outros, mas, no que concerne somente a
eles, o indivíduo ou a família não se perguntam: “O que eu prefiro? O
que combina melhor com meu caráter e minha disposição? O que permitiria
àquilo que eu tenho de melhor em mim crescer e se expandir?” Eles
se perguntam: “O que é mais adequado à minha posição? O que é
normalmente feito por pessoas de minha posição e posses?”, ou, o que é
ainda pior, “O que é normalmente feito por pessoas de posições e
circunstâncias superiores à minha?” Não quero dizer que eles escolham o
que é costumeiro ao invés do que agradaria às suas \mbox{verdadeiras}
disposições. Eles não têm nenhuma disposição, exceto por aquilo que é
o costumeiro. A própria mente assim se curva ao jugo, mesmo naquilo que
a pessoa faz por prazer a conformidade é a primeira coisa a ser levada
em conta, eles gostam de coisas em conjunto, exercem suas escolhas
apenas entre as coisas feitas normalmente; gostos particulares,
condutas excêntricas são rejeitadas do mesmo modo que os crimes, até
que, por não seguir sua própria natureza, acabam por não ter mais
natureza para seguir: as suas capacidades humanas se definham e
estiolam, elas se tornam incapazes de desejos fortes ou prazeres
próprios e geralmente não possuem nem sentimentos nem opiniões nascidas
deles mesmos. Agora, esta seria, ou não, a condição mais desejável da
natureza humana? 

Assim o é, para a teoria calvinista. De acordo com ela,
a grande ofensa do homem é a vontade própria. Tudo de bom de que a
humanidade é capaz advém da obediência. Não se tem escolha, deve"-se
fazer assim, e não de outra maneira: “O que quer que não seja um dever
é um pecado.” Sendo a natureza humana radicalmente corrupta, não
poderá haver redenção para uma pessoa até que a humanidade desta
seja liquidada. Para alguém que sustente essa teoria de vida, não é um
mal esmagar quaisquer das faculdades, capacidades e suscetibilidades
humanas: o homem não necessita de nenhuma outra capacidade, exceto
aquela de se render à vontade de Deus, e se ele usar qualquer de suas
faculdades para um outro propósito exceto aquele de fazer cumprir
essa suposta vontade de forma mais efetiva, então ficaria bem melhor
sem essa faculdade. Esta é a teoria do calvinismo, que é sustentada, de
um modo mais mitigado, por muitos que não se consideram calvinistas,
essa mitigação consistindo em se dar uma interpretação menos ascética
à alegada vontade de Deus, asseverando ser Sua vontade que a humanidade
deva justificar algumas de suas inclinações, certamente não da maneira
que as pessoas preferirem, mas sim como um modo de obediência, isto é,
do modo prescrito a elas pela autoridade e, portanto, pelas condições
necessárias do caso, do mesmo modo para todas.

Em algumas formas insidiosas, existe atualmente uma forte
inclinação para essa estreita teoria da vida e pelo limitado e torto
tipo de caráter humano que ela privilegia. Na certa com sinceridade,
muitas pessoas pensam que os seres humanos, assim manietados e
diminuídos, são como o seu Criador os designou para serem, assim como
muitos pensaram que as árvores ficam mais frondosas quando podadas em
formas de animais,\footnote{ Trata"-se da arte da topiaria, que molda
a folhagem das árvores de acordo com imagens geométricas, de animais
etc. [\versal{N.T.}]} do que como a natureza as fez. Mas,
se é parte de qualquer religião crer que o homem foi criado por um ser
bom, é mais consistente com essa fé crer que este ser criou todas as
faculdades humanas, que devem poder ser cultivadas e expandidas, não
extirpadas e consumidas, e que ele se delicia com uma aproximação maior
de suas criaturas à concepção ideal imbuída nelas com cada aumento de suas
capacidades de compreensão, ação e de aproveitamento. Há um tipo
diferente de excelência humana, além daquela do calvinismo, uma
concepção da humanidade que não vê a sua natureza como algo que 
lhe foi dado apenas com o propósito de ser negado. A “autoafirmação”
pagã é tanto um elemento do valor humano quanto a
“autonegação”\footnote{ Sterling's \textit{Essays}. [\versal{N.A.}]}
cristã. Há um ideal grego de autodesenvolvimento, com o qual o ideal
platônico ou cristão se mescla, mas não supera. Pode ser que seja
melhor ser um John Knox do que um Alcibíades, mas é melhor ser um
Péricles do que qualquer um dos primeiros, e um Péricles,
se tivéssemos um hoje em dia, não recusaria tudo de bom que tivesse
pertencido a John Knox. 

%It is not by wearing down into uniformity all that is individual 
%in themselves, but by cultivating it and calling it forth, within 
%the limits imposed by the rights and interests of others, that human 
%beings become a noble and beautiful object of contemplation ; and as 
%the works partake the character of those who do them, by the same 
%process human life also becomes rich, diversified, and animating, 
%furnishing more abundant aliment to high thoughts and elevating feelings, 
%and strengthening the tie which binds every individual to the race, by 
%making the race infinitely better worth belonging to. In proportion to 
%the development of his individuality, each person becomes more valuable 
%to himself, and is therefore capable of being more valuable to others. 
%There is a greater fulness of life about his own existence, and when 
%there is more life in the units there is more in the mass which is 
%composed of them. As much compression as is necessary to prevent the 
%stronger specimens of human nature from encroaching on the rights of 
%others, cannot be dispensed with ; but for this there is ample compensation 
%even in the point of view of human development. The means of development 
%which the individual loses by being prevented from gratifying his inclinations 
%to the injury of others, are chiefly obtained at the expense of the development 
%of other people. And even to himself there is a full equivalent in the better 
%development joi the social part of his nature, rendered possible by the 
%restraint put upon the selfish part. To be held to rigid rules of justice 
%for the sake of others, developes the feelings and capacities which have 
%the good of others for their object. But to be restrained in things not 
%affecting their good, by their mere displeasure, developes nothing valuable, 
%except such force of character as may unfold itself in resisting the restraint. 
%If acquiesced in, it dulls and blunts the whole nature. 
%To give any fair play to the nature of each, it is essential 
%that different persons should be allowed to lead different lives. 
%In proportion as this latitude has been exercised in any age, has that age 
%been noteworthy to posterity. Even despotism does not produce its worst 
%effects, so long as Individuality exists under it; and whatever crushes 
%individuality is despotism, by whatever name it may be called, and whether 
%it professes to be enforcing the will of God or the injunctions of men. 
%Having said that Individuality is the same thing with development, and 
%that it is only the cultivation of individuality which produces, or can 
%produce, well"-developed human beings, I might here close the argument: 
%for what more or better can be said of any condition of human affairs, 
%than that it brings human beings themselves nearer to the best thing 
%they can be ? or what worse can be said of any obstruction to good, 
%than that it prevents this ? Doubtless, however, these considerations 
%will not suffice to convince those who most need convincing; and it 
%is necessary further to show, that these developed human beings are 
%of some use to the undeveloped—to point out to those who do not 
%desire liberty, and would not avail themselves of it, that they may 
%be in some intelligible manner rewarded for allowing other people 
%to make use of it without hindrance.
%
%In the first place, then, I would suggest that they might possibly 
%learn something from them. It will not be denied by anybody, that 
%originality is a valuable element in human affairs. There is always 
%need of persons not only to discover new truths, and point out when 
%what were once truths are true no longer, but also to commence new 
%practices, and set the example of more enlightened conduct, and 
%better taste and sense in human life. This cannot well be gainsaid 
%by anybody who does not believe that the world has already attained 
%perfection in all its ways and practices. It is true that this 
%benefit is not capable of being rendered by everybody alike : there 
%are but few persons, in comparison with the whole of mankind, whose 
%experiments, if adopted by others, would be likely to be any improvement 
%on established practice. But these few are the salt of the earth; without 
%them, human life would become a stagnant pool. Not only is it they who 
%introduce good things which did not before exist; it is they who keep 
%the life in those which already existed. If there were nothing new to 
%be done, would human intellect cease to be necessary? 
Não é reduzindo até a uniformidade tudo o que é individual
em si, mas cultivando e impulsionando a individualidade, dentro dos
limites impostos pelos direitos e interesses dos outros, que o ser humano
se torna um nobre e belo objeto de contemplação; e como
participam dos trabalhos o caráter de todos aqueles que o fazem, pelo mesmo
processo a vida humana também se torna rica, diversificada e animada,
fornecendo alimento mais abundante para pensamentos e sentimentos elevados,
e reforçando o vínculo que liga cada indivíduo à raça,
tornando infinitamente mais valioso pertencer a ela. Proporcionalmente ao
desenvolvimento de sua individualidade, cada pessoa se torna mais valiosa
para si mesma, e portanto pode se tornar mais valiosa para os outros.
Existe uma plenitude de vida maior sobre a sua própria existência, e quando
há mais vidas individualmente, cresce a massa
composta por cada uma delas. É necessária uma grande energia para evitar que
indivíduos mais fortes invadam a esfera de direitos dos
outros, e isso é inevitável; mas para isso há uma ampla compensação
mesmo do ponto de vista do desenvolvimento humano. Os meios de desenvolvimento
que o indivíduo perde por ser impedido de satisfazer suas inclinações
com o prejuízo de outros são principalmente obtidos à custa do desenvolvimento
de outras pessoas. E até para o próprio indivíduo há um equivalente completo para o melhor
desenvolvimento da parte social de sua natureza, tornado possível pela
restrição imposta à parte egoísta. Praticar regras rígidas de justiça
para o bem dos outros desenvolve os sentimentos e capacidades que têm
o bem dos outros como seu objetivo. No entanto, impor restrições em coisas que não
afetam terceiros, mas que causam o simples descontentamento destes, não desenvolve nada de valor,
exceto a força de caráter que pode manifestar"-se como resistência à restrição.
Isso não deve ser tolerado, porque entorpece e embota toda a natureza.
Para ser justo com a natureza de cada um, é essencial
que pessoas diferentes possam levar vidas diferentes.
Todo período histórico em que isso foi praticado tornou"-se
notável para a posteridade. Mesmo o despotismo não produz seus piores
efeitos, enquanto a individualidade existe sob ele; e tudo que esmaga
a individualidade é despotismo, seja qual for o nome que se dê, e a despeito de tal despotismo 
afirmar cumprir a vontade de Deus ou as injunções dos homens.
Tendo afirmado que a individualidade é a mesma coisa que o desenvolvimento, e
que é somente o cultivo da individualidade que produz, ou pode
produzir, seres humanos bem desenvolvidos, eu poderia aqui fechar o argumento:
o que mais ou melhor pode ser dito sobre qualquer estado das relações humanas
exceto que aproximam os próprios seres humanos da melhor condição
a que podem chegar? Ou o que de pior pode ser dito sobre qualquer obstrução ao bem,
do que o que impede isso? Sem dúvida, no entanto, estas considerações
não serão suficientes para convencer aqueles que mais precisam ser convencidos, e 
é necessário mostrar, de forma mais clara, que esses seres humanos desenvolvidos
tem alguma utilidade para os não desenvolvidos, por chamar a atenção para aqueles que não
desejam a liberdade, e não recorrem a ela, para que possam
ser, de alguma maneira razoável, recompensados por permitir que outras pessoas
façam uso dela sem impedimentos.

Em primeiro lugar, então, eu sugeriria que estes possivelmente
aprenderiam algo com eles. Ninguém pode negar que
a originalidade é um elemento importante nos assuntos humanos. Há sempre a
necessidade de pessoas, não só para descobrir novas verdades e apontar quando
verdades antigas deixaram de ser verdadeiras, mas também para dar início a novas
práticas, e dar o exemplo de condutas mais esclarecidas, e
mais bom gosto e bom senso na vida humana. Isso só pode ser negado
por alguém que acredita que o mundo já atingiu a
perfeição em todas as suas formas e práticas. É verdade que este
benefício não é compreendido por todos igualmente: há
muito poucas pessoas, em comparação com o conjunto da humanidade, cujas
experiências, se adotadas por outros, provavelmente trariam alguma
melhora para costumes estabelecidos. Mas esses poucos são o sal da terra; sem
eles, a vida humana se tornaria estagnada. Eles não só 
introduzem coisas boas que antes não existiam, mas mantêm
vivas as que já existiam. Se não houvesse nada de novo para
ser feito, o intelecto humano deixaria de ser necessário? Seria essa uma razão 
para que aqueles que fazem as coisas ao modo antigo
devam esquecer por que o fizeram, e que o fizeram como gado e não como
seres humanos? Nas melhores crenças e práticas, é simplesmente muito
grande a tendência de degenerar no mecânico; e a menos que houvesse uma
sucessão de pessoas cuja originalidade sempre renovada impedisse que os
fundamentos de tais crenças e práticas se tornassem meramente
tradicionais, essa matéria morta não resistiria ao menor choque de algo
realmente vivo, e então não haveria nenhuma razão para que a
civilização não pudesse se extinguir, como ocorreu com o Império
Bizantino. É verdade que pessoas de gênio são e aparentemente sempre
serão uma pequena minoria; contudo, para tê"-los é necessário preservar
o solo em que eles crescem. Gênios só podem respirar livremente numa
\textit{atmosfera} de liberdade. Pessoas de gênio são, \textit{ex vi
termini},\footnote{ Por definição. [\versal{N.T.}]} 
mais individualistas do que as outras – menos capazes,
portanto, de se adaptar, sem pressão danosa, a qualquer um dos moldes
fornecidos em número reduzido pela sociedade com a finalidade de poupar
os seus membros do problema de formar o seu próprio caráter. Se, em
virtude da timidez, eles consentem em ser forçados em um desses moldes
e a deixar que permaneça restrita toda a parte deles que não pode
se expandir sob pressão, a sociedade em quase nada seria melhor em virtude
desses gênios. Se eles têm um caráter forte e rompem os seus grilhões,
tornam"-se o alvo da sociedade que não foi bem"-sucedida em reduzi"-los ao
lugar"-comum, e que, com um aviso solene, dirige"-se a eles como
“selvagens”, “excêntricos” e coisas semelhantes; é como se alguém
lamentasse que o rio Niágara não corre suavemente entre as suas
margens assim como um canal holandês.

Insisto enfaticamente na importância do gênio e na necessidade de
permitir seu livre desenvolvimento tanto no pensamento como na
prática, e estou bastante \mbox{convencido} de que ninguém negará essa posição
na teoria, mas também sei que, na realidade, a maioria é totalmente
indiferente em relação a isso. As pessoas consideram que ter gênio é
uma coisa digna de nota se ele possibilita ao homem escrever um poema
emocionante ou pintar um quadro. Contudo, no verdadeiro sentido, o de
originalidade no pensamento e na ação, embora ninguém diga que não seja
uma coisa a ser admirada, quase todos, no íntimo, acreditam passar
muito bem sem ele. Infelizmente, isso é tão natural que não causa
espanto. A originalidade é uma das coisas nas quais mentes não
originais não são capazes de ver utilidade. Elas não são capazes de ver
o que ela pode fazer por eles: como poderia ser possível para elas? Se
não podem ver o que pode fazer por elas, então não seria originalidade.
O primeiro serviço que a originalidade pode prestar para elas é o de
abrir os seus olhos; e tão logo isso fosse realizado completamente,
elas teriam uma chance de serem originais por si mesmas. Entretanto,
lembrando que nada foi feito sem que alguém o tivesse feito pela
primeira vez e que todas as coisas boas que existem são fruto da
originalidade, deixemos que sejam modestas o suficiente para acreditar
que ainda há algo por realizar, e garantamos às mesmas que têm tanto
mais necessidade da originalidade quanto menos forem conscientes de a
quererem.

Na verdade, qualquer que seja a homenagem que se queira prestar a uma
real ou suposta superioridade mental, a tendência geral das coisas no
mundo é fazer da mediocridade o poder dominante sobre a humanidade. Na
Antiguidade, na Idade Média e, em grau menor, na longa transição do
feudalismo até os tempos atuais, o indivíduo era um poder em si mesmo;
se possuía um grande talento ou uma posição social elevada, então esse
poder era considerável. No presente, os indivíduos estão perdidos na
multidão. É quase uma trivialidade, na política, dizer que a opinião
pública agora rege o mundo. O único poder que merece esse nome é o das
massas, e o dos governos na medida em que tornam a si mesmos órgão das
tendências ou instintos das massas. Isso é tão verdadeiro para as
relações morais e sociais da vida privada quanto para as transações
públicas. Aqueles a cujas opiniões se dá o nome de opinião pública não
pertencem sempre a uma mesma classe de público: na América, eles
são a população inteira de brancos; na Inglaterra, sobretudo a classe
média; mas eles são sempre uma massa, quer dizer, mediocridade
coletiva. E aqui há uma grande novidade, pois a massa não toma as suas
opiniões dos dignitários da igreja ou do Estado, nem de líderes
ostensivos ou de livros. O seu pensamento é formado por homens como ela,
que em jornais se dirigem a ela ou falam em nome dela, segundo o calor
do momento. Não estou reclamando disso. Não declaro que algo melhor
seja compatível, como regra geral, com o reduzido estado presente da
mente humana. Contudo, isso não impede que o governo da mediocridade
seja um governo medíocre. Nenhum governo, seja numa democracia ou numa
aristocracia numerosa, seja em seus atos políticos ou em suas opiniões,
nas qualidades e mentalidades que estimulou, elevou"-se ou pôde se
elevar sobre a mediocridade, exceto ali onde o soberano --- Muitos --- se
deixou guiar (o que ele fez em seus melhores tempos) pelos conselhos e
pela influência dos mais habilitados e instruídos --- Um ou Alguns. 
A iniciativa de todas as coisas sábias e nobres vem e deve
vir dos indivíduos; geralmente, pela primeira vez, de um único indivíduo.
A honra e a glória do homem mediano é ser capaz de seguir essa
iniciativa; pois ele pode responder internamente a coisas sábias e
nobres e ser guiado por elas com os olhos abertos. Não estou admitindo
aquele tipo de “culto ao herói”, que aplaude o homem forte de gênio
que, pela força, se apodera do governo do mundo e faz com que ele o
obedeça, a despeito de si próprio. Tudo o que ele pode reivindicar é a liberdade de
indicar o caminho. O poder de compelir os homens a esse caminho não só
é inconsistente com a liberdade e o desenvolvimento de todo o restante,
como também corrompe o próprio homem forte. Todavia, parece que quando
as opiniões de massas compostas tão"-somente de homens medianos se
convertem ou estão em vias de se converter na força dominante, então o
contrapeso e a correção dessa tendência deverá ser a individualidade
cada vez mais pronunciada daqueles que estão situados em uma
reconhecida superioridade do pensamento. É principalmente nessas
circunstâncias que indivíduos excepcionais devem ser encorajados e não
dissuadidos a atuar diferentemente da massa. Em outros tempos não havia
vantagem em procederem desse modo, a não ser que não atuassem apenas
diferentemente, mas melhor. Na nossa época, o mero exemplo da não
conformidade, a mera recusa de dobrar os joelhos diante do costume, já
é por si mesmo um serviço. Justamente porque a tirania da opinião torna
a excentricidade reprovável, é desejável que as pessoas sejam
excêntricas para romper com essa tirania. A excentricidade sempre foi
abundante quando e onde foi abundante a força do caráter; e a
quantidade de excentricidade de uma sociedade foi geralmente
proporcional à quantidade de gênio, vigor mental e coragem moral que
ela contém. Que tão poucos hoje ousem ser excêntricos constitui o
principal perigo da época.

Falei que é importante permitir tanto quanto possível as coisas
não costumeiras, de maneira que possa se mostrar, com o tempo, quais
delas estão aptas para se converterem em costumeiras. Mas a
independência da ação e o menosprezo pelo costume não apenas merecem
encorajamento em virtude da possibilidade de propiciar o surgimento de
melhores modos de ação e costumes mais apropriados para a adoção geral;
tampouco são apenas as pessoas de nítida superioridade mental as únicas
que podem reivindicar com justiça a condução de suas próprias vidas à
sua própria maneira. Não há razão para que todas as existências humanas
sejam erigidas segundo um único padrão ou por um pequeno número de
padrões. Se uma pessoa possui qualquer soma tolerável de senso comum e
experiência, o seu próprio modo de abordar a sua existência é o melhor,
não porque é o melhor em si mesmo, mas porque é o seu próprio modo. Os
seres humanos não são como carneiros, e mesmo os carneiros não são
indistintamente iguais. Um homem não pode pegar um casaco ou um par de
botas qualquer, a não ser que tenham sido feitos sob medida para ele ou
puder escolher em uma loja de roupas completa: porventura é mais fácil
adequá"-lo a uma vida ou a um casaco, ou os seres humanos são mais
iguais entre si em sua conformação física e espiritual como um todo do
que na forma de seus pés? Se as pessoas apenas tivessem diferenças de
gosto, isso já seria razão suficiente para não tentar moldá"-los segundo
um único modelo. Pessoas diferentes requerem diferentes condições para
o seu desenvolvimento espiritual; e elas não podem viver saudavelmente
na mesma moral, assim como toda a variedade de plantas não pode viver
na mesma atmosfera física e no mesmo clima. As mesmas coisas que
auxiliam uma pessoa a cultivar a sua natureza mais elevada são
impedimentos para outra pessoa. O mesmo modo de viver é estímulo sadio
para um, conservando nas melhores condições todas as suas faculdades de
ação e desfrute, enquanto para outro é um fardo penoso, que aniquila ou
oprime toda a sua vida interior. Nos seres humanos são tantas as
diferenças nas fontes do prazer, nas suscetibilidades à dor, nos
efeitos dos diversos agentes físicos e morais, que a não ser que haja
uma diversidade correspondente em seus modos de viver, eles jamais
obterão o quinhão de felicidade e jamais alcançarão o grau de
desenvolvimento mental, moral e estético de que a sua natureza é capaz.
Por que então a tolerância, na medida em que está em jogo o sentimento
público, deve se limitar apenas aos gostos e modos de viver que obtém
aquiescência da multidão de seus partidários? Em nenhuma parte (exceto
em algumas instituições monásticas) a diversidade de gostos é
totalmente ignorada; uma pessoa pode, sem censura, preferir ou não
remar, fumar, música, exercícios atléticos, xadrez, cartas, estudar,
porque aqueles que gostam de cada uma dessas coisas ou aqueles que não
gostam delas são por demais numerosos para serem desencorajados. Mas o
homem, e ainda mais a mulher, que podem ser acusados de fazer “o que
ninguém faz” ou de não fazer “o que todo mundo faz”, são alvo de
comentários muito mais depreciativos do que se ele ou ela tivessem
cometido algum crime moral grave. As pessoas precisam possuir um título
ou algum outro tipo de posição ou de consideração por parte daqueles da
mesma posição, para que sejam capazes de se permitir um pouco ao luxo de
fazer o que gostam sem prejuízo de sua própria estima. Para se permitir
um pouco, repito, porque quem quer que se permitir muito correrá o
risco de sofrer algo pior que palavras de desprezo – corre o risco de
ser tomado como lunático e de ver as suas propriedades tomadas e dadas
a parentes.\footnote{ Há algo de desprezível e repugnante na espécie
de evidência com que, nos últimos anos, qualquer pessoa pode ser
declarada judicialmente incapaz de administrar os seus negócios; e
depois de sua morte a cessão de suas propriedades pode ser suspensa se
houver o suficiente para pagar as despesas do litígio – o que será
cobrado das propriedades elas mesmas. Todos os pequenos detalhes da
vida cotidiana são examinados e, visto por intermédio das mais baixas
faculdades perceptivas e descritivas que pode haver, tudo o que for
encontrado que aparente diferença em relação ao senso comum absoluto é
apresentado ao júri como evidência de insanidade, e muitas vezes com
sucesso; os jurados costumam ser tão vulgares e ignorantes quanto as
testemunhas, quando não mais que elas; enquanto os juízes, com essa
extraordinária falta de conhecimento da natureza e vida humana que
continuamente assombra a nós, legistas ingleses, muitas vezes ajudam a
orientar mal os jurados. Esses julgamentos comportam livros e mais
livros sobre o estado do sentimento e da opinião do vulgo no que diz
respeito à liberdade humana. Longe de atribuírem algum valor à
individualidade – tão longe quanto estão de respeitarem o direito de
cada um a agir em assuntos indiferentes como bem lhe aprouver segundo o
seu próprio entendimento e inclinações –, juízes e jurados jamais podem
conceber que alguém em estado de sanidade possa desejar tal liberdade.
Em dias passados, quando se propôs queimar ateus, pessoas caridosas
sugeriram em lugar disso colocá"-los em hospícios: não seria uma
surpresa se voltássemos a ver isso nos dias atuais, assim como os
propositores aplaudirem a si mesmos, porque adotaram um modo tão humano
e cristão de lidar com esses desafortunados, ao invés de persegui"-los
por motivos religiosos, e não sem a silenciosa satisfação de que eles
obtiveram dessa maneira o que mereciam. [\versal{N.A.}]}

Há uma característica na atual direção da opinião pública que a
predispõe particularmente a ser intolerante com qualquer demonstração
acentuada de individualidade. A maior parte da humanidade não é apenas
moderada no intelecto, mas também em suas inclinações; ela não tem
gostos ou desejos fortes o suficiente para incliná"-la a fazer algo
inusual, e ela portanto não compreende aqueles que os têm,
classificando a todos como selvagens e destemperados e acostumando"-se
a olhá"-los de cima para baixo. Agora, além desse fato de caráter geral,
temos ainda de supor que há um forte movimento em direção ao
aperfeiçoamento moral, e é evidente o que devemos esperar. Nos dias de hoje,
esse movimento teve início; muito tem sido atualmente realizado para
aumentar a regularidade da conduta e desencorajar os excessos; trata"-se
de um espírito filantrópico conhecido, para cujo exercício nenhum campo
é mais convidativo do que o da elevação da moral e da prudência de
nossos semelhantes. Mais do que em períodos anteriores, essas
tendências do tempo tornam o público mais disposto a prescrever regras
gerais de conduta e a esforçar"-se para que todos se conformem
com o padrão aprovado. E esse padrão, seja expresso ou tácito, consiste
em não desejar nada fortemente. O seu ideal de caráter é ser sem
qualquer caráter pronunciado e mutilar por compressão, como se faz com
os pés das chinesas, qualquer parte da natureza humana que seja
proeminente e tenda a tornar a pessoa marcadamente diferente em relação
à humanidade comum.

Como é usual com dietas que excluam uma parte daquilo que é desejável, o
padrão atual de aprovação produz tão"-somente uma imitação inferior da
outra parte. Em vez de grandes energias guiadas por uma razão vigorosa e
sensações fortes controladas por uma vontade consciente, o que resulta
aqui são sensações e energias débeis, que por isso podem conformar"-se
externamente à regra sem qualquer esforço seja da vontade ou da razão.
Os caracteres energéticos já se tornaram em grande escala meramente
tradicionais. Apenas raramente há vazão de energia nesse país, a não
ser nos negócios. A energia dispendida nos negócios ainda pode ser dita
considerável. O pouco que sobra desta ocupação é gasto em algum
\textit{hobby}; ele pode ser útil, como um \textit{hobby} filantrópico,
mas sempre se trata de algo isolado e geralmente de pequenas dimensões.
A grandeza da Inglaterra é agora inteiramente coletiva: individualmente
pequenos, aparentemente somos capazes de algo grande apenas em virtude
de nossos hábitos de associação; e com isso os nossos filantropos
morais ou religiosos estão perfeitamente satisfeitos. Mas foram homens
de outra espécie que fizeram da Inglaterra o que ela tem sido; e homens
de outra espécie serão necessários para prevenir o seu declínio.

O despotismo do costume é por toda parte um obstáculo para o avanço da
humanidade, em virtude do incessante antagonismo em relação àquela
disposição de almejar algo melhor do que o costumeiro, o que se
denomina, de acordo com as circunstâncias, de espírito de liberdade ou
espírito de progresso e melhoria. O espírito de melhoria não é sempre
um espírito de liberdade, pois pode almejar, por meio da força, a
melhoria de um povo sem vontade; e o espírito de liberdade, na medida
em que ele resiste a tais tentativas, pode aliar"-se local e
temporalmente aos oponentes da melhoria; contudo, a única fonte
infalível e permanente de melhoria é a liberdade, pois, com ela, há
tantos centros independentes de melhoria quanto indivíduos. O princípio
progressivo, por qualquer meio e em qualquer forma, como amor da
liberdade ou amor da melhoria, antagoniza com a influência do Costume,
pois implica ao menos a emancipação do seu jugo; e a disputa entre os
dois constitui o principal interesse da história do gênero humano. A
maior parte do mundo não tem, propriamente falando, história, porque o
despotismo do Costume é absoluto. Esse é o caso em todo o Oriente. Ali
o costume é, em todas as coisas, a instância superior; justiça e
direito significam conformidade ao costume; ninguém pensa em resistir
ao argumento do costume, a não ser algum déspota intoxicado pelo poder.
E vemos o resultado. Em algum momento essas nações devem ter possuído
originalidade; elas não saíram da terra populosas, letradas e versadas
em muitas artes da vida; foram elas mesmas que fizeram tudo isso e
foram então as maiores e mais poderosas nações do mundo. E o que são
agora? Subordinados ou dependentes de tribos cujos ancestrais
perambulavam nas florestas, quando os seus próprios ancestrais tinham
palácios magníficos e templos suntuosos, sobre os quais todavia o
costume dividia o exercício das normas com a liberdade e o progresso.
Aparentemente, um povo pode ser progressista por certo período de tempo
e então ele para: mas por que ele para? Quando ele deixa de possuir
individualidade. Se uma transformação semelhante ocorresse nas nações
europeias, não seria de forma exatamente igual: o despotismo do costume
que intimida tais nações não é precisamente estacionário. Ele proíbe a
singularidade, mas isso não torna impossíveis as mudanças, desde que
sejam realizadas em conjunto. Descartamos os costumes fixos de nossos
antepassados; agora cada um deve se vestir igual ao outro, mas a moda
pode mudar uma ou duas vezes por ano. Quando há a mudança, cuidamos
para que seja apenas no interesse da mudança e não em virtude de
qualquer ideia de beleza ou conveniência; pois uma mesma ideia de
beleza ou conveniência não ocorreria a todo o mundo ao mesmo tempo, nem
tampouco seria abandonada por todos em um outro momento. Somos tão
progressistas quanto passíveis de mudança: realizamos continuamente
novas invenções para coisas mecânicas e as conservamos até serem
superadas por melhores; somos ávidos por aperfeiçoamentos na política,
na educação, mesmo na moral, embora nessa última a nossa ideia de
aperfeiçoamento consista principalmente em persuadir ou coagir outras
pessoas a serem melhores que nós mesmos. Não é o progresso que
objetamos; ao contrário, vangloriamo"-nos de ser as pessoas mais
progressistas que jamais viveram. É contra a individualidade que
travamos guerra: pensaríamos ter feito maravilhas se tivéssemos nos
tornado iguais; esquecemos que as desigualdades entre uma pessoa e outra
é geralmente a primeira coisa que fixa a atenção de seu próprio tipo e
na superioridade de outro, ou na possibilidade de, ao combinar a
vantagem de ambos, produzir algo melhor que cada um. Temos um exemplo,
que também serve de advertência, na China – uma nação de muito talento
e, em alguns aspectos, de muita sabedoria, em virtude da rara boa
fortuna de ter sido bem provida num período recente com um conjunto
particularmente bom de costumes, obra até certo ponto de homens
merecedores do título de sábios e filósofos, com o que têm de
concordar, sob certas restrições, até mesmo os mais esclarecidos dentre
os europeus. Ela é notável ainda pela excelência do seu aparato para
imprimir, tanto quanto possível, em cada mente da comunidade a melhor
sabedoria que possui, e por assegurar que aquele que melhor se
apropriou dela assumirá posições de honra e poder. Sem dúvida, o povo
que fez isso descobriu o segredo do progresso humano e deveria ter se
mantido com firmeza à frente do movimento do mundo. Mas, ao contrário,
ele se tornou estacionário – e assim permaneceu por milhares de anos;
e, se em algum momento voltar a avançar, será por causa de
estrangeiros. Ele teve sucesso, além de toda a esperança, naquilo que
os filantropos ingleses trabalharam de modo tão diligente – em tornar
todas as pessoas iguais, todas governando os seus pensamentos e
condutas segundo as mesmas máximas e regras; e esses são os frutos que
produziram. O regime moderno da opinião pública é, de forma
desorganizada, o que os sistemas educacional e político chineses são
organizadamente; e, a menos que a individualidade seja capaz de se
afirmar com êxito diante desse jugo, a Europa, a despeito dos seus
nobres antepassados e do seu cristianismo confesso, tenderá a se tornar
outra China.

O que até aqui preservou a Europa deste destino? O que fez da família das
nações europeias uma porção da humanidade em aperfeiçoamento em vez de
estagnação? Não foi nenhuma excelência superior nelas que, se existente,
existe como efeito, não como causa – mas sua notável diversidade de
caráter e cultura. Indivíduos, classes, nações, são muitíssimo diferentes
entre si: bateram uma grande variedade de trilhas, cada uma levando a algo
valioso e, embora em cada período as que tomaram trilhas diferentes tenham
sido intolerantes entre si e cada uma acharia excelente se todas as outras
fossem compelidas a tomar essa mesma trilha, suas tentativas de impedir o
desenvolvimento umas das outras raramente tiveram qualquer sucesso
permanente e cada uma recebeu o bem
que as outras ofereceram. A meu ver, a Europa muito deve a essa
pluralidade de trilhas para seu desenvolvimento progressivo e
multilateral, mas já começa a possuir esse benefício em grau
consideravelmente menor. Ela está avançando decididamente em direção ao
ideal chinês de tornar todas as pessoas iguais.  Tocqueville,\footnote{ Alexis de Tocqueville (1805---1859), filósofo francês, autor de \textit{Sobre a democracia}. Ver, na Coleção de Bolso Hedra, \textit{Viagem aos Estados Unidos}. [\versal{N.T.}]} em seu
último trabalho importante, observa o quanto os franceses do presente se
parecem uns com os outros, muito mais do que acontecia até mesmo na última
geração. A mesma observação poderia ser feita a respeito dos ingleses, em
grau bem maior. Em um trecho já citado, Wilhelm von Humboldt mostra duas     %nota
coisas como condições necessárias do desenvolvimento humano, porque 
necessárias para diferenciar as pessoas entre si: a liberdade e
uma variedade de situações. Neste país, a segunda dessas duas condições
diminui a cada dia. As circunstâncias que envolvem diferentes classes e
indivíduos e modelam seu caráter a cada dia são mais assimiladas.
Antigamente, diferentes classes sociais, diferentes bairros e regiões,
diferentes negócios e profissões viviam no que se poderia chamar de mundos
diferentes; no presente, em enorme grau, vivem no mesmo mundo. Falando
relativamente, eles agora leem as mesmas coisas, escutam as mesmas coisas,
veem as mesmas coisas, vão para os mesmos lugares, têm suas esperanças e
temores dirigidos aos mesmos objetos, têm os mesmos direitos e liberdades
e os mesmos meios de afirmá"-los. Por maior que sejam as diferenças de
posição que restam, elas nada são em relação às que já acabaram. E a
assimilação continua. Todas as mudanças políticas da era a promovem, desde
que todas tenham a tendência a elevar o baixo e a rebaixar o alto. Cada
extensão da educação a promove, porque a educação deixa as pessoas sob
influências comuns, proporcionando"-lhes acesso ao conjunto geral dos fatos
e sentimentos. As melhorias nos meios de comunicação a promovem, \mbox{trazendo}
habitantes de lugares distantes ao contato pessoal e mantendo um rápido
fluxo de mudanças de residência entre um lugar e outro. O aumento do
comércio e as manufaturas a promovem, difundindo mais amplamente as
vantagens de circunstâncias fáceis e abrindo à competição geral todos os
objetos de ambição, mesmo a mais alta, com o que o desejo de elevar"-se já
não é mais característica de uma determinada classe, mas de todas as
classes. Uma influência mais poderosa do que até mesmo todas essas, para
levar a efeito uma semelhança geral entre a humanidade, é o
estabelecimento completo, neste e em outros países livres, da ascendência
da opinião pública no Estado. Como as diversas eminências sociais que
permitiram que pessoas nelas entrincheiradas 
desconsiderassem a opinião da multidão gradualmente se nivelaram, 
como a própria ideia de se resistir à vontade do público, quando se sabe
inequivocamente que o público tem uma vontade, desaparece cada vez mais
das mentes dos políticos atuantes, deixa então de existir qualquer apoio
social ao não conformismo, qualquer poder substantivo na sociedade que,
em si oposto à ascendência dos números, está interessado em tomar sob sua
proteção opiniões e tendências variando com as do público.

A combinação de todas essas causas forma imensa massa de influências hostis
à individualidade, e não é fácil ver como ela consegue se manter firme.
Conseguirá, com dificuldade cada vez maior, a menos que se possa fazer a
parte inteligente do público sentir seu valor --- ver que é bom que haja
diferenças, ainda que não para melhor, ainda que algumas, como pode
parecer, para pior. Se algum dia as reivindicações de individualidade
tiverem de ser afirmadas, o momento é agora, enquanto boa parte ainda
continua desejando completar a assimilação forçada. Apenas nas primeiras
etapas qualquer posição poderá ter sucesso contra a usurpação. A exigência
de que todas as outras pessoas tenham de parecer conosco aumenta pelo que
se alimenta. Se a resistência espera até que a vida esteja reduzida \textit{quase}
a um tipo uniforme, todos os desvios desse tipo virão a ser considerados
ímpios, imorais e até monstruosos e contrários à natureza. A humanidade
rapidamente se torna incapaz de conceber a diversidade, quando por algum
tempo se desacostuma a \mbox{vê"-la}.

\chapter[Dos limites da autoridade]{Dos limites da autoridade da sociedade sobre o indivíduo}

\textsc{Qual seria} então o justo limite para a soberania do indivíduo sobre si
mesmo? Onde começa a autoridade da sociedade? Quanto da vida humana
deve caber à individualidade, e quanto à sociedade?

Cada uma delas receberá o seu quinhão merecido, se cada uma tiver a parte
que mais lhe concernir. À individualidade deve pertencer a parte
da vida em que o indivíduo está interessado, e à sociedade,
a parte que interessa principalmente a sociedade.

Apesar de a sociedade não estar fundada num contrato, e apesar de nenhum
bom propósito ser alcançado ao se inventar um contrato para se deduzir
a partir dele as obrigações sociais, cada um que receba a proteção da
sociedade deve algo em troca desse benefício; o fato de se viver
em sociedade torna indispensável que cada um deva ser obrigado a
observar certa linha de conduta para com os outros. Esta conduta
consiste primeiro em não ferir o interesse de outra pessoa, ou melhor,
certos interesses que, seja por expressa provisão legal ou por
entendimento tácito, devam ser considerados como direitos e, em
segundo, consiste em que cada pessoa deve suportar a sua cota (a ser
definida em algum princípio de igualdade) de trabalhos e sacrifícios
incorridos ao se defender a sociedade ou seus membros de danos e
inconvenientes. É justo que a sociedade imponha essas condições a todos
aqueles que tentem fugir dessas obrigações. E isso não é tudo o que a
sociedade pode fazer. Os atos de um indivíduo podem ser danosos para os
outros, ou demonstrarem falta de consideração pelo bem"-estar deles, sem
que chegue ao ponto de violarem qualquer direito constituído. O ofensor
pode ser então justamente punido pela opinião, mas não pela lei. Tão
logo alguma parte da conduta de uma pessoa afeta de forma prejudicial o
interesse das outras, a sociedade tem jurisdição sobre ela, e saber
se o bem"-estar geral será ou não beneficiado se houver uma intervenção
naquela conduta torna"-se uma questão aberta à discussão. Mas não há espaço para
tratar dessa questão quando a conduta de uma pessoa não afeta o
interesse de ninguém além dela mesma, ou não precisa afetar,
exceto se elas concordem (todas as pessoas sendo, no caso, maiores de
idade e dentro do limite normal da compreensão). Em tais casos deve
haver liberdade perfeita, legal e social, para realizar a ação e se
responsabilizar pelas consequências.

Seria entender muito mal esta doutrina supor que ela sustenta a
indiferença egoísta, que assume que os seres humanos não têm nenhuma
relação com a conduta das outras pessoas, e que eles não devam se
preocupar com as boas ações ou o bem"-estar uns dos outros, a menos que
seu próprio interesse esteja envolvido. Ao invés de qualquer
diminuição, há necessidade de um grande aumento na ação desinteressada
em promover o bem dos outros. Mas a benevolência desinteressada pode
encontrar outros instrumentos, além do chicote e das pancadas, sejam
esses literais ou metafóricos, para persuadir as pessoas pelo seu
próprio bem. Eu seria a última pessoa a fazer pouco caso das
virtudes que se relacionam conosco mesmos --- elas estão em segundo lugar
em importância, se é que estão apenas em segundo, em relação às
virtudes sociais. A educação deve cultivar ambas. Mas mesmo a educação
funciona tanto por convicção e persuasão quanto por coação, e é
apenas pelas primeiras que, passado o período da educação, as virtudes
relacionadas à própria pessoa devem ser inculcadas. Os seres humanos
devem uns aos outros uma ajuda, para distinguir o melhor do pior, e
encorajamento para escolher o primeiro e para evitar o segundo. Eles
devem sempre estar estimulando uns aos outros a um aumento do uso das
suas capacidades mais elevadas, e tentando levar seus sentimentos e
intenções para objetivos e aspirações sábias e elevadas, ao invés de
tolas e degradantes. Mas nem uma só pessoa, ou grupo de pessoas, pode
dizer a outro ser humano, já maduro, que ele não pode, para seu próprio
benefício, fazer de sua vida o que ele escolheu. Ele é
a pessoa mais interessada em seu próprio bem"-estar, o interesse que
qualquer outra pessoa, exceto em casos de fortíssima ligação emocional,
possa ter nesse bem"-estar é diminuto perto do que ele próprio tem; o
interesse que a sociedade tem para com ele, como indivíduo (excetuando
a sua conduta para com os outros), é parcial, e completamente indireto,
enquanto que, em relação aos seus próprios sentimentos e
circunstâncias, o mais comum dos homens ou das mulheres tem meios de
conhecê"-los que em muito superam os que qualquer outra pessoa possa
ter. A interferência da sociedade que anule seu julgamento e propósitos
no que se refere a sua própria vida, deve ser baseada em presunções
gerais, que podem ser totalmente errôneas e que, mesmo se corretas, no
mais das vezes são inaplicáveis a casos individuais por pessoas que não
tem maior conhecimento das circunstâncias do caso do que aqueles que o
observam de fora. Neste departamento, portanto, dos assuntos humanos, a
Individualidade tem o seu campo de ação próprio. Na conduta dos seres
humanos uns com os outros, é necessário que as regras gerais devam ser
observadas na maioria das vezes, para que as pessoas possam saber o que
esperar, mas em relação ao que importa a cada pessoa, à sua
individualidade espontânea deve"-se permitir um exercício livre.
Considerações que ajudem ao seu julgamento, exortações para fortalecer
sua vontade podem ser"-lhe oferecidas, e mesmo jogadas sobre ela, pelas
outras pessoas, mas ela própria é a instância final de julgamento.
Todos os erros que ela provavelmente possa cometer mesmo contra os
conselhos e avisos são de longe superados pelo mal de se permitir que
outros a forçassem àquilo que consideram ser o melhor para ela. 

Não quero com isso dizer que o modo como uma pessoa é
considerada pelos outros não deva de maneira nenhuma ser afetado pelas suas
qualidades ou deficiências individuais. Isso não é nem possível nem
desejável. Se ela é eminente em quaisquer das qualidades que conduzem
para o seu próprio bem, ela é, até aqui, um adequado objeto de admiração.
Ela está bem mais perto da perfeição ideal da natureza humana. Se
ela é muito deficiente nessas qualidades, um sentimento oposto ao de
admiração surgirá. Há um grau de loucura, e um grau do que pode ser
chamado (apesar da expressão poder ser objetada) de baixeza ou
depravação de gosto, que apesar de não poder, de forma justa, causar
danos a quem o demonstra ter, o torna, necessária e propriamente, um
objeto de desgosto ou, em casos extremos, mesmo de desprezo: uma pessoa
não pode ter essas más qualidades sem que tal ocorra. Apesar de não
fazer mal para ninguém, uma pessoa pode agir de tal modo que nos
compele a julgá"-la e percebê"-la como um tolo, ou como
um ser de uma ordem inferior, e posto que esses julgamentos são algo
que ela preferiria evitar, é prestar"-lhe um serviço avisá"-la disso
de antemão, assim como de qualquer outra consequência desagradável à
qual ela possa se expor. Seria bom, na verdade, que essa ação meritória
pudesse ser mais frequentemente realizada do que as noções comuns de
polidez permitem atualmente, e que uma pessoa pudesse apontar para
outra no que ela pensa o que está em falta, sem que tal fosse
considerado mau comportamento ou presunção. Também temos o direito de agir 
de vários modos segundo nossa opinião desfavorável
sobre alguma pessoa, não para oprimir a sua individualidade, mas sim
para exercer a nossa. Não somos obrigados, por exemplo, a procurar a
sua companhia, temos o direito de evitá"-la (mas não o de proclamar
ostentosamente isso), pois temos o direito de procurar a companhia que
nos seja mais aceitável. Nós temos o direito, e pode ser até nosso
dever, de alertar as outras pessoas sobre ela, se pensarmos que seu
exemplo ou conversa provavelmente terá um efeito ruim sobre aquelas
pessoas que permanecem em sua companhia. Podemos dar preferência a
outros sobre ela, quando se tratar de boas ações opcionais, exceto
naquelas que tendam a ajudar a melhorar. Essas são as diferentes
maneiras pelas quais uma pessoa pode sofrer penalidades muito severas
nas mãos de outras, por falhas que só concernem diretamente a ela
própria, mas ela sofre essas penalidades tanto quanto elas sejam
naturais e, assim como é, as consequências espontâneas das próprias
falhas, e não porque as penalidades lhe são infringidas para fins de
punição. Uma pessoa que mostre ser exagerada, obstinada, orgulhosa,
que não possa viver com meios moderados, que não pode se negar
indulgências daninhas, que persegue prazeres animalescos em troca
daqueles dos sentimentos e intelecto, deve esperar ser diminuída na
opinião dos outros, e obter delas uma porção menor de seus sentimentos
favoráveis, mas disso ela não tem direito nenhum de reclamar, a menos
ela tenha conseguido o reconhecimento das pessoas devido a sua
excelência ímpar nas suas relações sociais, e portanto obtido delas um título
para sua boa vontade, que não é afetado pelo seus deméritos para
consigo mesmo. 

O que defendo é que as inconveniências que são estreitamente
inseparáveis do julgamento desfavorável dos outros, são as únicas às
quais uma pessoa deve ser sujeita, em relação àquela parte de sua
conduta e caráter que concerne ao seu próprio bem, mas que não afeta os
interesses dos outros nas suas relações com ela. Atos danosos às outras
pessoas requerem um tratamento totalmente diferente. Limitações aos
seus direitos, obrigação de pagar dano ou perda não justificadas,
falsificações ou duplicidades quando se lida com as pessoas, utilização
de vantagens injustas ou egoístas sobre as pessoas, e mesmo a recusa
egoísta de defender os outros contra danos --- esses são os objetos
adequados da reprovação moral e, nos casos mais graves, de retratação
moral e punição. E não somente esses atos, mas também a disposição que
leva a eles, são objetos adequados de desaprovação, que podem levar ao
desprezo. Crueldade de disposição, malícia e natureza má, aquela mais
antissocial das paixões, a inveja, dissimulação e insinceridade,
irascibilidade por razões insuficientes, ressentimento desproporcional
com a provocação, o amor de mandar nos outros, o desejo de obter mais
do que a sua cota de vantagens (a palavra pleoneksia [\grk{πλεονεξία}] dos gregos), o
orgulho que obtém sua gratificação ao diminuir os outros, o egotismo
que faz alguém imaginar que ele mesmo e seus interesses são mais importantes do que
tudo o mais, e decide todas as questões duvidosas a seu favor --- esses
são vícios morais, e formam um caráter mau e odioso, diferentes das
previamente mencionadas falhas egoístas, que não são propriamente
imoralidades e que, não importando até que baixo nível leve a pessoa,
não constituem maldade. Podem ser provas de tolice ou da inexistência
de uma dignidade pessoal e respeito próprio, mas só são objetos de
reprovação moral quando envolvem uma quebra do dever que se tem com os
outros, pelo qual o indivíduo é obrigado a tomar conta de si mesmo. Os
que chamamos de deveres para conosco não são socialmente obrigatórios,
a menos que as circunstâncias os tornem deveres para com os outros. O
termo dever para si próprio, quando significa algo além da prudência,
significa respeito próprio ou autodesenvolvimento, e nenhum deles
diz respeito às outras pessoas, porque não seria bom para a humanidade se assim fosse.

A distinção entre a perda de consideração que uma pessoa pode sofrer
de forma justa por falta de prudência ou de dignidade pessoal, e a
reprovação que lhe é devida por ofensas aos direitos dos outros não é
uma distinção meramente nominal. Há uma enorme diferença, tanto em
nossos sentimentos quanto em nossa conduta, se ela nos desagrada em
coisas sobre as quais pensamos que temos direito de controlá"-la, ou
se nos desagrada em coisas que sabemos que nós não temos. Se ela nos
desagrada, podemos expressar nosso desgosto, e podemos nos manter
afastados tanto de uma pessoa quanto de coisas que nos desagradem, mas
não nos sentimos compelidos a tornar a vida dessa pessoa desconfortável.
Podemos refletir que ela já carrega, ou irá carregar, todo o peso por
seus erros; se ela arruína sua vida por mau gerenciamento, nós não
desejaremos, por essa razão, arruiná"-la ainda mais: ao invés de
querer puni"-la, nós, nos esforçaremos para aliviar sua
punição, mostrando a ela como pode evitar ou curar os males que sua conduta lhe
traz. Ela pode ser para nós objeto de piedade, talvez de desapreço,
mas não de raiva e ressentimento, não devemos tratá"-la como inimiga
da sociedade, o pior que podemos pensar que é justo fazermos seria
deixá"-la por sua própria conta. Será completamente diferente se ela
infringiu as regras necessárias para a proteção das outras pessoas,
individual ou coletivamente. As consequências ruins de seus atos não
caem sobre ela mesma, mas sobre os outros, e a sociedade, como
protetora de todos os seus membros, deve retaliá"-la, deve
infligir dores a ela pelo explícito propósito de punição, e deve cuidar
que essa punição seja severa o suficiente. Neste caso, ela é um
delinquente no nosso tribunal, e nós somos chamados não só para 
julgá"-la, mas, de uma forma ou outra, para executar nossa
sentença; no outro caso, não cabe a nós infligir nenhum sofrimento a
ela, exceto o que possa vir, por acidente, do nosso uso da mesma
liberdade que temos para dirigir nossos próprios assuntos que
concedemos a ela nos seus. 

A distinção aqui apontada, entre a parte da vida de uma pessoa que
concerne apenas a ela e a parte que concerne aos outros, poderá ser
recusada por muitas pessoas. Como (pode"-se perguntar) pode qualquer
parte da vida de um membro da sociedade ser um assunto indiferente para
os outros membros da sociedade? Nenhuma pessoa é um ser completamente
isolado, e é impossível para uma pessoa fazer algo sério ou permanente
contra si mesma, sem que o malfeito não atinja pelo menos os que lhe
estão mais próximos, e às vezes indo bem além deles. Se ela prejudica a
sua propriedade, traz danos àqueles que direta ou indiretamente
dela dependem para viver e, em geral, diminui em grau maior ou menor os
recursos gerais da comunidade. Se ela faz deteriorar as suas faculdades
mentais ou físicas, não somente prejudica aqueles cuja felicidade, em
alguma medida, dependem dela, mas também desqualifica a si
mesma quando se trata de prestar os serviços que ela deve, de maneira
geral, às outras pessoas, tornando"-se talvez um peso para a afeição e
benevolência delas e, se tal conduta for muito frequente, dificilmente
qualquer outra ofensa cometida poderia extrair mais da soma geral dos
bens. Finalmente, se pelos seus vícios ou loucuras uma pessoa não causa
danos diretos aos outros, ela no entanto (pode"-se afirmar)
pode ser daninha pelo seu exemplo, e deve ser
compelida a se controlar, pelo bem daqueles que, por verem ou saberem
de sua conduta, poderiam vir a ser corrompidos ou enganados. 

E mesmo se (seria adicionado) as consequências da má conduta
pudessem ser confinadas ao indivíduo vicioso ou desavisado, deveria a
sociedade abandonar aos seus próprios recursos aqueles que claramente
são incapazes de se guiar? Se a proteção contra si mesma é devida às
crianças e pessoas menores de idade, não seria a sociedade também
obrigada a dá"-la a pessoas mais velhas que são igualmente incapazes
de se autogovernar? Se o jogo, ou o alcoolismo, ou a incontinência,
ou a vagabundice, são danosas para a felicidade, e são um obstáculo
para a melhoria da vida tanto ou ainda mais que os atos proibidos pela
lei, por que (pode"-se argumentar) não deveria a lei, tanto quanto seria
consistente com a praticidade e a conveniência social, se esforçar para
reprimir aquelas coisas também? E como um suplemento para as
imperfeições inevitáveis da lei, não deveria a opinião ao menos
organizar um poderoso movimento contra aqueles vícios, e visitar
duramente com penalidades sociais aqueles que são conhecidos por
praticá"-los? Aqui não se trata (pode ser dito) de restringir a
individualidade, ou de impedir as tentativas de novas e originais
experiências de vida. A única coisa que se pretende prevenir que
aconteça são coisas que já foram julgadas e condenadas desde o início
dos tempos até agora, coisas que a experiência mostra que não são nem
úteis nem adequadas à individualidade de qualquer pessoa. Deve haver
certo período de tempo e certa quantidade de experiências, depois dos
quais uma verdade moral e prudencial possa ser vista como estabelecida:
e o que se deseja é meramente se prevenir que geração após geração caia
no mesmo precipício que foi fatal para as suas predecessoras.

Admito sem reservas que os malfeitos que uma pessoa comete contra si
mesma pode afetar seriamente, através de suas simpatias e seus
interesses, àqueles que lhe são próximos e, em grau menor, à sociedade
como um todo. Quando, devido a condutas deste tipo, uma pessoa é levada
a violar uma obrigação distinta e reconhecível que tem com outra pessoa
ou pessoas, o caso já não mais pertence àqueles que interessam apenas a
ela mesma, e torna"-se portanto passível de uma desaprovação moral no
sentido próprio do termo. Se, por exemplo, um homem, pela intemperança
ou extravagância, torna"-se incapaz de saldar suas dívidas, tendo se
responsabilizado pela educação moral de uma família se torna, pelos
mesmos motivos, incapaz de sustentá"-la ou educá"-la, ele é
merecidamente criticado, e pode ser justamente punido, mas pela falha
no dever que tinha para com seus familiares e com seus credores, e não
pela sua extravagância. Se os recursos que a esses cabiam tivessem sido
desviados para o mais prudente investimento, a culpa moral seria a
mesma. George Barnwell matou seu tio para conseguir dinheiro para sua
amante, mas se ele tivesse cometido o crime para montar um negócio
próprio teria sido igualmente
enforcado.\footnote{ Trata"-se de um crime ocorrido na Inglaterra 
no início do século \textsc{xviii}. [\versal{N.T.}]}
 Do mesmo modo, no caso habitual de um homem que leva tristeza à sua
família devido aos seus maus hábitos, esse homem merece críticas por
sua falta de atenção ou ingratidão, mas também o mereceria se não
estivesse cultivando hábitos em si ruins, mas que fossem dolorosos para
aqueles com quem ele passa a vida, ou para aqueles que dele dependem
para o seu conforto. Quem quer que falhe na consideração geralmente
devida aos interesses e sentimentos dos outros, e que não seja
compelido por algum dever imperativo ou justificado por uma
autopreferência reconhecível, é objeto de uma reprovação moral por
essa falha, mas não pela causa dela, nem pelos erros, que são meramente
pessoais, que podem ter levado a ela. Da mesma maneira, quando uma
pessoa, por uma conduta completamente egoísta, se torna incapaz de
desempenhar um dever público que lhe foi incumbido, ela é culpada de uma
ofensa social. Ninguém deve ser punido simplesmente por estar bêbado,
mas um soldado ou policial deve ser punido por beber em serviço. Em
resumo, onde quer que haja um dano específico, ou um claro risco de
dano específico, seja para o indivíduo seja para o público, o caso é
tirado da província da liberdade e colocado naquela da moralidade ou da lei. 

Mas tendo em vista a meramente contingente, ou, como pode ser chamada,
injúria construtiva que uma pessoa causa à sociedade, por uma conduta
que nem viola nenhum dever específico para com o público, nem, em
nenhuma ocasião, fere alguma outra pessoa além de si mesma, a
inconveniência é daquelas que a sociedade pode suportar, pelo bem maior
da liberdade humana. Se pessoas já crescidas devessem ser punidas por
não tomar conta de si mesmas, eu preferiria antes que isso se desse
pelo bem delas, e não sob a pretensão de prevenir que elas prejudiquem
a capacidade que deveriam ter de oferecer à sociedade benefícios que a
sociedade não pode pretender de forma justa que tenha o direito de exigir.
Mas não posso consentir em debater este ponto como se a sociedade não
tivesse outros meios de levar os seus membros mais fracos ao nível de
um padrão comum de conduta, além de esperar que eles façam algo
irracional, para então puni"-los, legal e moralmente. A sociedade tem
um poder absoluto sobre eles durante toda a primeira parte de sua
existência: ela tem todo o período da infância e da juventude para
tentar conseguir que eles se tornem capazes de uma conduta racional
durante a vida. A atual geração é senhora tanto do treinamento quanto
das circunstâncias totais da próxima geração, ela não pode fazer com
que esta seja perfeitamente boa e sábia, pois ela mesma é deficiente
em bondade e sabedoria, e seus melhores esforços não são, em casos
individuais, os mais bem sucedidos, mas ela é perfeitamente capaz de
fazer a geração seguinte, como um todo, tão boa quanto, e um pouco
melhor, do que ela própria. Se a sociedade permite a um número
considerável de seus membros crescerem como se fossem crianças,
incapazes de agirem segundo uma consideração racional de motivos
distantes, a sociedade só pode culpar a si própria pelas consequências.
Armada não apenas com os poderes da educação, mas com a ascendência que
a autoridade das opiniões recebidas mantém sobre as mentes que são as
mais incapazes de julgar por si mesmas, e ajudada pelas penalidades
\textit{naturais} que não podem deixar de recair sobre aqueles que
incorrem na desaprovação ou desprezo das pessoas que os conhecem, que
não se deixe então a sociedade pretender que necessita, acima de tudo,
do poder de emitir comandos e de forçar a obediência em assuntos que são
do interesse pessoal dos indivíduos, os quais, em todos os princípios
de justiça e política, a decisão deve permanecer com aqueles que
podem aguentar as consequências. Não há nada que tenda a desacreditar e
a frustrar mais os melhores meios de influenciar as condutas do que o
recurso ao pior. Se entre aqueles nos quais está se tentando impor a
prudência ou a temperança houver alguma quantidade da matéria dos quais os
caracteres independentes são feitos, eles infalivelmente se rebelarão
contra essa tentativa. Nenhuma pessoa desse tipo jamais sentirá que o
outro tem o direito de controlá"-la no que toca só a ela, tal como
eles têm o direito de prevenir que ela lhes cause dano nos seus
interesses particulares, e facilmente chega a ser considerada uma
amostra de espírito e de coragem se opor diretamente a essa autoridade
usurpada, e ostensivamente fazer o contrário do que ela ordena, como
mostra a moda da grosseria que, nos tempos do rei Carlos \textsc{ii}, sucedeu à
intolerância fanática dos Puritanos. A respeito do que é dito sobre a
necessidade de proteger a sociedade do mau exemplo dado pelos viciosos
ou autoindulgentes às outras pessoas, é verdade que o mau exemplo
pode ter um efeito pernicioso, especialmente o exemplo de se fazer mal
aos outros sem que o malfeitor seja punido. Mas estamos aqui falando sobre um
tipo de conduta que, embora não faça mal aos outros, se supõe que faça um
grande dano ao próprio agente, e não vejo como alguém que acredita
nisso pode não pensar que o exemplo, ao cabo, deva
ser mais salutar que doloroso, já que, se ele mostra a má conduta,
mostra também as dolorosas ou degradantes consequências que, se tal
conduta é justamente censurada, devem supostamente seguir"-se à maior
parte dos casos semelhantes. 

Mas o mais poderoso de todos os argumentos contra a interferência do
público em questões de conduta puramente pessoal é o de que, quando ela
ocorre, as probabilidades sejam de que interfira erradamente, e no
lugar errado. Sobre questões de moralidade social, de dever para com os
outros, a opinião do público, isto é, da maioria, apesar de estar
frequentemente errada, o mais das vezes provavelmente estará correta,
pois nestas questões do público se requererá que julgue seus próprios
interesses, isto é, de modo que nenhuma conduta, se permitida,
iria \mbox{afetá"-la}. Mas a opinião de semelhante maioria, imposta como lei
sobre a minoria, sobre questões de conduta particular, pode estar tanto
certa quanto errada. Porque nestes casos a opinião pública significa,
no melhor dos casos, a opinião de algumas pessoas sobre o que é bom ou
mau para outras pessoas, e na verdade na maioria dos casos nem sequer chega
a tanto: o público, na mais perfeita indiferença, sobrepujando o prazer
ou a conveniência daqueles a quem censura, e atentando apenas para seus
próprios interesses. Há muitos que consideram uma injúria para si
qualquer conduta que lhes possa ser desagradável, e que se ressentem
dela como um ultraje aos seus sentimentos, tal como aquele fanático
religioso que, quando acusado de desrespeitar os sentimentos religiosos
dos outros, respondeu que eles é que desrespeitavam os seus
sentimentos, ao persistirem nos seus abomináveis rituais e crenças. Mas
não há paridade entre o sentimento de uma pessoa por sua própria
opinião e o sentimento de outra pessoa que se sente ofendida por aquela
pessoa ter essa opinião, não mais do que o desejo de um ladrão em levar
a carteira de uma pessoa e o desejo do legítimo dono dela em
conservá"-la. E o gosto de uma pessoa é tanto de seu interesse
particular quanto a sua opinião ou sua carteira. Seria fácil para
alguém imaginar um público ideal, que deixa em paz a liberdade e a
escolha dos indivíduos em todos os assuntos incertos, e que somente
requer deles que se abstenham de modos de conduta que a
experiência universal condenou. Mas onde se viu um público que colocou
esses limites para a sua censura? Ou desde quando o público se importa
com a experiência universal? Nas suas interferências na conduta de uma
pessoa, o público raramente pensa em outra coisa que a enormidade dos
atos e sentimentos com que discorda, e esse padrão de julgamento, mal
ocultado, é o que é oferecido à humanidade, por nove  em cada dez
moralistas e escritores especulativos, como prescrições da
religião e da filosofia. Esses autores ensinam que coisas são certas
porque são certas, porque nós sentimos que assim é. Dizem"-nos que
procuremos em nossas mentes e corações pelas leis da conduta que nos
sujeitam. O que pode o pobre público fazer senão
aplicar essas instruções, e tornar os seus sentimentos pessoais do bem
e do mal, se esses são toleravelmente unânimes nele, obrigatórios para
todo o resto do mundo? 

O mal aqui apontado não existe apenas em teoria, e deve ser
talvez esperado que eu especifique as instâncias nas quais o público
dessa nossa época e país de forma imprópria dignifica suas preferências
com o caráter de um sentimento moral. Não estou escrevendo um ensaio a
respeito das aberrações do atual sentimento moral. Este é um assunto
pesado demais para ser discutido parenteticamente, e por meio de
exemplos. Ainda assim, exemplos são necessários, para mostrar que o
princípio que sustento é sério e tem importância prática, e que não
estou lutando para levantar uma barreira contra males imaginários. E
não há dificuldade nenhuma em mostrar, através de vários exemplos, que
entender os limites do que pode ser chamado de policiamento moral, até
que este cerceie a mais inquestionavelmente legítima liberdade do
indivíduo, é uma das mais universais de todas as propensões humanas. 

Como primeiro exemplo, que se considerem as antipatias que os homens
cultivam, sem bom fundamento, com as pessoas cujas opiniões religiosas
são diferentes das deles, que não praticam as observâncias de sua religião,
especialmente as abstinências de cunho religioso. Para mencionar um
exemplo até trivial, nada na prática e nas crenças dos cristãos aumenta
ainda mais o ódio dos maometanos contra eles do que o fato de os cristãos
comerem carne de porco. Há poucos atos que os cristãos e os europeus
contemplem com maior desgosto, do que o desgosto com que os maometanos
contemplam esse modo específico de se saciar a fome. Em primeiro lugar,
é uma ofensa contra a sua religião, mas essa circunstância não explica
nem o grau nem a qualidade de sua repugnância, pois o vinho é também
proibido por sua religião, e consumi"-lo é visto por todos os
muçulmanos como errado, mas não como asqueroso. A aversão deles pela
carne de um ``animal impuro'' é, pelo contrário, de um caráter peculiar,
assemelhando"-se a uma antipatia instintiva, que a ideia de impureza,
uma vez completamente inserida nos sentimentos, parece sempre excitar,
mesmo naqueles cujos hábitos pessoais estejam longe de serem
escrupulosamente limpos, e nos quais o sentimento de impureza religiosa, tão forte
entre os hindus, é um exemplo marcante. Suponha"-se agora que exista um
povo, do qual a maioria é formada por muçulmanos, e que essa maioria
insista em não permitir que carne de porco seja consumida dentro das
fronteiras deste país. Essa não seria nenhuma novidade em um país
maometano.\footnote{ O caso dos parses em Bombaim é um curioso exemplo 
deste ponto. Quando essa tribo trabalhadora e empreendedora, os descendentes dos persas
adoradores do fogo, fugindo de seu país natal com o avanço dos califas,
chegaram até a Índia ocidental, eles foram admitidos indulgentemente
pelo soberano indiano, sob a condição de não comerem carne de vaca.
Quando mais tarde aquelas regiões caíram sob o domínio de
conquistadores maometanos, os parses obtiveram a continuação da
indulgência, sob a condição de se absterem da carne de porco. O que
primeiro era obediência à autoridade tornou"-se uma segunda natureza,
e os parses, até o dia de hoje, se abstém tanto da carne de vaca quanto
da de porco. Apesar de não ser requerida por sua religião, a dupla
abstinência com o tempo tornou"-se um costume da sua tribo, e o
costume, no Oriente, é uma religião. [\versal{N.A.}]}
 Legitimaria isso o exercício da autoridade moral da opinião pública? E,
se não, por que não? A prática é de fato revoltante para tal público. E
eles sinceramente pensam que ela é proibida e abominada por sua
divindade. Nem sequer pode essa proibição ser censurada como uma
perseguição religiosa. Ela pode ter sido religiosa em sua origem, mas
não poderia ser uma perseguição por motivos religiosos, já que nenhuma
religião faz do consumo da carne de porco um dever. O único
terreno sensível de condenação seria o de que não é assunto do público
interferir com os gostos pessoais e interesses particulares dos indivíduos. 

 Trazendo o tema mais para perto de nós: a maioria dos espanhóis
considera uma grande impiedade, ofensiva no mais alto grau ao Ser
Supremo, que se preste culto a ele de outra maneira que não a católica
romana, e nenhuma outra forma de culto é legal em solo espanhol. Os
povos de todo o sul da Europa olham para o casamento dos clérigos não
somente como antirreligioso, mas também como algo voluptuoso,
indecente, baixo e revoltante. O que os protestantes pensam sobre esses
sentimentos perfeitamente sinceros, e sobre as tentativas de impô"-los
sobre os não"-católicos? No entanto, se a humanidade tem
justificativas para interferir com a liberdade de pessoas em coisas que
não são do interesse das outras pessoas, sob qual princípio é possível se
excluir, com consistência, os casos mostrados acima? Ou quem pode
culpar pessoas por desejar suprimir aquilo que eles percebem como
um escândalo aos olhos de Deus e dos homens? Um caso mais forte não
pode ser apresentado, no que se refere a proibir aquelas práticas que,
na visão de alguns, são impiedosas, e a menos que estejamos dispostos a
adotar a lógica dos inquisidores, e dizer que devemos perseguir os
outros porque estamos certos, e que eles não devem nos perseguir porque
estão errados, devemos tomar cuidado em admitir um princípio que
veríamos como uma grande injustiça se fosse aplicado contra nós.

Os exemplos anteriores podem ser refutados, apesar de irracionalmente,
como tendo sido retirados de contingências impossíveis de acontecer
entre nós: neste país a opinião não sendo, provavelmente, de molde a
forçar a abstinência de carnes, ou interferir no culto das pessoas, ou
em casar ou não casar, de acordo com sua crença e vontade. No entanto,
o próximo exemplo deve ser tomado como uma interferência com a
liberdade que de maneira nenhuma nós superamos de vez. Onde quer que os
puritanos fossem suficientemente fortes, como na Nova Inglaterra e na
Grã Bretanha nos tempos de Cromwell, eles se esforçaram, com
considerável êxito, em abolir todos os divertimentos públicos, e quase
que todos os privados: música, dança, esportes, ou outras reuniões com
o propósito de diversão, e o teatro. Ainda existe neste país um largo
contingente de pessoas cuja noção de moralidade e religião condena
essas recreações, e desde que essas pessoas pertencem à classe média
ascendente na presente condição política e social do reino, não é nada
impossível que pessoas que compartilham desses sentimentos cedo ou
tarde comandem a maioria no Parlamento. Será que a porção restante da
comunidade apreciará ter as diversões que lhes são permitidas
regulamentadas pelos sentimentos religiosos e morais dos mais fechados
calvinistas e metodistas? Não irá ela, com considerável energia,
desejar que esses pios e mandões membros da sociedade cuidem de seus
próprios negócios? E é isso exatamente que deve ser dito a todo governo
e todo público que pretenda que toda e qualquer pessoa não possa
usufruir dos prazeres que eles pensam ser errados. Mas se o princípio
da pretensão for admitido, ninguém pode razoavelmente objetar que ele
esteja sendo usado no sentido que a maioria, ou outro poder
preponderante no país, queira, e todas as pessoas devem estar dispostas
a se conformar com a ideia de uma comunidade cristã, tal como entendida
pelos primeiros colonizadores da Nova Inglaterra, se uma religião
similar à deles um dia conseguir recuperar o terreno que perdeu, como as
religiões supostamente em declínio são conhecidas por conseguirem.

Pode"-se imaginar outra situação, talvez mais provável de acontecer do
que a última mencionada. Há, confessadamente, uma forte tendência no
mundo moderno na direção de uma constituição democrática da sociedade,
acompanhada ou não por instituições políticas populares. Diz"-se que
no país onde essa tendência é mais clara, onde a sociedade e o governo
são os mais democráticos --- os Estados Unidos ---, o sentimento da maioria, para quem qualquer
amostra de um estilo de vida mais caro ou suntuoso do que o que ela
pode aspirar a ter é desagradável, pôs em operação uma lei suntuária
razoavelmente efetiva, e que em muitas partes da União torna de fato
difícil para uma pessoa que tenha uma renda muito alta achar um modo de
gastar o seu dinheiro sem que incorra na desaprovação popular. Apesar
de declarações como essas serem sem dúvida um exagero diante da
situação real, o estado de coisas assim descrito não é apenas
concebível e possível, mas também um resultado provável do sentimento
democrático, combinado com a noção de que o público tem o direito de
veto sobre as maneiras que o indivíduo pode gastar a sua renda. Temos
somente que supor uma difusão considerável das opiniões socialistas, e
pode se tornar infamante aos olhos da maioria possuir mais propriedade
além de uma pequena quantidade, ou que se tenha renda que não advenha
do trabalho manual. Opiniões semelhantes a essas já prevalecem entre a
classe dos artesãos, e pesa opressivamente naqueles que são
suscetíveis às opiniões dessa classe, isto é, os seus próprios
membros. É sabido que os maus trabalhadores, que formam a maioria dos
que trabalham em muitos ramos da indústria, são decididamente a favor
da opinião de que os maus trabalhadores devem receber o mesmo salário
que os bons, e que ninguém deve, por trabalho extra ou de outra
maneira, ganhar mais pela sua habilidade maior, do que os outros, que
não a tem, conseguem. E esses trabalhadores utilizam um controle moral,
que pode se tornar físico, para impedir os trabalhadores
habilidosos de receber, e os patrões de pagarem, uma remuneração maior
por um serviço melhor. Se o público tivesse qualquer jurisdição sobre
assuntos particulares, não posso ver como essas pessoas poderiam estar
em erro, e como o público a que cada indivíduo particular pertence pode
ser criticado por exercer a mesma autoridade sobre a conduta individual
que o público em geral exerce sobre as pessoas em geral. 

Mas, sem ficar parando em casos supostos, nos nossos dias ocorrem
grandes usurpações da liberdade da vida privada, realmente praticadas,
e ameaças ainda maiores, que tem alguma possibilidade de sucesso, e
opiniões que asseveram ser um direito ilimitado do público não somente
proibir por lei tudo o que ele pensa ser errado, mas também para se
conseguir chegar ao que se pensa ser errado, se proibir conjuntamente
uma série de coisas que se admitem serem inocentes. 

 Sob alegação de se prevenir a intemperança, o povo de uma colônia
inglesa, e de quase metade dos Estados Unidos, foi proibido por lei de
fazer qualquer uso de bebidas fermentadas, exceto para fins médicos,
pois a proibição da fabricação dessas bebidas é, de fato, e como foi
planejado que fosse, a proibição de seu uso. E apesar de a
impossibilidade de pôr a lei em prática ter feito com que ela fosse
revogada em vários estados que a haviam adotado, apesar disso uma
tentativa começou, e está sendo prosseguida com considerável zelo por
muitos filantropos, para que uma lei semelhante seja aplicada ao
país inteiro. A associação, ou ``Aliança'', como ela se chama a si mesma,
que foi formada com esse propósito, tem adquirido certa notoriedade pela
publicidade que foi dada à correspondência ocorrida entre o seu
secretário e um dos pouquíssimos homens públicos que mantém que as
opiniões de um político devem ser baseadas sobre princípios. A parcela
de Lorde Stanley nessa correspondência confirma que as esperanças já
nele depositadas sejam fortalecidas por aqueles que sabem o quão raras
são essas qualidades, tais como \mbox{manifestadas} em algumas de suas aparições
públicas, e que infelizmente pouco aparecem naqueles que fazem parte da
vida política. O órgão da Aliança, que desejaria indicar que ``deplora
profundamente o reconhecimento de qualquer princípio que pode ser
entendido como justificando o fanatismo e a perseguição'', intenta
apontar a ``barreira ampla e intransponível'' que divide tais princípios
dos da associação. 

\begin{hedraquote}
Todos os assuntos relacionados com o pensamento, a
opinião, consciência, para mim, afirma o órgão, são vistos como
estando além da esfera da legislação; e todos aqueles que pertencem aos
atos, hábitos e relações sociais, sujeitos a um poder discricionário
assumido pelo próprio Estado, pertencem àquela esfera. 
\end{hedraquote}

Nenhuma menção é
feita a uma terceira classe, diferente de ambas as anteriores, a saber,
atos e hábitos que não são sociais, mas individuais, apesar de ser
certamente a esta classe que o hábito de beber licores fermentados
pertence. Vender licores fermentados, no entanto, é comércio, e o
comércio é um ato social. Mas o que os intransigentes reclamam não é da
liberdade do vendedor, mas sim da liberdade do comprador e usuário, já
que o Estado pode na prática proibi"-lo de beber, ao tornar
impossível conseguir a bebida. No entanto,
o secretário da Aliança afirma: ``Reivindico o direito, como cidadão, de
legislar quando os meus direitos sociais são invadidos pelos atos
sociais dos outros''. A definição desses direitos sociais é: 

\begin{hedraquote}
Se alguma coisa invade meus direitos sociais, certamente é o comércio de
bebidas fortes. Ele destrói o meu direito primordial de segurança,
através da criação e estímulos contínuos da desordem social. Invade meu
direito à equidade, pela criação de uma miséria que eu devo, através de
meus impostos, ajudar a manter. Impede o meu direito ao livre
desenvolvimento moral e intelectual, ao cercar meus caminhos com
perigos, e por enfraquecer e desmoralizar a sociedade, da qual eu tenho
o direito de reivindicar auxílio mútuo e relações. 
\end{hedraquote}

Uma teoria de ``direitos sociais'', da qual nunca se viu nada igual expresso numa
linguagem distinta, que se resume a isso: Que é o absoluto direito de
todo indivíduo que cada outro indivíduo deve agir em cada instância
como ele próprio, e quem quer que falhe o mínimo que seja nisso, viola
o meu direito social, e me capacita a requerer da legislação que essa
perturbação seja removida. Um princípio tão monstruoso é mais perigoso
do que uma única interferência com a liberdade, pois não há nenhuma
interferência com a liberdade que não se possa justificar; esse princípio
não reconhece nenhum direito à liberdade, exceto talvez o de manter
algumas opiniões em segredo, sem jamais revelá"-las, pois, no momento
em que uma opinião que eu considere prejudicial saia pelos lábios de
alguém, ela invade os meus ``direitos sociais'', se seguirmos os
princípios da Aliança. Essa doutrina concede a toda humanidade um
interesse explícito na perfeição intelectual, moral e até física, de
toda outra pessoa, que será definido por cada reclamante de acordo com
os próprios padrões deste. 

Outro exemplo importante da interferência ilegítima com a justa
liberdade do indivíduo, que não é apenas uma ameaça, mas que há muito
já foi levada em efeito triunfal, é a legislação a respeito do sábado.
Sem dúvida, a abstinência de um dia da semana da ocupação diária
habitual, tanto quanto as exigências da vida o permitirem, apesar de
não ser uma observância religiosa obrigatória para ninguém exceto os
judeus, é um costume altamente benéfico. E já que este costume não
poderia ser observado sem o consentimento geral das classes
industriais, e porque algumas pessoas, por trabalharem neste dia, podem
acabar impondo a mesma necessidade a outras, pode ser permitido e é
justo que a lei deva garantir a cada um a observância pelos outros
deste costume, através da suspensão das operações maiores da indústria
em um dia determinado. Mas esta justificação, fundamentada no interesse
direto que as pessoas têm na observância individual desta prática, não
se aplica às atividades que uma pessoa escolhe por pensar que lhe são
adequadas em seu tempo livre, e nem pode ser tida como adequada, nem no
mais ínfimo grau, para restrições legais sobre os divertimentos. É bem
verdade que a diversão de alguns é o trabalho de outros, mas o prazer,
para nada dizer da recreação útil, vale bem o trabalho de alguns, desde
que este trabalho seja livremente escolhido e possa ser livremente
abandonado. Os trabalhadores estão perfeitamente certos ao pensarem que
se todos trabalhassem aos domingos, sete dias de trabalho seriam
trocados pelo salário de seis dias, mas se o trabalho de uma grande massa de
trabalhadores ficar suspenso, o pequeno número daqueles que
ainda devem trabalhar para o divertimento dos outros obtém um aumento
proporcional de sua renda, e esses não seriam obrigados a seguir essa
ocupação, se preferissem o descanso ao pagamento. Se outro remédio é
procurado, ele pode ser encontrado no estabelecimento de um dia de
descanso em qualquer outro dia da semana para essa classe particular de
pessoas. A única base, portanto, na qual restrições aos divertimentos
dominicais podem ser defendidas, deve ser a de que eles são de um ponto
de vista religioso errados, um motivo de legislação que nunca se deve
deixar de protestar com todas as forças. ``\textit{Deorum injuriae, Diis
curae}''.\footnote{ ``Das injúrias aos deuses os deuses 
se vingam''. Tácito, \textit{Anais}. [\versal{N.T.}]}
 Permanece ainda para ser provado que a sociedade ou qualquer de seus
detentores de cargos foram autorizados pelos céus para vingar
qualquer suposta ofensa ao Onipotente que não seja uma ofensa também
aos nossos pares. A noção de que o dever de um homem para com o
outro deva ser religioso foi o fundamento de todas as perseguições
religiosas já perpetradas e, se admitida, as justificaria plenamente.
Apesar do sentimento que aflora nas repetidas tentativas de se impedir
as viagens de trens aos domingos, na resistência à abertura dos museus,
o estado da mente indicado por essas ações é fundamentalmente o mesmo.
É uma determinação em não se tolerar que outros façam o que é tolerado
pela religião deles, porque tal coisa não é tolerada pela religião do
perseguidor. É uma crença na qual Deus não só abomina o ato do descrente, 
mas que também não nos achará sem culpa, se deixarmos o descrente impune. 

Não posso deixar de adicionar a estes exemplos da pouca importância dada
à liberdade humana a linguagem da mais extremada perseguição que
aparece na imprensa deste país, cada vez que é trazido à tona o
fenômeno ímpar do Mormonismo. Muito pode ser dito sobre o inesperado e
instrutivo fato de que uma alegada nova revelação, e uma religião fundada
sobre ela, produto de uma palpável impostura, nem sequer sustentada pelo
\textit{prestige} de qualidades extraordinárias de seu
fundador,\footnote{ Joseph Smith, linchado depois de ser retirado 
da cadeia onde aguardava julgamento. [\versal{N.T.}]}
 é acreditada por centenas de milhares de pessoas, e que foi
transformada no fundamento de uma sociedade, na época dos jornais, das
ferrovias e do telégrafo elétrico. O que nos interessaria se essa
religião, como outras e melhores religiões, possui os seus mártires,
se o seu profeta e fundador foi, por seus ensinamentos, morto por uma
multidão, se outros de seus adeptos perderam suas vidas pela mesma
violência sem lei, se eles foram expulsos à força, como um corpo, da
região onde tinham primeiro surgido, enquanto que, agora que eles foram
perseguidos até um lugar solitário no meio de um deserto, muitos neste
país declaram abertamente que seria correto (mas que não é conveniente)
enviar uma expedição contra eles, e obrigá"-los a se conformar com as
opiniões das outras pessoas? O artigo da doutrina mormonista que mais
provoca a antipatia, que mais impede sua passagem através das restrições
comuns da tolerância religiosa, é a aceitação da poligamia, a qual,
apesar de ser permitida aos maometanos, hindus e chineses, parece
excitar uma animosidade interminável quando praticada por pessoas que
falam inglês e que pretendem ser um tipo de cristão. Ninguém tem mais
profunda desaprovação dessa instituição mórmon do que eu, seja por
outras razões, ou seja porque ela, longe de ser apoiada pelo princípio
da liberdade, é uma infração direta deste princípio, pondo metade da
comunidade debaixo de correntes, e emancipando a outra metade de
reciprocidade de obrigações com a metade acorrentada. Ainda assim, deve
ser relembrado que essa relação é tão voluntária da parte das mulheres
envolvidas nela, e que podem ser consideradas como sofrendo por isso,
como o é em qualquer outra forma da instituição do casamento e,
conquanto surpreendente esse fato possa aparecer, ele tem uma
explicação nas ideias e costumes comuns do mundo, que ao ensinar as
mulheres que o casamento deve ser considerado necessário, torna
compreensível que muitas mulheres prefiram ser uma dentre várias
esposas do que nunca se casar. Outros lugares não são inquiridos a
reconhecer essas uniões, ou a liberar qualquer parte de sua população
de suas próprias leis por causa das opiniões mormonistas. Mas quando os
dissidentes acederam aos sentimentos hostis dos outros, muito mais do
que poderia ser justamente demandado, quando eles deixaram os lugares
nos quais sua doutrina era inaceitável, e estabeleceram"-se num canto
remoto da terra, que foram os primeiros a tornar habitável para seres
humanos, é difícil ver por quais princípios, exceto os da tirania, eles
podem ser impedidos de ali viverem sob quaisquer leis que lhes agradem,
desde que não agridam outras nações, e permitam perfeita liberdade de
ir embora àqueles que estiverem insatisfeitos com os modos ali
vigentes. Um escritor, que em alguns aspectos é de considerável mérito,
propôs (para usar suas próprias palavras) não uma cruzada, mas uma
\textit{civilizada} contra aquela comunidade polígama, para acabar com
o que parece ser para ele um passo atrás na civilização. Penso também
que é assim, mas não estou cônscio de que nenhuma comunidade tenha o
direito de forçar outra a se tornar civilizada. Enquanto aqueles que
sofrem sob leis ruins não apelarem para a ajuda de outras
comunidades, não posso admitir que pessoas completamente estranhas a
eles devam intervir e exigir que uma condição que todos
os diretamente envolvidos achem satisfatória seja impedida porque é
motivo de escândalo a pessoas que vivem milhares de milhas distantes, e
que não tem parte ou interesse nela. Que elas enviem missionários, se
quiserem, para pregar contra a situação e que elas, por meios justos
(silenciar os pregadores não é um deles) se oponham ao
progresso de doutrinas semelhantes entre seu próprio povo. Se a
civilização levou a melhor sobre o barbarismo, quando o barbarismo
tinha o mundo todo sob seus pés, é demais pretender ter medo de que o
barbarismo, depois de ter decaído tanto, vai reviver e conquistar a
civilização. Uma civilização que sucumbisse a um inimigo derrotado
deveria primeiro se tornar tão degenerada a ponto de ninguém, nem os
seus sacerdotes e mestres, nem ninguém mais, ter a capacidade, ou a
vontade de se dar ao trabalho de defendê"-la. Se for assim, quanto mais
rápido essa civilização receber a ordem de partir, melhor. Ela pode ir
do ruim ao péssimo, até que seja destruída e regenerada (como o Império do
Ocidente)\footnote{ Isto é, o Império Romano. [\versal{N.T.}]} pelos enérgicos bárbaros. 

\chapter{Aplicações}

\textsc{O princípio} expresso nestas páginas deve ser, de modo geral, reconhecido
como a base de discussão dos detalhes, antes que uma aplicação
consistente deles em todos os numerosos departamentos do governo possa
ser tentada com alguma possibilidade de êxito. As poucas observações
que proponho fazer em questões de detalhes estão destinadas a ilustrar
esse princípio, mais do que explicitá"-los em todas as suas
consequências. Ofereço não tanto aplicações, mas exemplos de
aplicações, que podem deixar mais claros os significados e os limites
das duas máximas que juntas formam toda a doutrina deste ensaio, e
ajudar o julgamento, ao sustentar a balança entre eles, nos casos em
que pareça ser duvidosa sua aplicação.

As máximas são que, primeiro, o indivíduo não pode ser responsabilizado,
diante da sociedade, por suas ações, desde que elas não afetem o
interesse de ninguém além dele mesmo. Conselho, intrusão, persuasão e
afastamento das outras pessoas, e elas acharem isso necessário para o
próprio bem delas, são as únicas medidas pelas quais a sociedade pode
justificadamente exprimir o seu desagrado ou desaprovação com a conduta
deste indivíduo. E, segundo, que pelas ações que forem prejudiciais aos
interesses dos outros, o indivíduo é responsável, e pode estar sujeito
seja a punições sociais ou legais, se a sociedade for de opinião que
uma ou outra é necessária para a sua proteção.

Em primeiro lugar, não deve ser de maneira nenhuma suposto que, por
causa dos danos, ou da probabilidade de danos, ao interesse dos outros,
justamente o que justifica a interferência da sociedade, seja
sempre justificável essa interferência. Em muitos casos, um
indivíduo, enquanto busca um objetivo legítimo, necessariamente e
portanto justificadamente, causa sofrimentos ou perdas a outras
pessoas, ou impede que elas obtenham um bem que tinham uma
razoável esperança de conseguir. Tais oposições de interesse entre
indivíduos muitas vezes surgem de más instituições sociais, mas são
inevitáveis enquanto essas instituições durarem, e algumas seriam
inevitáveis sob qualquer instituição. Quem quer que seja bem sucedido
numa profissão muito procurada ou num exame competitivo, quem quer que
seja preferido a outra pessoa num objetivo que ambas desejem,
colhe benefícios com as perdas de outras pessoas, de seus esforços em
vão e de seus desapontamentos. Mas isso, por reconhecimento comum, é
melhor para o interesse geral da humanidade do que se permitir que as
pessoas busquem seus objetivos sem a ameaça deste tipo de
consequências. Em outras palavras, a sociedade não dá direitos, sejam
legais ou morais, para competidores desapontados, que imunizem contra
esse tipo de sofrimento, e se sente chamada a interferir apenas quando
os meios empregados para se alcançar um bom resultado não devem ser
permitidos pelo interesse geral, a saber, fraude, engano ou força. 

Do mesmo modo, o comércio é um ato social. Quem quer que se lance a
vender qualquer tipo de mercadoria ao público afeta os
interesses de outras pessoas, e da sociedade em geral, e portanto a sua
conduta, em princípio, fica sob a jurisdição da sociedade: por isso foi
sustentado anteriormente ser dever dos governos, em todos os casos
considerados de importância, fixar os preços e regulamentar os
processos de manufatura. Mas agora se percebe, não sem antes uma longa
luta ter ocorrido, que tanto o preço barato quanto a boa qualidade das
mercadorias estão mais bem garantidas quando se deixa produtores e
consumidores perfeitamente livres, a única restrição sendo a liberdade
igual para todos os consumidores de fazer suas compras onde quiserem.
Esta é a assim chamada doutrina do livre comércio, que se baseia em
diferentes fundamentos, apesar de igualmente sólidos, que o princípio
de liberdade individual defendido neste ensaio. As restrições sobre o
comércio ou sobre a produção de artigos de comércio são de fato
controles, e todo controle, \textit{qualquer} controle, é um mal. Mas os controles
neste assunto afetam somente a parte da conduta que a sociedade é
competente para restringir, e são errados somente no sentido que não
atingem o objetivo esperado. Já que o princípio da liberdade individual
não está envolvido na doutrina do livre comércio, também não está
envolvido na maioria das questões que surgem a respeito dos limites
dessa doutrina como, por exemplo, quanto controle público é admissível
com o fito de prevenir a fraude pela adulteração de produtos, ou até
onde as precauções sanitárias ou medidas para proteger trabalhadores
em ocupações perigosas devem ser impostas aos empregadores. Tais
questões envolvem considerações de liberdade, mas somente no sentido
em que deixar as pessoas por si mesmas é sempre melhor, \textit{caeteris
paribus},\footnote{ Tudo o mais sendo o mesmo. [\versal{N.T.}]} 
do que controlá"-las, mas que essas questões possam vir a ser
legitimamente controladas para se obter aqueles fins é, em princípio,
inegável. Por outro lado, há questões relacionadas com a
interferência no comércio que são essencialmente questões de liberdade,
tais como a Lei de Maine, já mencionada, a proibição de importação de
ópio para a China, a restrição para a venda de venenos, todos casos,
resumindo, em que o objetivo da interferência é o de impossibilitar ou
dificultar a aquisição de uma determinada mercadoria. Essas
interferências são passíveis de objeção, não como infração da liberdade
do produtor ou do vendedor, mas da liberdade do comprador. 

 Um desses exemplos, o da venda de venenos, abre uma nova questão: os
limites corretos do que pode ser chamado de funções da polícia, o
quanto a liberdade pode ser invadida legitimamente tendo em
vista a prevenção de crimes ou de acidentes. Uma das mais indisputadas
funções do governo, além de descobrir e punir o criminoso, é a de tomar
precauções contra o crime antes de algum ser cometido. Entretanto, a
função preventiva do governo está muito mais propensa a abusar da
liberdade do que a função punitiva, pois dificilmente haverá uma parte
da liberdade legítima de ação de um ser humano que não possa ser vista,
até justamente, como aumentando as possibilidades de um tipo ou outro
de delinquência. Mesmo assim, se a autoridade pública, ou mesmo uma
pessoa privada, percebe que alguém está evidentemente se preparando
para cometer um crime, eles não são obrigados a ficar inativos até que
o crime seja cometido, mas sim podem interferir para prevenir que
aconteça. Se venenos só fossem comprados ou usados para a
realização de \mbox{homicídios}, seria correto proibir a sua confecção e
venda. No entanto, eles podem ser procurados para fins não só
inocentes, mas até úteis, e restrições não podem ser impostas no
primeiro caso sem afetar o segundo. Do mesmo modo, é função precípua
da autoridade pública estar alerta para que acidentes não ocorram. Se
um funcionário público ou qualquer outra pessoa vê alguém tentando
cruzar uma ponte que se sabe que é insegura, e não há tempo para
adverti"-lo do perigo, eles podem segurá"-lo e trazê"-lo de volta,
sem que sua liberdade tenha sido de fato restringida, pois a liberdade
consiste em se fazer o que se quer, e ninguém quer cair num rio. Mesmo
assim, quando não houver a certeza, mas só a possibilidade, de um
acidente, ninguém, além da própria pessoa, pode julgar sobre a
suficiência dos motivos que a levam a enfrentar o risco. Neste caso,
portanto (a menos que ela seja uma criança, ou delirante, ou em algum
estado de excitação ou ensimesmamento que a impeça de utilizar"-se
completamente de sua faculdade judicativa), ela deve, assim penso, ser
alertada para o perigo apenas, e não ser impedida à força de se expor a
ele. Considerações semelhantes, aplicadas à questão da venda de
venenos, podem nos ajudar a decidir quais dentre os possíveis meios de
regulamentação de venda são ou não contrários ao princípio da
liberdade. Precauções tais como a de etiquetar a substância com
palavras que indiquem o caráter perigoso podem ser implementadas sem
violar a liberdade: o comprador não pode não querer saber que a droga
que comprou tem qualidades venenosas. Mas exigir para todos os casos de
compra uma receita de um médico tornaria a obtenção desse artigo para
usos legítimos até impossível às vezes, e sempre mais cara. Para mim, o
único modo pelo qual obstáculos podem ser levantados contra
os crimes cometidos por esse meio, sem nenhuma infração que valha a
pena ser levada em conta, para a liberdade daqueles que desejam a
substância venenosa para outros propósitos, consiste em apresentar o
que, nos termos adequados de Bentham, é chamado de ``evidência
pré"-apontada''. Essa provisão é conhecida por todos nos casos de
contrato. É corriqueiro e correto que a lei, quando um contrato é
registrado, requeira, como condições de seu cumprimento, que certas
formalidades sejam observadas, como assinaturas, declarações de
testemunhas, e assim por diante, para que em caso de alguma disputa
posterior, haja provas de que o contrato foi de fato estabelecido, e
que não havia nada nas suas circunstâncias que o tornasse legalmente
inválido: o efeito sendo o de impedir a feitura de contratos
fictícios, ou de contratos feitos sob circunstâncias que, se
conhecidas, destruiriam sua validade. Precauções de natureza
semelhante podem ser postas em uso quando da venda de artigos que
podem ser utilizados como instrumentos de crimes. O vendedor, por
exemplo, pode ter que registrar a hora exata da transação, o nome e o
endereço do comprador, a quantidade e a qualidade do produto vendido,
perguntar ao comprador qual a finalidade da compra e registrar a
resposta dada. Quando não houver uma prescrição médica, a presença de
uma terceira pessoa pode ser necessária, de modo que o comprador fique
ciente, em caso de, posteriormente, haver razões para se acreditar que o
artigo possa ter sido usado com intuitos criminosos. Regulamentos assim
não seriam impedimentos materiais de monta para a obtenção do produto,
mas seriam consideráveis impedimentos para quem pretendesse usar impropriamente o
produto sem ser descoberto. 

O direito, inerente à sociedade, de prevenir crimes contra si mesma
através de precauções anteriores, sugere óbvias limitações para a máxima
de que uma má conduta puramente autocentrada não está sujeita a
meios de prevenção e de punição. A embriaguez, por exemplo, em casos
comuns, não está sujeita à interferência legislativa, mas eu
consideraria perfeitamente legítimo que uma pessoa, que tenha sido uma
vez condenada por atos de violência contra os outros quando estava sob
a influência da bebida, viesse a ser posta sob uma restrição legal
especial, exclusivamente pessoal, e que se ela fosse mais tarde
encontrada em estado de embriaguez, que ela pudesse ser punida e que,
se nesse estado ela cometesse outra ofensa, a punição que viesse a ser
imposta pudesse ter a sua severidade aumentada. Ficar embriagado,
para alguém que a embriaguez leva a fazer mal aos outros, é um crime
contra os outros. Assim também a vagabundagem, exceto para uma pessoa
que esteja recebendo auxílio público, ou quando constitui uma quebra de
contrato, não pode, a não ser tiranicamente, se tornar objeto de
punição legal, mas se, seja por vagabundagem seja por qualquer outro
motivo que pudesse ser superado, um homem não consegue cumprir seus
deveres legais para com os outros, por exemplo, sustentar seus filhos,
não é tirania forçá"-lo a cumprir suas obrigações através de trabalho
forçado, se outros meios não estiverem à disposição. 

Assim também, há muitos atos que, sendo injuriosos de forma direta
apenas para seus agentes, não devem ser impedidos legalmente, mas que,
se praticados em público, são uma violação das boas maneiras e, caindo
então sob a categoria de ofensas contra as outras pessoas, podem ser
então legalmente proibidos. Deste tipo são as ofensas contra a
decência, nas quais é desnecessário se demorar, já que estão
conectadas apenas indiretamente com nosso assunto, a objeção da
publicidade dos atos sendo igualmente forte contra muitas ações que
não são em si condenáveis e que não se pensa que o devam ser. 

Há outra questão para a qual uma resposta tem de ser encontrada, e que
seja consistente com os princípios que já foram estabelecidos. Nos
casos de conduta pessoal que se suponha serem criticáveis, mas que o
respeito à liberdade não permite que a sociedade impeça ou puna, porque o
mal resultante recai completamente sobre o agente, o que o agente é
livre para fazer, outras pessoas não deveriam ser igualmente livres
para aconselhar ou instigar? Esta questão não está isenta de
dificuldades. O caso de uma pessoa que pede que outra cometa
determinado ato não é estritamente um caso de conduta que só se refere a
si mesma. Oferecer conselhos ou incentivos é um ato social e pode,
portanto, como as ações em geral que afetam os outros, estar
supostamente sujeita ao controle social. Mas um pouco de reflexão
corrige essa primeira impressão, mostrando que se o caso não está
estritamente dentro da definição de liberdade individual, ainda assim
as razões nas quais os princípios da liberdade individual estão
embasados são aplicáveis a ele. Se às pessoas deve ser permitido, no
que só toca a elas, agir como lhes parecer melhor ao seu próprio
risco, a elas deve ser permitido também que consultem livremente outras
pessoas sobre o que deve ser feito, que troquem opiniões, e que recebam
sugestões. O que quer que seja permitido fazer também pode ser
aconselhado que se faça. A questão se torna duvidosa apenas quando o
instigador recebe um benefício pessoal de seus conselhos, quando ele
tem como ocupação, para sua subsistência ou ganho pecuniário, promover
aquilo que a sociedade e o estado consideram ser um mal. Aí então um
novo tipo de complicação se introduz, a saber, a existência de classes
de pessoas que possuem um interesse oposto ao que é considerado o bem
comum, e cujo modo de vida está baseado na contradição dele. Deve"-se
interferir nisso ou não? A fornicação, por exemplo, deve ser
tolerada, assim como o jogo de azar, mas deve alguém ser livre para ser
um cafetão, ou para ser o proprietário de um cassino? O caso é um dos
que ficam na linha de demarcação entre os dois princípios, e não é
aparentemente claro a qual dos lados esse caso propriamente pertence.
Há argumentos para ambos os lados. Pelo lado da tolerância, pode"-se
alegar que o fato de ter algo como ocupação, e viver e ter lucro pela
prática disso, não pode constituir um crime, já que tal prática
seria de outra forma admissível; que o ato deve ser consistentemente
permitido ou proibido; que se os princípios que até aqui temos
defendido são verdadeiros, a sociedade não deve, \textit{como}
sociedade, decidir que algo que só afeta o indivíduo seja errado; que
ela não pode ir além da dissuasão, e que uma pessoa deve ser livre para
persuadir tanto quanto outra deve ser livre para dissuadir. Contrário a
isto, pode"-se afirmar que apesar de o público, ou o Estado, não ter o
direito de decidir de forma taxativa, para propósitos de repressão
ou punição, que tal ou tal conduta que afeta apenas os interesses de um
indivíduo é boa ou má, eles estariam plenamente justificados em
assumir, caso considerem essa conduta como má, que se ela é mesmo assim
ou não é uma questão a ser discutida: pois, supondo desta maneira, eles não
podem agir de forma errônea ao tentar excluir a influência de
solicitações que não são desinteressadas, vindas de instigadores que
não podem ser imparciais --- que tem um interesse pessoal direto em um
lado, o lado que o Estado acredita estar errado, e que confessam
abertamente estarem promovendo apenas seus interesses pessoais. Na
certa, como pode ser arguido, nada há para se perder, nenhum sacrifício
do que é bom, por um ordenamento que leve as pessoas a fazer suas
escolhas, certas ou erradas, por sua própria conta, tão livres quanto
possível das artes de gente que estimularia as inclinações dos outros
para servir a seus propósitos particulares. Assim (pode"-se arrazoar), apesar
das leis a respeito de jogos ilegais serem indefensáveis --- apesar
de todo mundo ser livre para jogar em sua própria casa ou na casa de
outras pessoas, ou em qualquer lugar estabelecido por sua própria
subscrição, aberto apenas aos membros registrados ou a visitantes ---,
os cassinos abertos ao público não devem ser permitidos. É verdade que
a proibição nunca funciona muito bem, não importando qual o nível de
poder tirânico que seja dado à polícia, as casas de jogos podem
funcionar sob vários disfarces, mas elas podem ser forçadas a
funcionar debaixo de certo grau de segredo e de mistério, de forma que
só saberão algo sobre elas aqueles que as procurarem, e mais que isso, a
sociedade não deve almejar. Há uma força considerável nestes
argumentos. Não ousarei decidir se eles são suficientes para justificar
a anomalia moral de se punir o acessório, enquanto que o principal fica
(e deve ficar) incólume, multando ou aprisionando o intermediário, e
não o fornicador, o dono da casa de jogos, mas não o jogador. Menos
ainda a operação comum de venda e compra deve sofrer interferência por
motivos semelhantes. Quase qualquer artigo que é comprado e vendido
pode ser usado em excesso, e os vendedores têm um interesse pecuniário
para encorajar esse excesso, mas nenhum argumento pode ser nisso
baseado que favoreça, por exemplo, a Lei de Maine, pois a classe dos
vendedores de bebidas fortes, apesar de interessada no uso abusivo da
bebida, é indispensável ao uso legítimo da bebida.
No entanto, o interesse desses vendedores em promover a intemperança é
um mal real, e justifica que o Estado imponha restrições e peça
garantias que, se não fosse pela justificação apresentada logo acima,
seriam infrações da liberdade legítima. 

Outra questão é se o Estado, embora as permita, não deveria
desencorajar de forma indireta condutas que são vistas como
contrárias ao melhor interesse do agente, se, por exemplo, ele não
deveria tomar medidas para tornar os meios de se ficar
embriagado mais caros, ou os tornar mais difíceis de serem encontrados
limitando os números de locais de venda. Sobre essas, como sobre a
maioria das questões práticas, muitas distinções precisam ser feitas.
Taxar estimulantes apenas com o propósito de torná"-los mais difíceis
de serem obtidos é uma medida que difere apenas em grau da proibição
completa, e seria justificável apenas se essa última fosse
justificável. Cada aumento de preço é uma proibição, para aqueles cujos
meios não alcançam os preços aumentados, e para aqueles que o
conseguem, é uma penalidade dada a eles por gratificar um gosto
particular. A escolha de prazeres, e o modo de gastar a renda, depois
de satisfazer as suas obrigações legais e morais com o Estado e com
outros, interessa apenas a cada pessoa, e deve permanecer dentro de seu
próprio julgamento. Estas considerações podem à primeira vista parecer
que condenam a escolha de estimulantes como objetos especiais de
taxação para o propósito de aumento da renda estatal. Mas deve ser
relembrado que a taxação para propósitos fiscais é absolutamente
inevitável, e que em muitos países é necessário que uma considerável
parte da taxação seja indireta, que o Estado, portanto, não pode deixar
de impor taxas sobre o uso de alguns artigos de consumo que para
algumas pessoas serão proibitivas. Portanto, é dever do Estado
considerar, quando da imposição de taxas, quais mercadorias os
consumidores podem melhor deixar de lado e, \textit{a fortiori}, selecionar de
preferência aqueles produtos cujo uso, além de uma quantidade
moderada, pode ser positivamente prejudicial. Portanto, a taxação de
estimulantes, até o ponto que produz a maior entrada de dinheiro
(supondo que o Estado necessite de toda a renda que possa recolher) não
é somente admissível, mas deve ser aprovada.

A questão sobre tornar a venda dessas mercadorias um privilégio mais ou
menos exclusivo pode ter diferentes respostas, de acordo com o
propósito para o qual a restrição é planejada. Todos os lugares com
frequentação pública necessitam da atenção da polícia, e lugares
onde se vendem bebidas mais ainda, já que ofensas à sociedade estão
mais propensas a ocorrerem ali. Portanto, é adequado que se limite o
poder de vender essas mercadorias a pessoas de conhecida, ou atestada,
respeitabilidade de conduta, marcar horários de abertura e fechamento
de modo a preencher os requisitos da vigilância pública, e retirar
a licença de um lugar desses se perturbações da paz ocorrerem repetidamente
nele, seja pela conveniência ou pela incapacidade do proprietário, ou
se o lugar se tornar um valhacouto para planejamento de operações
contra a lei. Qualquer outra restrição eu não concebo, em princípio,
ser justificada. Por exemplo, a limitação do número de estabelecimentos
que vendam cervejas e bebidas fortes, feita com o propósito expresso de
tornar o acesso mais difícil a esses lugares, e diminuir as ocasiões
para a tentação, não somente expõe a todos a uma inconveniência, porque
na certa haverá alguns que abusarão das bebidas nesses lugares, mas
essa limitação só é adequada a um estado da sociedade no qual as
classes trabalhadoras são claramente tratadas como crianças ou
selvagens, postos sob uma educação repressora, para adequá"-los para
uma futura admissão aos privilégios da liberdade. Esse não é o
princípio no qual as classes trabalhadoras são reconhecidamente
governadas em qualquer país livre, e ninguém que dê o devido valor à
liberdade concordará que elas sejam governadas desta maneira, a menos
que todos os esforços para educá"-las para a liberdade e governá"-las
como homens livres tenham sido exauridos, e que tenha sido
definitivamente provado que elas podem ser apenas governadas quais
crianças. A simples expressão da alternativa mostra o absurdo de se
supor que tais esforços tenham sido tentados em qualquer caso que
necessite ser aqui considerado. É apenas porque as instituições deste
país são uma massa de inconsistências que coisas assim, que pertencem
ao sistema do despotismo ou do que é chamado de governo paternal, são
admitidas nas nossas práticas, enquanto que a liberdade geral de nossas
instituições impede o exercício da quantidade de controle necessário
para tornar as restrições de qualquer eficácia como educação moral.

Foi apontado numa parte anterior deste ensaio que a liberdade do
indivíduo, nas coisas que digam respeito somente ao indivíduo, implica
uma liberdade correspondente para qualquer número de indivíduos
regrarem por concordância mútua as coisas que dizem respeito a eles
conjuntamente, e que não diz respeito a ninguém mais além deles mesmos.
Esta questão não apresenta dificuldades, desde que as vontades das
pessoas envolvidas continuem inalteradas, mas como isso pode
mudar, é frequentemente necessário, mesmo nas coisas que só dizem
respeito a elas, que entrem em negociações umas com as outras, e
quando o fizerem, é adequado, como regra geral, que esses acordos sejam
mantidos. No entanto, nas leis de provavelmente todos os países, essa
regra geral tem exceções. Não somente ninguém deve manter acordos que
violem os direitos de uma terceira parte, mas é muitas vezes
considerada como uma razão para livrar as pessoas de um acordo
que este seja daninho para elas. Neste, e em outros países civilizados,
por exemplo, um acordo pelo qual uma pessoa possa vender a si mesma, ou
se deixe vender, como escrava, deve ser tido como nulo e vazio, e não
ser imposto nem pela lei nem pela opinião. A base para essa limitação
do poder voluntário de uma pessoa de dispor voluntariamente de seu
quinhão da vida é clara, e é vista com nitidez nesse caso extremo. A
razão para a não interferência com os atos voluntários de uma pessoa, a
menos que haja outras pessoas envolvidas, é a consideração pela sua
liberdade. A escolha voluntária é a evidência de que o que ela escolheu
lhe é desejável, ou pelo menos suportável, e que o seu bem é, no geral,
mais bem alcançado permitindo que ela utilize os seus próprios meios para
tanto. Mas, vendendo a si mesma ela abdica de sua liberdade,
anulando qualquer uso dela no futuro através de um simples ato.
Ela portanto derrota, em seu próprio caso, o real propósito que é a
justificativa para deixá"-la dispor de si mesma. Ela não está mais
livre, mas está a partir daí numa posição na qual não valerá a
prerrogativa a seu favor que poderia ser fornecida pela sua permanência
voluntária nessa situação. O princípio da liberdade não pode exigir que
ela deva ser livre para não ser livre. Não é liberdade se ter permissão
para alienar a liberdade. Esses raciocínios, cuja força é
conspícua neste caso particular, possuem evidentemente uma ampla
aplicação, no entanto um limite é posto neles em toda parte pelas
necessidades da vida, que continuadamente requerem não que devamos
resignar a nossa liberdade, mas que devamos consentir com essa ou
aquela limitação dela. Entretanto, o princípio, que exige liberdade de
ação incontrolada no que só diz respeito aos agentes mesmos, requer que
aqueles que fizeram acordos entre si, em coisas que envolvem uma
terceira parte, possam separar uns dos outros desses compromissos: e
mesmo sem tal desligamento voluntário, talvez não haja contratos ou
compromissos, exceto no que se relaciona a dinheiro ou a valores, sobre os quais
se possa ousar dizer que não há nenhuma liberdade de se
retirar deles. O barão Wilhelm von Humboldt, no seu excelente ensaio que
já citei, afirma ser sua convicção que os compromissos que envolvem
relações pessoais ou de serviços só devem ser legalmente obrigatórios
por um período limitado de tempo, e que o mais importante desses
compromissos, o casamento, tendo a peculiaridade de seu objetivo estar
frustrado a menos que os sentimentos das duas partes estejam em
harmonia sobre ele, deve exigir apenas a vontade declarada de qualquer
das partes para que seja dissolvido. Este assunto é importante demais,
e complicado demais para ser discutido num parêntese, e tocarei nele
apenas o quanto for necessário para propósitos de ilustração. Se a
concisão e a generalidade da dissertação do barão Humboldt não o
tivesse obrigado a se contentar com a enunciação de suas conclusões sem
discutir as premissas, ele certamente teria reconhecido que a questão
não pode ser decidida em bases tão simples como aquelas às quais ele se
confinou. Quando uma pessoa, seja por promessa explícita ou por sua
conduta, encorajou outra a confiar na continuação de seus atos de certa
maneira --- a criar expectativas e cálculos, e a colocar qualquer parte de
seu plano de vida sob essa suposição ---, um novo tipo de obrigações
morais surge de sua parte para com a outra pessoa, que podem ser talvez
descartadas, mas não ignoradas. Mais ainda, se a relação entre duas
partes contratantes for acompanhada de consequências para outras
pessoas, se colocar uma terceira parte numa posição específica ou, como
no caso do casamento, fizer com que terceiras partes existam, surgem
obrigações de ambos os contratantes para com aquelas
terceiras pessoas --- o cumprimento, ou o modo de cumprimento dessas
obrigações, será grandemente afetado pela continuação ou disrupção da
relação entre as partes originais do contrato. Admito que daí não se
segue que essas obrigações se estendam de tal forma que exijam o
cumprimento do contrato a todo custo, contra a felicidade da parte
relutante, mas elas são um elemento necessário na questão, e mesmo se,
como Von Humboldt sustenta, elas não devem apresentar nenhuma
diferença na liberdade \textit{legal} que as partes possuem de se
livrar do compromisso (também mantenho que elas não devem fazer
\textit{muita} diferença), necessariamente fazem uma grande
diferença para a liberdade moral. Uma pessoa deve levar em conta todas
essas circunstâncias, antes de resolver a dar um passo que poderá
vir a afetar importantes interesses de outras pessoas, e se ela não der
o peso apropriado a esses interesses, ela é moralmente responsável pelo
mal que pode acontecer. Lancei essas observações óbvias para ilustrar
melhor o princípio geral da liberdade, e não porque elas sejam de
maneira nenhuma necessárias para esta questão em particular, que, pelo
contrário, é usualmente discutida como se os interesses dos filhos
fossem tudo, e o das pessoas adultas nada. 

Observei anteriormente que, devido à ausência de um princípio geral
reconhecido, a liberdade é frequentemente concedida quando deveria ser
retirada, e retirada quando deveria ser concedida, e um dos casos no
qual, no atual mundo europeu, o sentimento de liberdade é mais
profundo, esse sentimento está totalmente mal colocado. Uma pessoa deve
ser livre para fazer como quiser em seus próprios assuntos, mas ela não
deve ser livre para agir como quiser por outra pessoa,
pretextando que os assuntos de outras pessoas são também seus assuntos.
O Estado, enquanto respeita a liberdade de cada pessoa no que
especificamente diz respeito a ela, é obrigado a manter um controle
vigilante sobre qualquer poder que
lhe seja concedido exercer sobre outras pessoas. Essa obrigação é quase
que totalmente negligenciada no caso das relações familiares, um caso
que, pela sua influência sobre a felicidade humana, é mais importante
do que todos os outros somados. O poder quase despótico dos maridos
sobre as esposas não precisa ser tratado com maior espaço agora, porque
nada mais é tão necessário para a remoção completa do mal do que as
esposas virem a ter os mesmos direitos, e receberem a mesma proteção da
lei da mesma maneira que todos e porque, neste assunto, os
defensores da injustiça vigente não se apresentam como campeões da
liberdade, mas sim, e abertamente, como campeões do poder. É no caso
das crianças que aquelas deslocadas noções de liberdade são um
obstáculo real para que o Estado cumpra os seus deveres. Alguém poderia
vir a pensar até que os filhos de um homem são literalmente, e não
metaforicamente, uma parte dele, tal o ciúmes demonstrado diante
da menor interferência da lei com o controle absoluto e exclusivo que
mantém sobre eles, mais ciumento do que em quase qualquer
interferência sobre a sua própria liberdade de ação, tanto a maior parte da
humanidade valoriza mais o poder do que a liberdade. Considere, por
exemplo, o caso da educação. Não seria um axioma autoevidente, que o
Estado deve requerer e exigir a educação de todo ser humano que nasceu
seu cidadão, até certo padrão? No entanto, quem é que não tem receio de
reconhecer e afirmar esta verdade? Dificilmente se negará que é um dos
deveres sagrados dos pais (ou, como a lei e o uso agora colocam, do 
pai), depois de colocar um ser humano no mundo, proporcionar
àquele ser uma educação que lhe dê condição de cumprir bem a sua parte
na vida para com os outros e para consigo mesmo. Mas, enquanto se
afirma unanimemente ser este o dever do pai, dificilmente se encontrará,
neste país, \mbox{alguém} que queira ouvir sobre ser obrigado a cumprir esse
dever. Ao invés de ser obrigado a fazer qualquer esforço ou sacrifício
para assegurar a educação da criança, se deixa à sua escolha aceitar ou
não a educação quando esta é dada gratuitamente! Permanece ainda sem
ser reconhecido que trazer uma criança ao mundo sem que haja uma boa
possibilidade de dar a ela não apenas alimento para seu corpo, mas
instrução e treinamento para sua mente, é um crime moral, tanto contra
o desafortunado rebento quanto contra a sociedade, e que se o pai não
cumpre essa obrigação, o Estado deve fazer com que ela seja cumprida,
com o próprio pai, tanto quanto isso seja possível, pagando as despesas. 

Fosse admitido o dever da exigência da educação universal, não haveria
fim nas controvérsias sobre o quê e como o Estado deveria ensinar, que
atualmente tornam esses assuntos um mero campo de batalha para seitas
e partidos, despendendo o tempo e o trabalho que deveria ter sido
gasto para educar em querelas sobre a educação. Se o governo
decidisse \textit{exigir} que cada criança tivesse uma boa educação,
poderia se livrar do problema de \textit{prover} essa
educação. Deveria deixar aos pais arranjar a educação como e
onde achassem melhor, e se contentar em ajudar a pagar as
despesas escolares para as crianças das classes mais pobres, e custear
todo o estudo daquelas que não tivessem ninguém mais que pudesse pagar
por elas. As objeções corretamente levantadas contra a educação estatal
não se aplicam à exigência da educação obrigatória por parte do Estado,
mas sim à direção dessa educação pelo Estado, o que é algo totalmente
distinto. Eu vou tão longe quanto qualquer um nas
críticas à ideia de que toda a educação das pessoas, ou a sua maior parte, deva
estar nas mãos do Estado. Tudo o que tem sido dito sobre a importância da
individualidade do caráter e sobre as diversidades de opiniões e modos
de conduta envolve, com importância inexprimível, a
diversidade de educação. Uma educação geral estatal é apenas um meio
para se moldar as pessoas uma exatamente como a outra, e como os moldes
nas quais elas são postas são aqueles que agradam ao poder dominante,
seja ele monárquico, sacerdotal, aristocrático, ou o da maior parte da
geração atual, e, na proporção em que ela é eficiente e bem sucedida,
estabelece um despotismo sobre a mente que leva, por uma tendência
natural, a um despotismo sobre o corpo. Uma educação estabelecida e controlada
pelo Estado deveria apenas existir, se devesse existir alguma, apenas
um entre muitos experimentos em competição, levado a cabo com o
propósito de exemplo e estímulo, para manter os outros dentro de certo
padrão de excelência. De fato, apenas se a sociedade estiver num estado
tão atrasado que não consiga prover por si própria nenhuma instituição
educacional adequada, o governo deve tomar o encargo; então, o governo
pode, como o menor entre dois males, se encarregar da administração de
escolas e universidades, como pode fazer com companhias
privadas, quando estas não existirem no país de um modo que lhes
permita fazer grandes empreendimentos da indústria. Mas, de modo geral,
se o país tiver um suficiente número de pessoas qualificadas que possam
dar educação sob auspícios governamentais, essas mesmas pessoas estarão
hábeis e dispostas a dar uma educação igualmente boa em princípios
voluntários, sob a segurança de uma remuneração dada por uma lei que
obrigue a uma educação compulsória, combinada com a ajuda estatal para
aqueles que não puderem arcar com as despesas com a educação.

O instrumento para fazer a lei funcionar não pode ser outro que o dos
exames públicos, que abrangeriam todas as crianças, e que começariam
quando estas tivessem ainda pouca idade. Deve ser fixada uma idade a
partir da qual cada criança seria submetida a um exame, para se
constatar se ele, ou ela, sabe ler. Se uma criança mostra que é incapaz
disso, seu pai, a menos que apresente desculpas aceitáveis, seria
multado numa quantia módica que, se tal fosse preciso, seria retirada
de seu trabalho, e a criança posta numa escola onde as despesas seriam
pagas por ele. Uma vez por ano o exame deve ser repetido,
aumentando"-se a gama de matérias gradualmente, de forma a tornar a
aquisição universal de certo grau mínimo de conhecimento, e mais ainda
sua retenção, virtualmente compulsória. Além desse mínimo, devem
ocorrer exames voluntários em todos os assuntos, e todos que chegassem
a um nível padrão de proficiência poderiam exigir um certificado. Para
prevenir que o Estado, através dessas medidas, exerça uma influência
inapropriada sobre a opinião, o conhecimento a ser requerido para que
se passe num exame (além das partes meramente instrumentais do
conhecimento, como as das linguagens e seu uso) deve, mesmo nos exames
mais avançados, estar confinado exclusivamente aos fatos e às ciências.
Os exames sobre religião, política, ou outros tópicos disputados, não
devem girar sobre verdades e falsidades de opiniões, mas sobre a  matéria
de fato em que tal e tal opinião é sustentada, em tal ou tal
base, por tal ou tal autor, ou escola, ou igreja. Sob este sistema, a
geração que surge não estará em pior situação, a respeito de todas as
verdades em disputas, do que a geração atual: ela crescerá como
ortodoxa ou heterodoxa, tal como é agora, o Estado apenas tomando
cuidado para que seja instruída por ortodoxos e heterodoxos bem
instruídos. Não há nada que impeça que a religião seja ensinada nas
mesmas escolas em que outras matérias o são. Todas as tentativas do
Estado em influenciar as conclusões de seus cidadãos são más, mas seria
muito apropriado que se oferecesse a oportunidade de se aferir e
certificar que uma pessoa realmente seja detentora de um conhecimento,
sobre um determinado assunto, que poderia se levar em conta. Um
estudante de filosofia estaria mais bem posicionado se fosse capaz de
passar por um exame que versasse sobre Locke e Kant, seja qualquer dos
dois, ou nenhum, o que ele siga, e não há objeção razoável em se
examinar um ateu sobre as evidências do cristianismo, desde que não se
requeira que ele professe sua crença nesta fé. No entanto, os exames
nos altos ramos do conhecimento, penso que devem ser completamente
voluntários. Seria dar ao governo um poder excessivamente perigoso se
ele tivesse a capacidade de excluir qualquer pessoa de uma profissão,
mesmo da profissão de professor, devido a uma alegada falta de
qualificações, e penso, juntamente com Wilhelm von
Humboldt\footnote{ Ver \textit{The Sphere of Government}, op. cit., p.~123. [\versal{N.A.}]}
 que graus, ou outros certificados públicos de conhecimentos científicos
ou profissionais, devem ser concedidos a todos aqueles que se
apresentarem para fazerem os exames e que passarem neles, mas que esses
certificados não devem conferir nenhuma vantagem sobre outros
competidores, mais do que o peso que possa ser dado pela opinião
pública à sua existência.

 Não é apenas em assuntos de educação que mal colocadas noções de
liberdade impedem que obrigações morais por parte dos pais sejam
reconhecidas, e que obrigações legais sejam impostas, isso quando há
sempre sólidos motivos para as primeiras e muitas vezes para as
segundas. O próprio fato de causar a existência de um ser humano é uma
das ações que carregam mais responsabilidades dentre os confins da vida
humana. Tomar essa responsabilidade --- fazer surgir uma vida que pode
ser tanto uma maldição quanto uma bênção ---, a menos que o ser ao qual
a vida está sendo dada tenha pelo menos uma chance mediana de usufruir
de uma existência agradável, é um crime contra esse ser. E num país
superpovoado, ou a ponto de se tornar, produzir filhos, além de um
número bem pequeno, com o propósito de reduzir a paga do trabalho
através do aumento de competidores, é uma séria ofensa contra aqueles
que vivem de seu trabalho. As leis que, nos países da Europa,
proíbem o casamento a menos que os interessados demonstrem que podem
sustentar uma família não excedem os poderes legítimos do Estado, e
sejam essas leis funcionais ou não (o que depende principalmente de
circunstâncias e sentimentos locais), não podem ser objetadas como
violações da liberdade. Essas leis são uma interferência do Estado para
proibir um ato maléfico --- um ato tão danoso para os outros que deveria
ser objeto de reprovação e de um estigma social, mesmo quando não se
ache cabível que se acrescente ainda alguma punição legal. No entanto,
as ideias correntes sobre a liberdade, que se inclinam tão facilmente a
infringir de verdade a \mbox{liberdade} de um \mbox{indivíduo}, por coisas que só
dizem respeito a ele, repeliriam qualquer tentativa de opor qualquer
impedimento sobre suas inclinações, quando a consequência dessa
indulgência é uma vida ou vidas de privações para os rebentos, com
males dos mais diversos aspectos afetando todos aqueles que estejam
perto o suficiente. Quando
comparamos o estranho respeito da humanidade para com a liberdade com a
sua estranha falta de respeito por ela, podemos imaginar que um homem
tenha um direito inalienável para fazer mal aos outros, e nenhum
direito para divertir"-se sem causar dor a ninguém mais. 

 Reservei para o último lugar uma grande classe de questões a respeito
dos limites da interferência governamental, as quais, apesar de estarem
muito próximas do assunto deste ensaio, não pertencem estritamente a
esse. São casos nos quais as razões contra a interferência não giram
sobre o princípio da liberdade, a questão sendo não a restrição das
ações dos indivíduos, mas sim sobre a ajuda aos indivíduos:
pergunta"-se se o governo deve, ou pode fazer com que aconteça algo em
benefício das pessoas, ao invés de deixar que tudo seja feito por elas,
individualmente ou em combinação voluntária. 

As objeções contra a interferência governamental, quando esta não for de
modo a envolver infrações da liberdade, podem ser de três tipos: 

A primeira quando a coisa a ser feita será provavelmente
mais bem feita por indivíduos do que pelo governo. Falando de forma
geral, não há ninguém tão adequado para conduzir algum negócio, ou
determinar como e por quem ele será conduzido, do que aqueles que
estiverem pessoalmente interessados nele. Esse princípio condena as
interferências, antes tão comuns, pela legislatura por oficiais do
governo nos processos corriqueiros da indústria. Mas esta parte deste assunto
foi trabalhada suficientemente pelos economistas políticos, e não está
particularmente relacionada com os princípios deste ensaio.

A segunda objeção está mais próxima de nosso assunto. Em muitos casos,
apesar de alguns indivíduos não conseguirem fazer alguma coisa
específica, no geral, tão bem quanto os funcionários do governo, mesmo
assim é desejável que essa ação seja realizada por eles, em vez do
governo, como um meio de educação mental para eles --- um modo de
fortalecer suas faculdades ativas, exercendo seu julgamento, e lhes
dando um conhecimento maior dos assuntos com os quais terão de
lidar. Essa é a principal, se bem que não a única, recomendação para o
julgamento por júri (em casos que não forem políticos), as
instituições municipais e populares livres, a direção de
empreendimentos industriais filantropos por meio de associações
voluntárias. Essas não são questões de liberdade, estando conectadas
com o assunto apenas remotamente, mas sim questões de desenvolvimento.
Em outra ocasião poder"-se"-á tratar dessas coisas
como partes de uma educação nacional, como, na verdade, o
treinamento específico de um cidadão, a parte prática da educação
política de um povo livre, retirando"-o do estrito círculo dos
egoísmos privados e familiares e acostumando"-o à compreensão de
interesses conjuntos, à gerencia de negócios em comum --- habituando"-o a
agir por motivos públicos ou semipúblicos, e a guiar sua conduta por
objetivos que unam em vez de isolar uma pessoa da outra. Sem esses
hábitos e poderes, uma livre constituição não pode nem ser feita nem
preservada, como mostram a natureza tão transitória da liberdade
política em países onde essa não se fundamenta sobre uma base
suficiente de liberdades locais. A gerência de negócios puramente
locais pelas localidades, e das grandes empresas industriais pela união
daqueles que voluntariamente possam suprir os meios pecuniários, vem
recomendada por todas as vantagens que foram postas nesse ensaio como
pertencendo à individualidade do desenvolvimento, e diversidade de
meios de ações. As operações governamentais tendem a ser parecidas em
todos os lugares. Já com indivíduos e associações voluntárias se dá o
contrário, há experimentos variados e uma diversidade de experiência
sem fim. O que o Estado pode fazer de útil é tornar"-se o repositório
central, e ativo distribuidor das experiências
resultantes de muitas tentativas. O negócio do Estado é possibilitar
que cada experimentador se beneficie das experiências dos outros,
em vez de não tolerar nenhum experimento além dos próprios.

A terceira, e mais cogente razão para se restringir a interferência
governamental é o grande mal de se aumentar desnecessariamente o seu
poder. Cada função adicionada àquelas já executadas pelo governo faz
com que sua influência sobre esperanças e meios se torne mais ampla e
converta, cada vez mais, a parte ativa e ambiciosa do público e meios
dependentes do governo, ou de algum partido que almeje tornar"-se o
governo. Se as estradas, as ferrovias, os bancos, as seguradoras, as
grandes companhias de capital aberto, as universidades e as entidades assistenciais
públicas forem todos ramais do governo, e se, além disso, as corporações
municipais e as juntas locais, com tudo aquilo por que agora elas são
responsáveis, tornarem"-se departamentos da administração central, se os
empregados de todas essas empresas forem colocados e pagos pelo
governo, e passarem a esperar do governo por toda e qualquer melhoria
de vida, nem toda a liberdade de imprensa e a constituição popular da
legislatura poderia tornar este, ou qualquer outro, país livre, exceto
no nome. E o mal seria ainda maior o quanto mais eficiente e
cientificamente a máquina administrativa fosse construída --- os
arranjos para se conseguir cabeças e mão de obra qualificadas mais bem
planejados. Na Inglaterra tem sido sugerido que todos os membros do
serviço civil do governo devam ser selecionados através de exames, para
que as pessoas mais inteligentes e bem instruídas sejam escolhidas, e
muito se tem escrito a favor e contra essa proposta. Um dos argumentos
mais utilizados pelos oponentes é o de que a ocupação de
funcionário oficial permanente do Estado não possui atrativos
suficientes, em termos de dinheiro e de posição, para conseguir os mais
talentosos, que sempre poderão achar uma carreira mais
vantajosa nas diversas profissões, ou no serviço das companhias e
outros empreendimentos. Não haveria nenhuma surpresa se esse argumento
tivesse sido utilizado pelos favoráveis à proposta, como uma resposta à
sua principal dificuldade. Vindo de seus oponentes, seria um argumento bem
estranho. O que está apontado como uma objeção é na verdade a válvula
de segurança do sistema proposto. Se de fato todos os grandes talentos
do país \textit{pudessem} ser engajados nos serviços governamentais,
uma proposta que tendesse a levar a esse resultado poderia bem vir a
causar inquietação. Se cada parte dos negócios da sociedade que
requerem ação organizada, ou visões largas e compreensivas, estivessem
nas mãos do governo, e se os escritórios do governo estivessem povoados
pelos homens mais hábeis, toda a alta cultura e a inteligência prática
do país, excetuando"-se a puramente especulativa, estaria concentrada
numa numerosa burocracia, da qual o resto da comunidade esperaria todas
as coisas: a multidão por direção e ordens em tudo que tem que ser
feito, os capazes e ambiciosos para avanços pessoais. Ser admitido nos
quadros da burocracia e, quando admitido, subir através de seus
degraus, seria o único objeto de ambição. Debaixo deste
\textit{régime}, não somente estaria o público de fora mal qualificado,
por falta de experiência prática, para criticar ou julgar o modo de
operação da burocracia, mas mesmo que, acidental ou naturalmente, o
funcionamento das instituições populares façam subir por acaso aos
píncaros do poder um dirigente, ou grupo de dirigentes de inclinações
reformistas, nenhuma reforma poderá ser efetuada que seja contrária aos
interesses da burocracia. Esta é a melancólica condição do império
russo, segundo os relatos daqueles que tiveram
oportunidade de observá"-lo. O próprio tzar é incapaz
diante do corpo burocrático, ele pode mandar qualquer burocrata para a
Sibéria, mas não pode governar sem eles, ou contra a vontade deles.
Sobre cada decreto seu, os burocratas tem o poder de veto tácito,
bastando que se abstenham de pô"-lo para funcionar. Em países de
civilização mais avançada e com um espírito mais insurrecto, o público,
acostumado a esperar que tudo seja feito para ele pelo Estado, ou pelo
menos a nada fazer sem antes pedir ao Estado não só permissão mas
também instruções de como deve ser feito, naturalmente pensa ser o
Estado responsável por todo o mal que possa cair sobre si, e quando os
males excedem a sua quota de tolerância, ergue"-se contra o governo e
faz o que é chamado de revolução, com a qual alguém, com ou sem
autoridade legítima dada pela nação, assume o comando, dá as suas
ordens para a burocracia, e tudo vai como dantes, a burocracia
continuando a mesma e ninguém mais sendo capaz de assumir o lugar dela.


Um espetáculo muito diferente é exibido por um povo acostumado a fazer
seus próprios negócios. Na França, na qual uma grande parte da
população serviu nas forças armadas, e muitos chegaram pelo menos até o
grau de oficiais não"-comissionados, em cada insurreição popular há
várias pessoas competentes para tomar a liderança, e que podem
improvisar um plano aceitável de ação. O que os franceses são em
assuntos militares, os americanos são em todos os tipos de negócios
civis: que se os deixem sem governo, e cada grupo de americanos será
capaz de improvisar um, e levar adiante esse ou qualquer outro negócio
público com um grau suficiente de inteligência, ordem e decisão. É
assim que cada povo livre deveria ser, e um povo capaz disso certamente
é livre, e nunca será escravizado por um homem ou por um grupo de
homens porque ele ou eles são hábeis o bastante para puxar as rédeas da
administração central. Nenhuma burocracia pode esperar que um povo
assim faça ou suporte algo que ele não quer. Mas onde tudo é feito
através da burocracia, nada daquilo a que a burocracia for realmente
contrária poderá ser feito. A constituição desses países é a
organização da experiência e da habilidade prática da nação em um corpo
disciplinado com o propósito de governar o resto, e quanto mais essa
organização for perfeita, quanto mais bem"-sucedida em acolher em si, e
educar para si, as pessoas de maior capacidade de todos os níveis da
comunidade, maior será a servidão de todos, os membros da burocracia
incluídos. Pois os que mandam são tão escravos de sua organização e
disciplina como os comandados os são dos comandantes. Um mandarim
chinês não passa de instrumento e criatura de um despotismo, tal qual o mais
humilde camponês. Um jesuíta é, até o mais baixo nível, o escravo da
sua ordem, apesar da ordem mesma existir para o poder coletivo e
importância de seus membros. 

Não se pode também esquecer que a absorção de todas as principais
habilidades do país pelo corpo governante será cedo ou tarde fatal para
a atividade mental e o progresso do próprio corpo. Unidos como
são, trabalhando dentro de um sistema que, como todos os sistemas, age
necessariamente em grande medida através de regras fixas --- o corpo de
funcionários está sempre sob a constante tentação de cair numa rotina
indolente ou, se de vez em quando eles deixam de lado esse círculo
estafante, é para correr atrás de alguma barbaridade mal entendida que
atiçou a fantasia de algum líder membro do corpo, e o único controle
dessas tendências aliadas, embora aparentemente opostas, o único
estímulo que pode manter a capacidade num alto nível do próprio corpo é
a possibilidade real de que críticas vindas de fora, feitas por pessoas
de igual capacidade, tenham de ser levadas em conta. Portanto, é
indispensável que devam existir meios, independentes do governo, de
formar essas capacidades, e proporcionar a elas oportunidades e
experiências necessárias para que possam fazer julgamentos corretos
sobre casos práticos de grande envergadura. Se devemos manter um capaz
e eficiente corpo permanente de funcionários --- acima de tudo, um corpo
capaz de originar e implementar melhorias ---, se não quisermos que nossa
burocracia se degenere em pedantocracia, esse corpo não deve abarcar
todas as ocupações que formam e cultivam as faculdades necessárias para
o governo da humanidade.

Determinar o ponto no qual males tão formidáveis para a liberdade e o
progresso humano começam, ou melhor, quando eles começam a predominar
sobre os benefícios que traz a aplicação coletiva da força da
sociedade, sob seus chefes reconhecidos, para a remoção de obstáculos
que ficam no caminho de seu bem"-estar, assegurar quanto das vantagens
de um poder e informações centralizados se pode ter antes que os
canais governamentais tomem uma proporção exagerada da atividade
geral --- é uma das questões mais dificultosas e complicadas da arte de
governar. Em grande medida, é uma questão de detalhes, nos quais muitas
e variadas considerações devem ser mantidas sob a vista, e nenhuma
regra absoluta pode ser estabelecida. Mas creio que o princípio prático
no qual a segurança se assenta, o ideal que deve ser mantido sob a
vista, o padrão pelo qual se deve medir as tentativas de se superar as
dificuldades, pode ser expresso nessas palavras: A maior disseminação
de poder consistente com a eficiência, mas a maior centralização de
informação possível, aliada à difusão desta a partir do centro. Assim,
na administração municipal, como acontece nos estados da Nova
Inglaterra, uma minuciosa divisão entre diversos funcionários,
escolhidos pelas localidades, de todos os negócios que seria melhor não
deixar com as pessoas diretamente interessadas neles mas, ao lado
disso, teria que haver, em cada departamento dos assuntos locais, uma
\mbox{superintendência} central, que seria um ramo do governo. A intenção
dessa superintendência seria a de concentrar, como num foco, a
variedade de informações e experiências derivadas das ações daquele
ramo dos negócios públicos em todas as localidades, das experiências
análogas que são feitas em países estrangeiros, e dos princípios gerais
da ciência política. Esse órgão central deverá ter o direito de saber
tudo que está sendo feito, e o dever específico de tornar o
conhecimento adquirido em um lugar acessível aos outros. Liberado dos
pequenos preconceitos e vistas estreitas pela sua elevada posição e sua
abrangente esfera de observação, as suas recomendações certamente terão
muito peso, mas o seu poder real deverá, como penso, estar limitado a
obrigar os funcionários locais a obedecer as leis baixadas para a
orientação deles. Em tudo que não estiver previsto pelos regulamentos
gerais, esses funcionários devem ser deixados ao seu próprio juízo,
sob a responsabilidade de seus contratadores. Em caso de violação das
regras, eles devem ser responsáveis diante da lei, e os próprios
regulamentos devem emanar da legislatura, a autoridade administrativa
central apenas supervisionando a sua execução, e se eles não forem
aplicados, dependendo da natureza do caso, apelar ao tribunal para que
a lei seja cumprida, ou para os contratadores para que demitam os
funcionários que não a executaram de acordo com seu espírito. Na sua
concepção geral, esta é a superintendência central que se espera que
o Comitê da Lei dos Pobres exerça sobre os administradores do Imposto para os Pobres neste
país. Quaisquer poderes que o Comitê exerça além desse limite serão
corretos e necessários em um caso específico, para a correção de hábitos de
má administração profundamente enfronhados em assuntos que afetam
não apenas as localidades, mas toda a comunidade; pois
nenhuma localidade tem o direito de fazer de si mesma, por má
administração, um ninho de pobreza, que necessariamente fluirá para
outras localidades, e porá em risco a condição física e moral de toda
a comunidade trabalhadora. Os poderes da coerção administrativa e da
legislação subordinada que possui o Comitê da Lei dos Pobres (mas que,
devido ao estado de opinião nesse assunto, raramente são exercidos),
apesar de perfeitamente justificáveis em um caso de interesse nacional,
estariam completamente fora de lugar numa superintendência de interesses
puramente locais. Mas um órgão central de informação e instrução para
todas as localidades seria igualmente valioso para todos os
departamentos da administração. Para um governo, tal atividade
nunca será excessiva, pois ela não impede, mas ajuda e estimula a atividade
individual e o seu desenvolvimento. O malefício começa quando, ao invés
de fazer surgir atividades e poderes de indivíduos e grupos de pessoas,
o governo substitui a atividade deles pela sua própria, quando,
em vez de informar, aconselhar e, dependendo da ocasião, denunciar, ele
faz as pessoas trabalharem acorrentadas, ou as deixa de lado, e faz
o trabalho no lugar delas. O valor de um Estado, a longo prazo, é o
valor dos indivíduos que o compõem, e um Estado que adia o interesse
que seus integrantes têm na expansão e elevação mental, em um pouco
mais de capacidade administrativa, ou na coisa semelhante a essa última
que a prática fornece, nos detalhes dos negócios, um Estado que diminui
seus homens, para que estes sejam um instrumento mais dócil, mesmo que
seja com bons propósitos --- descobrirá que com homens pequenos nada de
grande pode ser alcançado, e que a perfeição da maquinaria para a qual
ele tudo sacrificou, no final não servirá para nada, por falta do poder
vital que, para que a máquina pudesse funcionar sem percalços, 
o Estado preferiu banir. 


%\part{APENDICE}
%\printindex

\SVN $Id: FINAIS.tex 10297 2011-11-28 15:08:28Z oliveira $

\SVN $Id: PUBLICIDADE.tex 7302 2010-08-25 14:36:09Z iuri $ 

\ifodd\thepage\paginabranca\else\clearpage\fi
\pagestyle{empty}

{\textsc{coleção de bolso hedra}
\begin{enumerate}
\setlength\itemsep{-1.4mm}
\fontsize{5}{6}\selectfont
\item \textit{Iracema}, Alencar
\item \textit{Don Juan}, Molière
\item \textit{Contos indianos}, Mallarmé
\item \textit{Auto da barca do Inferno}, Gil Vicente
\item \textit{Poemas completos de Alberto Caeiro}, Pessoa
\item \textit{Triunfos}, Petrarca
\item \textit{A cidade e as serras}, Eça
\item \textit{O retrato de Dorian Gray}, Wilde
\item \textit{A história trágica do Doutor Fausto}, Marlowe
\item \textit{Os sofrimentos do jovem Werther}, Goethe
\item \textit{Dos novos sistemas na arte}, Maliévitch
\item \textit{Mensagem}, Pessoa
\item \textit{Metamorfoses}, Ovídio
\item \textit{Micromegas e outros contos}, Voltaire
\item \textit{O sobrinho de Rameau}, Diderot
\item \textit{Carta sobre a tolerância}, Locke
\item \textit{Discursos ímpios}, Sade
\item \textit{O príncipe}, Maquiavel
\item \textit{Dao De Jing}, Laozi
\item \textit{O fim do ciúme e outros contos}, Proust
\item \textit{Pequenos poemas em prosa}, Baudelaire
\item \textit{Fé e saber}, Hegel
\item \textit{Joana d'Arc}, Michelet
\item \textit{Livro dos mandamentos: 248 preceitos positivos}, Maimônides
\item \mbox{\textit{O indivíduo, a sociedade e o Estado, e outros ensaios}, 
		Emma Goldman}
\item \textit{Eu acuso!}, Zola | \textit{O processo do capitão Dreyfus}, Rui Barbosa
\item \textit{Apologia de Galileu}, Campanella 
\item \textit{Sobre verdade e mentira}, Nietzsche
\item \textit{O princípio anarquista e outros ensaios}, Kropotkin
\item \textit{Os sovietes traídos pelos bolcheviques}, Rocker
\item \textit{Poemas}, Byron
\item \textit{Sonetos}, Shakespeare
\item \textit{A vida é sonho}, Calderón
\item \textit{Escritos revolucionários}, Malatesta
\item \textit{Sagas}, Strindberg
\item \textit{O mundo ou tratado da luz}, Descartes
\item \textit{O Ateneu}, Raul Pompeia
\item \textit{Fábula de Polifemo e Galateia e outros poemas}, Góngora
\item \textit{A vênus das peles}, Sacher{}-Masoch
\item \textit{Escritos sobre arte}, Baudelaire
\item \textit{Cântico dos cânticos}, [Salomão]
\item \textit{Americanismo e fordismo}, Gramsci
\item \textit{O princípio do Estado e outros ensaios}, Bakunin
\item \textit{O gato preto e outros contos}, Poe
\item \textit{História da província Santa Cruz}, Gandavo
\item \textit{Balada dos enforcados e outros poemas}, Villon
\item \textit{Sátiras, fábulas, aforismos e profecias}, Da Vinci
\item \textit{O cego e outros contos}, D.H.~Lawrence
\item \textit{Rashômon e outros contos}, Akutagawa
\item \textit{História da anarquia (vol.~1)}, Max Nettlau
\item \textit{Imitação de Cristo}, Tomás de Kempis
\item \textit{O casamento do Céu e do Inferno}, Blake
\item \textit{Cartas a favor da escravidão}, Alencar
\item \textit{Utopia Brasil}, Darcy Ribeiro
\item \textit{Flossie, a Vênus de quinze anos}, [Swinburne]
\item \textit{Teleny, ou o reverso da medalha}, [Wilde et al.]
\item \textit{A filosofia na era trágica dos gregos}, Nietzsche
\item \textit{No coração das trevas}, Conrad
\item \textit{Viagem sentimental}, Sterne
\item \textit{Arcana C\oe lestia} e \textit{Apocalipsis revelata}, Swedenborg
\item \textit{Saga dos Volsungos}, Anônimo do séc.~\textsc{xiii}
\item \textit{Um anarquista e outros contos}, Conrad
\item \textit{A monadologia e outros textos}, Leibniz
\item \textit{Cultura estética e liberdade}, Schiller
\item \textit{A pele do lobo e outras peças}, Artur Azevedo
\item \textit{Poesia basca: das origens à Guerra Civil} 
\item \textit{Poesia catalã: das origens à Guerra Civil} 
\item \textit{Poesia espanhola: das origens à Guerra Civil} 
\item \textit{Poesia galega: das origens à Guerra Civil} 
\item \textit{O chamado de Cthulhu e outros contos}, H.P.~Lovecraft 
\item \textit{O pequeno Zacarias, chamado Cinábrio}, E.T.A.~Hoffmann
\item \textit{Tratados da terra e gente do Brasil}, Fernão Cardim 
\item \textit{Entre camponeses}, Malatesta
\item \textit{O Rabi de Bacherach}, Heine
\item \textit{Bom Crioulo}, Adolfo Caminha
\item \textit{Um gato indiscreto e outros contos}, Saki
\item \textit{Viagem em volta do meu quarto}, Xavier de Maistre 
\item \textit{Hawthorne e seus musgos}, Melville
\item \textit{A metamorfose}, Kafka
\item \textit{Ode ao Vento Oeste e outros poemas}, Shelley
\item \textit{Oração aos moços}, Rui Barbosa
\item \textit{Feitiço de amor e outros contos}, Ludwig Tieck
\item \textit{O corno de si próprio e outros contos}, Sade
\item \textit{Investigação sobre o entendimento humano}, Hume
\item \textit{Sobre os sonhos e outros diálogos}, Borges | Osvaldo Ferrari
\item \textit{Sobre a filosofia e outros diálogos}, Borges | Osvaldo Ferrari
\item \textit{Sobre a amizade e outros diálogos}, Borges | Osvaldo Ferrari
\item \textit{A voz dos botequins e outros poemas}, Verlaine 
\item \textit{Gente de Hemsö}, Strindberg 
\item \textit{Senhorita Júlia e outras peças}, Strindberg 
\item \textit{Correspondência}, Goethe | Schiller
\item \textit{Índice das coisas mais notáveis}, Vieira
\item \textit{Tratado descritivo do Brasil em 1587}, Gabriel Soares de Sousa
\item \textit{Poemas da cabana montanhesa}, Saigy\=o
\item \textit{Autobiografia de uma pulga}, [Stanislas de Rhodes]
\item \textit{A volta do parafuso}, Henry James
\item \textit{Ode sobre a melancolia e outros poemas}, Keats 
\item \textit{Teatro de êxtase}, Pessoa
\item \textit{Carmilla -- A vampira de Karnstein}, Sheridan Le Fanu
\item \textit{Pensamento político de Maquiavel}, Fichte
\item \textit{Inferno}, Strindberg
\item \textit{Contos clássicos de vampiro}, Byron, Stoker e outros
\item \textit{O primeiro Hamlet}, Shakespeare
\item \textit{Noites egípcias e outros contos}, Púchkin
\item \textit{A carteira de meu tio}, Macedo
\item \textit{O desertor}, Silva Alvarenga
\item \textit{Jerusalém}, Blake
\item \textit{As bacantes}, Eurípides
\item \textit{Emília Galotti}, Lessing
\item \textit{Contos húngaros}, Kosztolányi, Karinthy, Csáth e Krúdy
\item \textit{A sombra de Innsmouth}, H.P.~Lovecraft
\item \textit{Viagem aos Estados Unidos}, Tocqueville
\item \textit{Sobre a filosofia e seu método}, Schopenhauer
\item \textit{Émile e Sophie ou os solitários}, Rousseau 
\item \textit{Manifesto comunista}, Marx e Engels
\item \textit{A fábrica de robôs}, Karel Tchápek 
\item \textit{Sobre a filosofia e seu método (Parerga e paralipomena)}, Schopenhauer 
\item \textit{O novo Epicuro: as delícias do sexo}, Edward Sellon
\item \textit{Sobre a liberdade}, Mill
\vfill
\end{enumerate}
}%


\pagebreak

%\begin{blackpages}
\thispagestyle{empty}
\begin{techpage}{42mm}
		\putline{Edição}{Iuri Pereira}
		\putline{Coedição}{Jorge Sallum e Oliver Tolle}
		\putline{Preparação}{Oliver Tolle e Iuri Pereira}
		\putline{Revisão}{Bruno Oliveira}
		\putline{Capa e projeto gráfico}{Júlio Dui e Renan Costa Lima}
		\putline{Imagem de capa}{David Dennis, ``Rubber Ducks with Sunglasses'' (2004)}
		\putline{Programação em LaTeX}{Marcelo Freitas}
		\putline{Assistência editorial}{Bruno Oliveira}
% 		\putline{Preparação}{Oliver Tolle}          
		\putline{Colofão}{Adverte-se aos curiosos que se
			imprimiu esta obra em nossas oficinas em \today, em papel 
			\mbox{off-set} 90~g/m²,
			composta em tipologia Minion Pro, 
			em \textsc{gnu}/Linux (Gentoo, Sabayon e Ubuntu), 
			com os softwares livres 
			\LaTeX, De\TeX, \textsc{vim}, Evince, Pdftk, 
			Aspell, \textsc{svn} e \textsc{trac}.}

\end{techpage}
%\end{blackpages}


\ifdefined\printcheck\printcheck\fi

\end{document}
