
\chapter[Introdução, por Ari R. Tank Brito]{introdução}
\hedramarkboth{Introdução}{Ari R. Tank Brito}

\noindent\textsc{O filósofo} britânico John Stuart Mill escreveu
extensamente sobre todos os assuntos tidos como importantes e ganhou em
vida a reputação de sábio. Também foi portador de uma
personalidade excepcional, austera e justa, sendo chamado de ``Santo do
Utilitarismo'' por um importante e sagaz político britânico da época. O
lugar que Mill obteve no panteão dos sábios e dos santos laicos, que
isso fique claro desde o início, é absolutamente merecido. Ele foi um
dos principais filósofos de sua época e um dos maiores pensadores
liberais de todos os tempos. E muito embora a sua vida pessoal esteja
repleta de peculiaridades, hoje em dia dificilmente compreensíveis,
pode"-se perceber nela um comprometimento com a causa da liberdade
humana que poucos jamais tiveram, seja antes ou depois dele. 

Talvez a chave desse comprometimento de John Mill resida na sua
formação, uma vez que foi educado para ser um gênio desde a mais tenra
infância, no que teve sucesso, embora à custa de muito sofrimento
pessoal. Todo o processo teve lugar em seu próprio lar: seu pai, o
também filósofo James Mill (1773--1836), assumiu a sua educação
pessoalmente, seguindo um rígido processo de aprendizado. Aos dez anos,
Mill já lia e escrevia em grego antigo e em latim, estudando obras que
mesmo adultos achariam difíceis e complicadas, como, por exemplo, os diálogos de
Platão. Ao estudo de autores antigos acrescente"-se o estudo de
matemática, direito, história, lógica, economia, ciências, além de uma
bela dose de literatura. Não se pode afirmar que John Mill teve uma
infância feliz e alegre, e que simplesmente não teve infância nenhuma
talvez não seja uma afirmação exagerada. Não bastasse ter de seguir o
estafante esquema de seu pai, com aulas de manhã até a noite, tão logo
John Stuart Mill chegou à juventude, um outro preceptor lhe foi dado, o
fundador do Utilitarismo, Jeremy Bentham (1748--1832), amigo de seu
pai. Esses dois, James Mill e Bentham, fizeram o melhor que puderam
para transformar o menino Mill em perfeito utilitarista. Durante um bom
tempo pareceu que tinham conseguido uma proeza: a criação, através de um
rígido esquema pedagógico, de uma mente inteligente, fria e racional.
Nada foi esquecido, no que dizia respeito a inculcar na cabeça de John
Mill não apenas os conteúdos programáticos da doutrina utilitarista,
mas também um imenso cabedal de conhecimentos que facilitasse a defesa
dessa doutrina. Aos treze anos, Mill estava de posse de uma instrução
universitária completa de sua época, com a diferença de que não tinha
sido imposta nenhuma fé religiosa, pois foi educado para ser um
ateu, assim como eram seus preceptores. 

A criatura não decepcionou por completo os criadores: durante toda a sua
vida John Mill continuou sendo um utilitarista. O que aconteceu é que a
versão do utilitarismo que lhe foi inculcada acabou por
decepcioná"-lo. Isso resultou numa revolta sentimental e intelectual,
seguida de uma profunda busca por outras bases filosóficas para o
utilitarismo, diferentes daquelas propostas por seu pai James Mill e
por seu preceptor Jeremy Bentham. 

O princípio básico do utilitarismo é que ``a maior felicidade para
o maior número (de pessoas) é o fundamento da moral e da legislação'',
princípio que Bentham generosa e honestamente reputou a escritores
anteriores como o jurista italiano Beccaria e o homem de ciências
britânico Priestley, mas que apenas com ele, Bentham, tornou"-se
absolutamente relevante. Este assim chamado Grande Princípio da
Felicidade, ou o Princípio da Utilidade, formulado em sua obra
\textit{Princípios da moral e da legislação}, de 1789, constituiu o
ponto de partida sobre o qual ergueu"-se todo um sistema de normas e
diretrizes, tidas como racionais e que, se seguidas, deveriam levar a
um bom sistema político, a uma boa vida social, além de resultados
correlatos, como bons asilos para os pobres e boas prisões para os
malfeitores. Pois, para Bentham, a sua filosofia utilitarista era um
projeto de reforma de caráter universal, sendo o próprio Bentham o
legislador de que o mundo precisava para ser posto em eixos racionais.
E ele bem que tentou passar adiante seus extensos planos de reformas,
se esforçando para que os governantes do seu e de outros países
pusessem em prática as ideias utilitaristas. Bentham sequer chegou
perto de ter o sucesso prático que almejava, mas no campo do Direito
exerceu uma influência até hoje sentida. Não apenas na forma de se
entender o que vem a ser o Direito, a Lei, mas também na prática, em
forma de legislação, particularmente a penal, o que não deixa de ser um
sucesso para alguém que tinha como profissão a advocacia. Mas é
irônico que o defensor de um princípio à primeira vista tão
liberal como o do utilitarismo se tornasse conhecido principalmente por
propor um tipo de edifício em que tudo o que viesse a acontecer
estivesse sob a vigilância de guardas. O \textit{Panopticon} (``onde
tudo é visto'') proposto por Bentham foi primeiro percebido como um
prédio especialmente construído para vigiar incessantemente seus
ocupantes, isto é, os prisioneiros, sem que estes pudessem ter um
instante de privacidade. A arquitetura peculiar desse edifício, e
deve"-se anotar que penitenciárias foram de fato construídas segundo
as prescrições de Bentham, com sua torre central e suas celas
direcionadas para esta torre, não é tão interessante como a sua ideia
central: a de que os seres humanos devem ser vigiados o tempo todo, sob
pena de se rebelarem ou não trabalharem a contento. A intenção de
Bentham era, pelas suas luzes, humanitária. Vigiados, os prisioneiros
não se rebelariam, o que sempre acarretava perdas de vidas e
propriedades, e trabalhariam com afinco, escapando assim das garras da
preguiça, a principal causa de seus erros e de sua miserável situação.
Vigiados para não cometer tolices e obrigados a trabalhar, os marginais
(bandidos e pobres, pois o Panopticon deveria servir tanto para cadeias
como para os lugares onde os moradores de rua de então eram deslocados)
deixariam de ser perniciosos para a sociedade. Sendo assim, a
felicidade geral aumentaria consideravelmente. Pois, numa outra
formulação do Princípio da Utilidade, encontrada em uma obra de
Bentham publicada em 1776, ``a felicidade do maior número 
é a medida do certo e do errado''. 

Infelizmente para Bentham e os defensores do Panopticon, a realidade
revelou"-se bem outra: o controle total prometido, ao ser aplicado,
tornava as coisas ainda piores, já que os prisioneiros, sem privacidade
nenhuma em suas vidas, apelavam para ações de resistência passiva para
as quais seus vigias não estavam preparados, isto é, deixavam de
cumprir o que lhes era exigido. Também as rebeliões não deixaram de
acontecer, não obstante toda a vigilância. O Panopticon, tal como
proposto por Bentham, não é útil. Mesmo que prisioneiras, as pessoas
devem ter certo grau de privacidade, de liberdade mesmo. Mas o que
levou Bentham a propor um esquema assim? 

O utilitarismo que surge nos finais do século \textsc{xviii}, o Século das Luzes,
mantém e intensifica características das ideias de \textit{philosophes}
como Voltaire, Diderot e D'Alembert. Ele é racional, antirreligioso,
cientificista e contrário aos poderes constituídos. Duas
especificidades utilitaristas são fundamentais para uma boa compreensão
da obra \textit{Sobre a liberdade} de Mill: a posição tomada contra os
direitos naturais, que separa essa obra dos movimentos que levaram à
Independência dos Estados Unidos e à Revolução Francesa, e o método que
levaria a maior felicidade ao maior número de pessoas, o ``Cálculo
Felicífico''. Ser contra os direitos naturais significava ser contra a
suposição de que os direitos humanos existissem anteriormente a
qualquer Lei que os tivesse criado. Não havendo Deus, como propunham os
utilitaristas, e não dando a Natureza nenhum direito aos homens, os
direitos humanos só poderiam existir se uma lei positiva, feita pelos
homens dentro de um contexto legal, assim o proclamasse. A Declaração
de Independência dos Estados Unidos, que contém a célebre frase
``Mantemos ser autoevidente que todos os homens são criados
iguais, e que são agraciados pelo seu Criador com certos direitos
inalienáveis, entre os quais estão o direito à vida, à liberdade e à
busca da felicidade'', e a Declaração dos Direitos do Homem proclamada
pelos revolucionários franceses, não passavam, para Bentham, de exemplos
do pomposo absurdo que era a doutrina dos direitos humanos naturais. O
que significa, entre outras coisas, que Mill não poderia, para defender
a causa da liberdade, utilizar nenhum argumento que se
relacionasse com a suposição de que a liberdade é um direito natural do
ser humano. Isso não poderia ser feito sem abandonar o utilitarismo
como um todo. Os argumentos em \textit{Sobre a liberdade} têm e só
podem ter um caráter utilitarista. O que foi conseguido com a
introdução de mudanças de monta no segundo aspecto acima mencionado, o
do Cálculo Felicífico. Como medir a felicidade? Para Bentham, a questão
se limitava a fugir da dor e se aproximar do prazer, de forma
quantitativa. Mais dor, mais infelicidade. John Mill, ao contrário,
praticamente liquida com a possibilidade de se calcular matematicamente
o índice de felicidade de uma pessoa, com a introdução do aspecto
qualitativo dessa felicidade. Não é só importante o quanto uma pessoa é
feliz, o quanto ela está afastada das dores e próxima dos prazeres, mas,
e principalmente, como esta felicidade está construída, isto é, se ela é
de qualidade ou não. Afinal, como colocou John Mill, ``é melhor
ser um Sócrates insatisfeito do que um homem satisfeito''. Não se trata
de poder calcular o quanto de felicidade um homem satisfeito teria a
mais que um filósofo insatisfeito, mas sim de declarar que a posição do
segundo é melhor, isto é, qualitativamente superior à do primeiro. O
que vem a significar que muitas vezes é melhor ser infeliz do que
feliz, uma proposição que não teria sentido para um utilitarista ortodoxo. 

A introdução de critérios qualitativos para aferir não se alguém é mais
ou menos feliz, mas sim se vive uma vida melhor ou pior, poderia levar
a crer que Mill tivesse tornado o utilitarismo inconsistente, em
desacordo consigo próprio, e não faltaram ou faltam críticos para
apontar as possíveis inconsistências que apareceram no tipo de
utilitarismo proposto por Mill. Se esses críticos têm ou não razão é
uma questão espinhosa, e talvez insolúvel. Cabe lembrar apenas que
para Mill, autor de uma \textit{Lógica} que fez muito sucesso, a sua
posição não era nem ilógica nem incompatível com os princípios
utilitaristas. Pelo contrário, considerava as alterações necessárias
para que o utilitarismo pudesse dar conta dos problemas que tratava. 

Como se viu anteriormente, ao defender amplas reformas na legislação, o
utilitarismo envolveu"-se na política. Passada a época em que Jeremy
Bentham ainda acreditava que as classes governantes aceitariam ser os
responsáveis por essas reformas, os utilitaristas se tornaram um grupo
político específico, independente tanto dos Tories (conservadores)
quanto dos Whigs (liberais). Eles denominaram a si mesmos de Radicais
Filosóficos; e era como um Radical que John Mill via a si mesmo, e como
Radical ele foi durante algum tempo membro do Parlamento britânico. As
palavras mudam de sentido, o que não deixou de acontecer com a palavra
``radical'': longe de querer implicar mudanças de caráter socialista, o
que os utilitaristas propunham com radical significou na prática
reformas no sistema capitalista de então para torná"-lo mais racional
e eficiente. Embora essas propostas tivessem um viés democrático, como, 
por exemplo, propor que a cada homem correspondesse um voto, isso
quando na Grã"-Bretanha de então a população simplesmente não tinha
direito de voto, elas carregavam consigo um pendor antipopular. Os
utilitaristas, os Radicais, se propunham ser a elite intelectual de que,
segundo eles, a nação precisava, sem que de fato atentassem ao país
como esse realmente era. Tudo o que não era racional e utilitário era
tido como antiquado e obviamente destinado a ser destruído e
substituído por algo mais científico. Pode"-se dizer que os Radicais
queriam fazer a parte que achavam ter sido a dos Girondinos na
Revolução Francesa, sem promover a efusão de sangue trazida pelo Terror
jacobino. E isso sem se ligar a ``falácias anárquicas'', como
consideravam, por exemplo, os direitos humanos. O que também
significava não se importar com os costumes, os modos de vida e as
crenças religiosas da população. A reforma viria do alto, à massa
cabendo apenas seguir os líderes e aprender a se comportar. 

No entanto, nem a população em geral nem a maior parte das classes
dominantes se dispuseram a seguir completamente o receituário
utilitarista. Os que mandavam, os Lordes e os grandes capitalistas, não
trocaram os seus interesses de classe pelo seu ``interesse bem
compreendido'', preferindo o mais das vezes ou ser contra qualquer
reforma ou a favor de reformas paulatinas. Não se sentiam inclinados a
apoiar algo tão abstrato e incomum como as propostas radicais. Já o
povo, de modo geral, passou a ter medo dos utilitaristas, medo que se
percebe claramente expresso no romance de Charles Dickens
\textit{Tempos difíceis} (\textit{Hard Times}, 1854). Neste romance, a figura tétrica de um
utilitarista, o senhor Thomas Gradgrind --- ``Um homem de
realidades. Um homem de fatos e cálculos. Um homem que age sob o
princípio de que dois e dois são quatro, e nada mais [\ldots]'' ---, é utilizada por
Dickens para mostrar como o utilitarismo chegava à vida das pessoas:
sem coração e sem alma, já que o utilitarista só conseguia calcular, e
neste cálculo punha por terra o que a vida tinha de bom. Amor,
sentimentos, poesia, eram deixados de lado, substituídos por um infame
calcular de lucros e perdas, que se aplicava tanto à vida familiar como
às fábricas, onde os trabalhadores eram submetidos a rígidos e
extenuantes horários de serviço e a cotas de produção desumanas. Neste
mesmo romance, aliás, a educação dos filhos do senhor Gradgrind é uma
paródia da educação a que John Mill foi submetido, com ênfase na
memorização mecânica de dados exatos sobre as coisas. Talvez os
utilitaristas tivessem boas intenções, mas o julgamento que se fazia a
seu respeito era que queriam pavimentar o caminho para o inferno. 

Para John Mill, certa inexperiência nas coisas da vida, tanto de 
Bentham quanto de seu pai, James Mill, foi responsável diretamente 
por este mal"-estar e pelas propostas inexequíveis (mesmo em termos 
utilitaristas, já que tendiam a aumentar a infelicidade das pessoas)
que o geraram. Bentham, principalmente, era para John Stuart Mill um homem
de ideias claras, mas pouco amplas, já que para ter ideias claras sobre
assuntos complexos é necessário, como ele escreveu em um ensaio crítico
a respeito de seu ex"-professor (``Essay on Bentham'', 1838),
negar ``a completamente não analisável experiência da raça humana''.
Bentham, em suma, era um doutrinário, talvez mesmo um dogmático, e como
tal, via as coisas de modo parcial e incompleto.

No entanto, se Mill escreveu e publicou quando tinha 22 anos
uma crítica ao seu mestre Bentham, isso não quer dizer que 
ele tenha  deixado de ser um adepto do utilitarismo. Isso nunca ocorreu,
John Mill continuou a ser um utilitarista até o fim da sua vida. Ele
não se tornou um apóstata, mas sim um herético. Isto é, ele introduziu
mudanças significativas no que se conhecia como utilitarismo, mudanças essas
que, como se verá adiante, acabaram por tornar sua filosofia bem mais
complexa e bem menos \mbox{dogmática}. Essas mudanças não ocorreram apenas por
razões teóricas, mas por razões pessoais e profundas.

O que veio primeiro foi uma desilusão sentimental sobre o utilitarismo.
Quando muito jovem ainda, mas já seriamente empenhado em discutir e 
espalhar os princípios utilitaristas, Mill, conforme nos conta em sua
\textit{Autobiografia} (publicada postumamente em 1874), fez a si mesmo 
a seguinte pergunta: ``Suponha que todos os seus objetivos na vida foram realizados,
que todas as mudanças nas instituições e opiniões pelas quais você vem
se esforçando pudessem acontecer neste exato momento: Isso seria uma
grande alegria e felicidade para você?''. A resposta foi um triste
\textit{Não!}, e essa resposta, vinda de seu íntimo, mudou a sua vida.
Se a concretização dos ideais pelos quais lutava não o deixaria feliz,
nada mais poderia fazê"-lo. Com essa ``crise mental'', John Stuart Mill
se tornou então a pessoa mais triste do mundo, carregando o peso de ter
de se esforçar por coisas que, ele sabia, não iriam jamais
satisfazê"-lo. Não contou a seu pai ou seus amigos nada a respeito
dessa revelação ou constatação, que manteve por anos em segredo.

Um filósofo utilitarista que descobre que o utilitarismo, mesmo
funcionando, não o deixaria feliz, certamente tem um grande problema
nas mãos. A questão, para Mill, foi solucionada com um grande custo
pessoal. Exteriormente, ele continuou como antes, o jovem radical,
sempre disposto a estudar, a escrever e a polemizar, defendendo as
ideias básicas do utilitarismo. Interiormente, passou a viver sob uma
sombra, com os seus sofrimentos e dilemas dilacerando"-o aos poucos.
Já que não podia abandonar as ideias aprendidas, ele tomou o único
caminho possível, o de remodelá"-las. E o fez introduzindo suas
\mbox{angústias} e expectativas no inadequado programa dentro do qual e pelo
qual vinha lutando.

Foram vários os passos que John Mill teve que dar para sair de sua crise
mental, todos dolorosos. Intelectualmente, o mais difícil certamente
foi o de ter de admitir que alguns inimigos do utilitarismo não estavam
tão errados assim. A sua leitura do poeta e crítico romântico Coleridge
certamente é a mais sintomática --- Coleridge atacou as ideias de James
Mill por escrito, o que levou John Stuart Mill a defender seu pai
também por escrito. Contudo, as críticas de Coleridge várias vezes
acertaram o alvo, já que de fato o utilitarismo pecava por ser ingênuo,
inclusive simplório, quando tratava da vida social, da história, da
arte. John Mill estava preparado para reconhecer isso, já que ele
próprio era um apreciador da literatura romântica, a qual proporcionava
alívio para os seus sofrimentos. Como tudo aquilo que consola pode ter
alguma utilidade, já que nos afasta da dor, a poesia romântica (Jeremy
Bentham não apreciava nenhum tipo de poesia e James Mill apenas a poesia
greco"-latina) não podia ser descartada sem mais. 

Mas, como obrigatoriamente tinha de ser, quando se tratava de John
Stuart Mill, não era apenas o caso da poesia sentimental de cunho
romântico ter o dom de aliviar suas dores. O que os românticos pensavam
sobre a sociedade o fazia duvidar, refletir e tentar expor melhor suas
próprias ideias. Os românticos, de modo geral, eram o contrário dos
utilitaristas: sentimentais, tradicionalistas, intuicionistas e, muitas
vezes, com pendores revolucionários. O que John Mill aprendeu com eles
foi a expressar ideias que já tinha consigo, mas que não conseguiria
formular claramente se usasse apenas as categorias utilitaristas. 

Tendo sofrido uma crise pessoal e intelectual, a saída possível de Mill
teria que tentar contemplar ambos os aspectos. No campo pessoal, o
encontro com uma mulher, Elizabeth Harriet Taylor, significou o ponto
de mudança tão esperado. Ela sendo casada, John Mill teve de esperar
anos até que ela enviuvasse e eles pudessem se casar, o que parece
uma história banal, até que se observe que Mill e Harriet jamais
foram amantes e que a importância dela na sua vida não foi
estritamente sentimental, mas também teve um caráter filosófico e
político. Harriet era uma radical também, mas de cunho religioso,
enquanto Mill era um ateu convicto, mas essa diferença não impediu que
Mill se sentisse impelido a acompanhar Harriet em empreendimentos
reformistas. Mill teria sido menos ousado em suas ideias sem o
incentivo da amiga e depois esposa. E nunca deixou de reconhecer a
importância que ela teve na sua vida intelectual, como demonstra a
dedicatória de \textit{Sobre a liberdade}, a última obra de Mill, que
contou com a colaboração de Harriet. Há aqueles que alegam que a
influência de Harriet foi deletéria para o pensamento de Mill, que as
ideias utilitaristas dele não se harmonizaram bem com as ideias religiosas e
democráticas dela, mas esse julgamento é mais que falho, é tolo. Mesmo
que se admita que as ideias de Mill se tornaram inconsistentes porque
ele teve de lidar com aspectos que o utilitarismo clássico não
comportava, além de incorporá"-los em sua obra, a abertura
proporcionada ao pensamento de Mill, a amplitude que ele pôde
desenvolver nas suas ideias, seria mais que suficiente para compensar
alguma inconsistência baseada em parâmetros pequenos e demasiado
rígidos. É o que se percebe lendo \textit{Sobre a liberdade}, a sua
obra mais influente. Se a ortodoxia do utilitarismo de Mill pode ser
posta em dúvida, o resultado dessa possível heterodoxia é nitidamente
superior a qualquer coisa que outros utilitaristas escreveram
anteriormente sobre a política e a sociedade, como se pode perceber à
medida em que a obra é lida. 

Logo no primeiro capítulo, Mill
apresenta o seu famoso princípio da liberdade: 

\begin{hedraquote}
que o único fim
pelo qual se permite que a humanidade, coletiva ou individualmente,
interfira com a liberdade de ação de qualquer um dos seus números é a
autoproteção. Que o único propósito pelo qual o poder pode ser exercido
de forma justa sobre qualquer membro de uma comunidade civilizada,
contra a vontade dele, é o de prevenir danos aos outros.
\end{hedraquote}

Este princípio, longe de ser simples e facilmente inteligível, foi motivo de
discussões acirradas, grande parte das quais girando sobre a sua
coerência. Mas nem sempre se dá atenção ao detalhe de que este
princípio, tal como formulado, é um princípio de restrição da
liberdade, já que Mill trata da liberdade civil ou social e ``a
natureza e limites do poder que pode ser legitimamente exercido pela
sociedade sobre o indivíduo''. O princípio mesmo se divide em dois
aspectos, um em relação à humanidade como um todo, e o segundo em
relação a uma sociedade já organizada. Tanto no primeiro como no
segundo caso, vale a mesma restrição, mas no segundo trata"-se
da aplicação de um poder legal, enquanto no primeiro as condições de
aplicação não estão bem definidas, valendo para todas as sociedades
humanas. Vale também para ambos os casos a mesma pergunta, sempre
repetida quando se trata de discutir as bases e a amplitude deste
princípio: o que significa ``autoproteção'' e ``prevenir danos às outras
pessoas''? Num primeiro momento, trata"-se do direito à vida, o direito
à autopreservação e à preservação da vida de outrem, o exemplo
clássico. Mas é claro que é preciso ir muito mais além: Mill continua o
seu texto delineando os casos em que o seu princípio se aplicaria ou
não. Por exemplo, até quanto pode um pai de família gastar a sua renda
em bebidas? Não pode gastar muito, se esse gasto prejudicar o
bem"-estar dos seus. Um homem solteiro e sem ligações, ele já pode
gastar o que quiser, já que o único prejudicado é ele mesmo. A única
pena que Mill aceita neste caso é a reprovação moral da sociedade. De
exemplos assim saíram muitas discussões inflamadas, girando em torno da
candente questão sobre se esses e outros exemplos ampliariam ou não o grau de
liberdade de uma sociedade. Como se Mill estivesse se referindo a uma
futura sociedade permissiva, quando homens e mulheres viveriam ``cada um
na sua'', e não à moralista, religiosa e intransigente sociedade
vitoriana. E como se a reprovação moral fosse algo que se pudesse
ignorar com um dar de ombros, debaixo da proteção de leis que garantam
as diferenças de ``experiências de viver'', justamente as garantias que
não existiam e que Mill queria implementar. A questão da liberdade
seria a questão vital do futuro, não por ser uma novidade, mas pelas
condições já alcançadas pelas ``mais civilizadas partes'' do globo. 

Essa liberdade que não podia, exceto em circunstâncias bem definidas,
ser afetada, era uma constante histórica oriunda da Grécia e Roma
antigas e que chegava até a Inglaterra. A menção conjunta à Inglaterra,
Roma e Grécia não é ocasional, pois indica o caminho da liberdade para
Mill. No mundo antigo, liberdade era entendida como ``proteção
contra a tirania dos governantes''. Não se trata de uma diferença entre
a liberdade dos antigos e dos modernos, mas da mesma liberdade, a
liberdade de não ser oprimido. Mill mostra o caminho que foi percorrido
para controlar o poder dos governantes de oprimir os governados,
primeiro pelo reconhecimento de certas imunidades, as liberdades ou
direitos políticos, e pelo posterior estabelecimento de controles
constitucionais. O primeiro passo acabou sendo aceito pela maioria dos
países europeus, mas não o segundo, que se tornou em toda parte o
``principal interesse dos amantes da liberdade''. Como
essa segunda etapa já começa a ser implantada e fortalecida pelo menos
nas nações mais avançadas, o problema de não ser oprimido pelos
governantes deixa de ser tão importante quanto o era anteriormente.
Cresce, porém, outro tipo de controle, o controle das sociedades sobre
os seus membros, que numa sociedade composta por muitos membros
caracteriza"-se por ser uma tirania da maioria, ``quando a
sociedade é ela mesma a sociedade"-tirana, coletivamente, sobre os
indivíduos que a compõem''. Se antes essa tirania era exercida por meio
dos atos das autoridades públicas, com o controle dessas pelas saídas
constitucionais, que fazem os governantes responsáveis diante da
comunidade (isto é, podendo ser punidos se exercerem mal o poder a eles
confiado), agora ela se exerce diretamente pela própria sociedade,
mesmo que ao arrepio da lei. Apenas a proteção diante do magistrado não
basta: 

\begin{hedraquote}
Precisa"-se também de proteção contra a tirania da
opinião e sentimento prevalecentes, contra a tendência da sociedade em
impor, por outros meios que as penas civis, as suas ideias e práticas
próprias como regras de conduta sobre aqueles que divirjam delas.
\end{hedraquote}

A proteção à liberdade é primordial, seja no campo social ou no campo do
conhecimento: só a manutenção da liberdade garante o aumento do
conhecimento. Este, em si, nunca está completamente garantido, mas
sempre há possibilidades de melhorá"-lo, tornando"-o mais acurado,
desde que haja liberdade de discussão e pesquisa. Na medida em que o
grau de conhecimento científico aumenta, mais e mais conclusões são
tidas como corretas e tiradas do terreno da discussão. Mas esse aumento
de certezas depende inicialmente da liberdade de defender pontos de
vista divergentes. Mill vai bem longe na defesa da liberdade de
opinião, pois se uma pessoa tem uma opinião, seja ela falsa ou
verdadeira, silenciá"-la é ruim, muito mais para aqueles contrários a
tal opinião do que para aqueles que a defendem.

O que já foi exposto nos permite compreender por que o já falecido
filósofo liberal britânico Isaiah Berlin (1909--1997), 
em seus \textit{Quatros ensaios sobre a liberdade}, considerou John
Stuart Mill como o grande filósofo da liberdade e, em termos mais
práticos, ``o mais apaixonado e o mais famoso defensor na
Inglaterra dos humilhados e oprimidos'', o defensor, pela vida toda,
``dos heréticos, dos apóstatas, dos blasfemos, da liberdade e do
perdão''. No que talvez seja uma colocação melhor, pode"-se afirmar
que, não importando o assunto em questão, Mill sempre entendeu o tema
da liberdade como extremamente relevante a esse assunto. Que dessa
importância dada à liberdade se tire, como Berlin, a conclusão de que
nas propostas e nos atos de John Mill o que de fato importava era única
e somente a liberdade e a justiça (a qualquer preço), e não a
capacidade de ser útil, não é, porém, nenhum exagero. Mas mesmo Berlin
tem de admitir que, para Mill, a liberdade não era um fim, mas sim um
meio. Como está colocado em \textit{Sobre a liberdade}, o objetivo da
\mbox{liberdade}, a sua razão de ser, a sua utilidade, é o aumento do
bem"-estar da humanidade. A liberdade constitui"-se na única e
infalível fonte do aprimoramento da humanidade. O avanço desta ocorre
quando o ``espírito da liberdade'' ou ``do progresso ou melhoria'', como é
chamado dependendo das circunstâncias, prevalece sobre o ``despotismo do
costume''. Nem sempre o espírito da melhoria é igual ao espírito da
liberdade, como explica Mill, já que o progresso pode ser imposto a um
povo que não o deseja. Neste caso, o espírito da liberdade pode se
aliar provisoriamente aos opositores do progresso. Esta concessão de
Mill em relação a um espírito de liberdade retrógrado não parece
significar muito, se é que de fato não passa de uma incoerência. Para
Mill, o progresso só ocorre pela obra de uma elite intelectual, que
carrega consigo o espírito da liberdade, em confronto com o conformismo
(ou revolta anárquica) das massas. A proposição de que uma elite
realize o desenvolvimento – o qual exige a livre discussão de ideias –
através de métodos despóticos pode ser creditada não a algum pessimismo
escondido dentro do utilitarismo milliano, mas sim a seu otimismo, que
o levava a considerar que o pior já tinha passado, pelo menos nos
países mais civilizados. O problema estaria em que, como o progresso
não se interrompe, ao contrário do que temia ou desejava Mill, uma
argumentação semelhante sobre o papel progressista do despotismo
poderia ser e realmente foi utilizada para justificar regimes
ditatoriais que surgiram depois da época vitoriana, muitos deles na
Europa já ``civilizada''. A tal tirania da maioria, tão temida, nem
sempre foi a responsável pelos horrores que se seguiram à
democratização (que Mill apontava como a mais forte tendência do mundo
moderno, mesmo que não fosse acompanhada pela implementação de
instituições políticas populares). 

Em \textit{Sobre a liberdade}, a liberdade que vale é a
liberdade individual, de pessoas conscientes, adultas e bem educadas.
São elas, seus gostos, seus modos de vida e suas ideias que devem ser
protegidos, tanto das ameaças que vêm de cima como de baixo. Que
Mill esteja também falando em causa própria é uma conclusão óbvia, mas
ela não esgota a questão. A proteção à liberdade, como já foi mostrado,
implica a continuidade do desenvolvimento e, portanto, a possibilidade
de ampliar não só as liberdades, mas o usufruto das benesses trazidas
pelo progresso à maioria da população. E é por este caminho que o
civismo ou republicanismo de Mill faz a sua entrada. A defesa da
liberdade individual que Mill verte página após página, frase após
frase, está ligada a uma concepção de dever séria, dura, exigente, que
oferece pouca margem a liberalidades no viver. Este é o grande golpe de
mágica de Mill: ao defender justamente o que poderia parecer
indefensável, a liberdade de qualquer um fazer o que quiser, desde que
não prejudique os outros, Mill está de fato defendendo de serem
perseguidos não só os melhores, como abrindo a possibilidade de que as
outras pessoas possam também se autoaperfeiçoar. Esta defesa tem um
custo, já que muitos farão, com toda a certeza, mal uso desta
liberdade. Este é um preço que se tem de pagar. Mas há também lucros,
já que a humanidade é a maior ganhadora quando suporta que cada qual
viva como lhe apetece. 

Novamente, a importância da manutenção da liberdade torna"-se geral, de
interesse público. Liberdade de pensamento e de gosto, não de ação:
``Ninguém defende que as ações possam ser tão livres quanto as
opiniões''. Ações têm consequências que estão sujeitas ao controle e
repressão governamentais. A liberdade de expressão de opiniões também
não é irrestrita: uma coisa é afirmar, numa roda de amigos, que toda a
propriedade é um roubo. Bem outra é expressar a mesma opinião aos
brados, acompanhado de uma multidão enfurecida, diante da mansão de uma
pessoa rica. Falar em público é, de certa forma, agir ou levar à ação.
E as más consequências da ação ou fala de uma determinada pessoa não
são protegidas pelo princípio da liberdade, encontrando"-se além de
seus limites, já que as consequências dos atos devem ser imputadas
àquele que pratica a ação, e a sociedade pode e deve punir severamente
aquele que por seus atos prejudicar as outras pessoas. 

Mill não é nem um pouco dócil em relação à aplicação de penas merecidas,
mas esta severidade não deve ser entendida como mal colocada dentro de
sua filosofia política. Embora Mill seja duro com os que quebram as
leis, essas devem ter os seus limites e aplicações muito bem pensadas
(na prática, restringidas) de modo a impedir injustiças. Como a questão
das provas, para os utilitaristas, pende a favor dos acusados, o rigor
de Mill é um tanto, se não muito, matizado. A linguagem de Mill está
sempre carregada de um rigorismo moral que se encontra espalhado por
toda a sua obra. Este, no entanto, desigualmente distribuído, as
classes populares sendo beneficiárias de uma maior amplitude de
compreensão que as elites. As primeiras, fracas, desunidas e
subjugadas, são mentirosas (embora envergonhadas dessa característica)
e propensas a cometerem ações que acabam por infligir mais mal ainda a
elas próprias. Para as elites bem pensantes, restam os deveres e os
prazeres intelectualmente mais refinados. Apesar de Mill afirmar
que não se trata de restringir a individualidade ou de impedir
a experiência com novas e originais maneiras de se viver, constitui
sempre um problema até onde essas novas experiências de viver podem ser
estendidas. Na vida particular de cada pessoa praticamente não há
limites \textit{a priori}, mas como essa vida particular ocorre
conjuntamente com a social, os problemas levantados por Mill sobre os
limites dessa liberdade tornam"-se extremamente relevantes. Os ``vícios
morais'' de uma pessoa podem afetar apenas a ela e àqueles que
compartilharem dos mesmos gostos, mas os excessos de novas e originais
experiências podem se tornar perigosos, cair na boca do povo, por assim
dizer. Daí Mill propor a contenção, o autocontrole, aliados ao
trabalho extenuante e a diversões de alto nível (Mill foi um botânico
amador respeitado, que descobriu inclusive espécies não"-catalogadas) 
como adequadas à elite, para a qual ``o mais apaixonado
amor à virtude e ao autocontrole'' são, ou deveriam ser, qualidades
intrínsecas. Como para as qualidades ou virtudes pessoais vale o mesmo
que para os defeitos pessoais, isto é, elas só valem para cada pessoa,
então o amor à virtude e o autocontrole devem ter um valor social que
os ligue à sociedade. Virtudes são, quase por definição, totalmente
públicas, e cabe à elite tê"-las e mostrá"-las como exemplo e como
amostra da capacidade de liderança e, por consequência, de suas
obrigações para com a população e o país. Mill reconhece que esta sua
visão tem um caráter passadista, já que o ``pouco de
reconhecimento que a ideia da obrigação para com o público consegue na
moderna moralidade é derivado de fontes gregas e romanas''. Mill não
propõe um retorno completo ao ambiente moral da Antiguidade, mas
sim a preservação de aspectos dessa moralidade dentro de um novo mundo.
O que faltaria aos tempos atuais é o espírito público, sem o qual uma
sociedade não poderia nem se desenvolver nem perseverar. Na verdade, é
necessária a ampliação desse espírito. O último baluarte contra o
domínio das massas num mundo democratizado, reservado até então às
elites, deve passar em grande parte para a responsabilidade do próprio
povo. Sem esta passagem, tudo está perdido, pois nem a liberdade nem o
desenvolvimento sobreviverão. Que Mill não tenha uma opinião muito
favorável sobre a grande massa do povo em nada impede que ele veja na
educação desta o supremo dever dos que possuem a riqueza intelectual e
também – em menor escala – dos que têm sob o seu controle as riquezas
materiais. O caminho pode ser longo e árduo, mas é o único. E, dentre
as fraquezas morais da massa, algumas podem ser utilizadas para ajudar
o desenvolvimento moral e político delas próprias, pois, ``de
modo geral, a humanidade não somente é moderada no seu intelecto, mas
também nas suas inclinações''. As massas podem ser
modeladas através dos bons exemplos, que ajudam a criar uma
consciência. O povo, para Mill, não é mau, não procura ativamente fazer
o mal, apenas não tem uma consciência moral evoluída, pois ``não
é porque os desejos dos homens são fortes que eles agem mal, mas sim
porque as suas consciências são fracas''. Ensinado, treinado, o povo, as
pessoas em geral aprenderão a se comportar bem e a assumir
responsabilidades que lhes permitirão participar dos negócios públicos,
cada qual segundo a sua capacidade, de forma racional e que atente aos
seus verdadeiros interesses. Para isto é necessário educação e também
participação na direção dos negócios públicos de caráter mais local e
próximo ao cotidiano das pessoas.

\textit{Sobre a liberdade} fez sucesso a partir do momento em que foi
publicado pela primeira vez, e durante anos foi considerado, em
universidades inglesas e norte"-americanas, como o texto básico sobre
o liberalismo. As ideias de Mill, no entanto, não se deram bem durante
a maior parte do século \textsc{xx}, quando movimentos autoritários e
antiliberais, como o comunismo, o fascismo e o nazismo, atraíram a
atenção e levaram ao engajamento de milhões e milhões de pessoas numa
busca desenfreada por uma nova época, uma nova ordem, um novo ser
humano, busca que acabou resultando na morte de dezenas de milhões de
pessoas. De fato, num mundo em que o que se queria era simplesmente
extirpar os inimigos, fossem eles de raças ou de classes diferentes, 
defender que é  melhor e muito mais útil que todas as opiniões sejam
ouvidas e debatidas, num clima de liberdade de opinião e de ação, era
pedir para ser tido como velho e antiquado. John Stuart Mill foi então
rotulado como um economista burguês, como um vitoriano ingênuo que
acreditava no progresso, e como um mero liberal que defendia sua
posição de classe. Hoje em dia, quando as ameaças contra a liberdade
avultam de todos os lados, como aliás também nos dias de Mill, a
leitura e a consideração de \textit{Sobre a liberdade} pode ser um
meio de se entender e enfrentar os problemas atuais. A liberdade é
sempre uma questão espinhosa, e quase sempre os espinhos dela
constituem a liberdade dos outros, daqueles que não pensam como nós.
Mill entendeu isso como ninguém e se empenhou em passar essa mensagem
para as gerações futuras. Se durante um bom tempo parecia pouco
provável que essa mensagem ainda encontrasse leitores, talvez seja hoje
o momento de reavaliar essa impressão. 


