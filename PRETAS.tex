%Jorge: Teste
\SVN $Id: PRETAS.tex 7796 2010-11-24 19:29:08Z oliveira $
\begin{resumopage}

\item[John Stuart Mill] (Londres, 1806---Avignon, 1873), filósofo, 
economista e parlamentar inglês, é um dos mais importantes pensadores 
liberais do século \textsc{xix}. Na infância foi submetido a uma 
educação rigorosa por seu pai, James Mill, e posteriormente por 
Jeremy Bentham, fundador do utilitarismo. Rapidamente Mill 
reconhecerá no utilitarismo clássico uma ameaça à liberdade 
individual e empreenderá uma profunda reformulação em seus 
princípios, sem todavia jamais deixar de ser utilitarista. 
Conhece a obra de Auguste Comte em 1828, com o qual manterá 
uma intensa correspondência entre os anos de 1841 e 1846. 
Em 1851 desposa Harriet Taylor, famosa defensora dos direitos 
da mulher, e com a qual tinha mantido até então mais de vinte 
anos de amizade. Dentre as inúmeras obras que publicou vale 
destacar: \textit{Sistema de lógica dedutiva} (1843), 
\textit{Princípios de economia política} (1848) e \textit{Utilitarismo} (1861). 
Fortemente engajado em movimentos de emancipação da mulher, 
em 1865 é eleito para a Câmara dos Comuns, a qual será dissolvida 
em 1868, obrigando Mill a mudar para o sul da França, em Avignon. 

\item[Sobre a liberdade] (On Liberty, 1859) é uma defesa da individualidade 
e de sua autonomia diante da sociedade e do Estado. Para Mill, a “tirania 
da maioria” impõe uma homogeneidade que não é saudável para o desenvolvimento 
da sociedade. Os indivíduos são livres para fazerem o que quiserem, desde 
que não causem prejuízo aos seus semelhantes.

\item[Ari R. Tank Brito] é mestre em Filosofia pela Universidade 
de Varsóvia, Polônia (1990) e doutor pela Universidade de São Paulo (2007). 
Atualmente é professor da Universidade Federal do Mato Grosso. Organizou 
e traduziu, para a Coleção de Bolso Hedra, a \textit{Carta sobre a tolerância}, de John Locke. 
\end{resumopage}

